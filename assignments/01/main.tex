\documentclass[11pt,a4paper]{article}
\usepackage[czech]{babel}
\usepackage[margin=1in]{geometry}

\usepackage{../cfg}
\setcounter{section}{1}

\begin{document}

\title{\bb{Domácí úkol 1}}
\author{Jakub Adamec\\XP01TGR}

\maketitle

\begin{exercise}
Dokažte, nebo vyvraťte: Je dán neorientovaný graf $G$. Pak:
\vspace{-1em}
\begin{enumerate}[noitemsep]
    \item Každy uzavřený sled v $G$ liché délky obsahuje alespoň jednu kružnici liché délky.
    \item Každý uzavřený sled v $G$ sudé délky obsahuje alespoň jednu kružnici.
\end{enumerate}
\vspace{-1em}
Tj. buď jednotlivá tvrzení dokažte, nebo najděte protipříklad.
\end{exercise}

\begin{solution}
\titlebreak
\begin{enumerate}
    \item Důkaz matematickou indukcí podle délky $k$ daného sledu $W$.
    \begin{enumerate}
        \item Základní krok \\
        Nejmenší možná lichá délka uzavřeného sledu je $k=1$. Takový sled má tvar $v_0, e_1, v_0$, tedy smyčka, která je 
        z definice kružnicí délky 1. \\
        Pro grafy bez smyček je nejkratší uzavřený sled liché délky $k=3$ ve tvaru \\ 
        $v_0, e_1, v_1, e_2, v_2, e_3, v_0$, ten je kružnicí délky $3$. Pro obě tyto situace tvrzení platí.
        \item Indukční předpoklad \\
        Předpokládejme, že tvrzení platí pro všechny uzavřené sledy liché délky menší než $k$, kde $k$ je liché.
        \item Indukční krok \\
        Mějme uzavřený sled $W = (v_0, e_1, v_1, \dots, e_k, v_k)$, kde $v_0 = v_k$. Musíme ukázat, že $W$ obsahuje 
        kružnici liché délky. Rozlišme dva případy:
        \begin{itemize}
            \item Sled $W$ je kružnice. Pokud se ve sledu $W$ žádný vrchol (mimo $v_0=v_k$) neopakuje, pak je $W$ z 
            definice kružnicí, která má lichou délku $k$.
            \item Sled $W$ není kružnice. To tedy znamená, že se v něm nějaký vrchol musí opakovat, takže existují 
            indexy $i$ a $j$ takové, že $0 \leq i < j < k$ a $v_i = v_j$. Rozdělme sled $W$ na dvě částí:
            \begin{enumerate}[(1)]
                \item Uzavřený sled $W_1 = (v_i, e_{i+1}, v_{i+1}, \dots, e_j, v_j)$. Délka tohoto sledu je $l_1 = j-i$.
                \item Uzavřený sled $W_2 = (v_0, \dots, e_i, v_i, e_{j+1}, v_{j+1}, \dots, e_k, v_k)$. Délka tohoto 
                sledu je $l_2 = k - (j-i) = k - l_1$.
            \end{enumerate}
            Součet délek těchto dvou sledu je $l_1 + l_2 = k$. Víme, že $k$ je liché. Součet dvou celých čísel je lichý
            iff je jedno z nich liché a druhé sudé. Máme tedy dvě možnosti:
            \begin{enumerate}[(1)]
                \item Délka $l_1$ je lichá. Sled $W_1$ je uzavřený sled liché délky. Protože $l_1 = j-i > 0$ a zároveň 
                $j < k$, platí $0 < l_1 < k$. Našli jsme kratší uzavřený sled liché délky. Podle \ii{indukčního 
                předpokladu} musí sled $W_1$ obsahovat kružnici liché délky. Jelikož všechny hrany sledu $W_1$ jsou 
                zároveň hranami původního sledu $W$, pak i $W$ obsahuje tuto lichou kružnici.
                \item Délka $l_2$ je lichá. Sled $W_2$ je uzavřený sled liché délky. Protože $l_1 = j-i \geq 1$, platí 
                $l_2 = k-l_1 <k$. Opět jsme našli kratší uzavřený sled liché délky. Podle \ii{indukčního předpokladu} 
                musí sled $W_2$ obsahovat kružnici liché délky. Jelikož všechny hrany sledu $W_2$ jsou zároveň hranami
                původního sledu $W$, pak i $W$ obsahuje tuto lichou kružnici.
            \end{enumerate}
        \end{itemize}
        V obou případech jsme ukázali, že pokud sled $W$ není sám o sobě lichou kružnicí, lze v něm nalézt kratší 
        uzavřený sled liché délky, který podle indukčního předpokladu obsahuje lichou kružnici.
    \end{enumerate}
    \hspace{\fill} \qed
    \item Mějme graf $G = (V,E)$, $V=\bc{A,B,C}$ a $E=\bc{v_1=\bc{A,B}, v_2=\bc{B,C}}$:
    \begin{figure}[H]
        \centering
        \begin{tikzpicture}
            \graph []{
                A -- B -- C;
            };
        \end{tikzpicture}
    \end{figure}
    Vezměme si sled $A, v_1, B, v_2, C, v_2, B, v_1, A$. Určitě se jedná o uzavřený sled, protože začíná a končí ve 
    stejném vrcholu, zároveň jeho počet hran je sudý, jedná se tedy i o sudý sled. Kružnice je uzavřená cesta, tedy
    nesmí se v ní opakovat vrcholy. V našem sledu ale $2 \times$ vejdeme do vrcholu $B$. Takže se nejedná o kružnici.
    Našli jsme protipříklad, při kterém tvrzení neplatí. 
    \hspace{\fill}\qed
\end{enumerate}
\end{solution}

\begin{exercise}
Ukažte, že pro každá dvě kladná přirozená čísla $n, m$ splňující
\begin{equation}\label{eq:1}
    m \leq \frac{n(n-1)}{2}
\end{equation}
existuje prostý neorientovaný graf bez smyček s $n$ vrcholy a $m$ hranami. (To znamená, že popíšete způsob, jak byste
takový graf zkonstruovali.)
\end{exercise}

\begin{solution}
Mějme tedy $V=\bc{v_1, v_2, \dots, v_n}$. \\
V prostém neorientovaném grafu s $n$ vrcholy je maximální možný počet hran roven počtu všech možných dvojic vrcholů.
Taková situace odpovídá úplnému grafu a jeho počet hran je
\begin{equation}
    |E_{\max{}}| = \binom{n}{2} = \frac{n(n-1)}{2}
\end{equation}
Pro vytvoření $G \in \S$ stačí vybrat libovolných $m$ různých hran. Protože z předpokladu \eqref{eq:1} máme zaručeno, že
$m$ není větší, než celkový počet možných hran, tento výběr je vždy proveditelný.

Výsledný graf $G = (V,E)$, kde $E$ je námi vybraná množina $m$ hran, má zjevně $n$ vrcholů a přesně $m$ hran. Protože
jsme hrany vybírali jako páry různých vrcholů z úplného grafu, je výsledný graf z definice prostý (tj. každou dvojici
vrcholů spojuje nejvýše jedna hrana) a bez smyček (hrany spojují vždy dva různé vrcholy).
\end{solution}

\begin{exercise}
Je dán prostý neorientovaný graf $G = (V,E)$ bez smyček. Definujme jeho doplňovký graf $G^{dopl} =
(V, E^{dopl})$ takto: pro $u \not= v$ je
\[
    \bc{u, v} \in E^{dopl} \hspace{3mm} \text{ právě tehdy, když } \hspace{3mm} \bc{u, v} \not\in E.
\]
Existuje prostý neorientovaný graf $G$ bez smyček takový, že $G$ a $G^{dopl}$ jsou isomorfní (tj. liší se pouze
pojmenováním vrcholů)? Jestliže takový graf existuje, uveďte příklad takového grafu; jestliže takový graf neexistuje,
zdůvodněte to.
\end{exercise}

\begin{solution}
Mějme graf $G=(V,E)$, kde $V=\bc{1,2,3,4}$ a $E=\bc{\bc{1,3}, \bc{3,2}, \bc{2,4}}$:
\begin{figure}[H]
    \centering
    \begin{tikzpicture}
        \graph []{
            1 -- 3 -- 2 -- 4;
        };
    \end{tikzpicture}
\end{figure}
Když ke grafu zkonstruujme doplňkový, tj. $G^{dopl} = (V, E^{dopl})$, $E^{dopl} = \bc{\bc{2,1}, \bc{1,4}, \bc{4,3}}$:
\begin{figure}[H]
    \centering
    \begin{tikzpicture}
        \graph []{
            2 -- 1 -- 4 -- 3;
        };
    \end{tikzpicture}
\end{figure}
Je očividné, že se jedná o grafy vzájemně isomorfní.
\end{solution}

\end{document}