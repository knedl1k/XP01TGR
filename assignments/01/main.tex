\documentclass[11pt,a4paper]{article}
\usepackage[czech]{babel}
\usepackage[margin=1in]{geometry}

\usepackage{../cfg}

\begin{document}

\title{\bb{Domácí úkol 1}}
\author{Jakub Adamec\\XP01TGR}

\maketitle

\begin{exercise}
Dokažte, nebo vyvraťte: Je dán neorientovaný graf $G$. Pak:
\begin{enumerate}[noitemsep]
    \item Každy uzavřený sled v $G$ liché délky obsahuje alespoň jednu kružnici liché délky.
    \item Každý uzavřený sled v $G$ sudé délky obsahuje alespoň jednu kružnici.
\end{enumerate}
Tj. buď jednotlivá tvrzení dokažte, nebo najděte protipříklad.
\end{exercise}

\begin{solution}
\titlebreak
\begin{enumerate}
    \item Dokažme indukcí podle délky sledu $k$, kde $k$ je liché číslo.\\
    \ii{Základní krok:}
    Nejkratší možný uzavřený sled liché délky je sled délky 1. To je smyčka u vrcholu $v_1$. Takový sled, 
    $v_1, e_1, v_1$, je sám o sobě kružnicí délky 1, což je liché číslo.
    % \vspace{-1em}
    % \begin{itemize}
    %     \item Nejkratší možný uzavřený sled liché délky je sled délky 1. To je smyčka u vrcholu $v_1$. Takový sled,
    %     $v_1, e_1, v_1$, je sám o sobě kružnicí délky 1, což je liché číslo.
    % \end{itemize}

    Indukční předpoklad: Předpokládejme, že tvrzení platí pro všechny uzavřené sledy liché délky menší než $k$.\\
    \ii{Indukční krok:} Mějme uzavřený sled $W = (v_0, e_1, v_1, \dots, e_k, v_k)$, kde $v_0 = v_k$ a jeho délka
    $k$ je lichá.
    \item Mějme uzavřený sled $W$ sudé délky $k \geq 2$.\\
    Pak nám mohou nastat přesně dvě situace:
    \begin{enumerate}[(a)]
        \item Sled $W$ neobsahuje žádný opakující se vrchol mimo počáteční a koncový (aby byl uzavřený). Tím tedy přesně
        splňuje definici kružnice (uzavřené cesty) a pro něj tvrzení platí.
        \item Sled $W$ obsahuje alespoň jeden opakující se vnitřní vrchol.\\
        Ať $v_i = v_j$ je první opakující se vrchol ve sledu $W$ pro $0 \leq i < j$. To znamená, že podposloupnost
        $(v_i, e_{i+1}, \dots, v_{j-1})$ již žádný opakující se vrchol neobsahuje. Označme si teď podsled
        $C=(v_i, e_{i+1}, \dots, v_j)$. Pro $C$ platí:
        \begin{itemize}
            \item jedná se o uzavřený sled, protože začíná a končí ve stejném vrcholu,
            \item jeho délka je $j-i > 0$,
            \item protože jsme zvolili $v_j$ jako první výskyt opakovaného vrcholu po $v_i$, žádný z vrcholů
            $v_{i+1}, \dots, v_{j-1}$ se neopakuje a není roven $v_i$.
        \end{itemize}
        Tedy sled $C$ splňuje definici kružnice, takže jsme našli kružnici, která je obsažena v původním sledu $W$.
    \end{enumerate}
\end{enumerate}
\end{solution}

\newpage
\begin{exercise}
Ukažte, že pro každá dvě kladná přirozená čísla $n, m$ splňující
\begin{equation}\label{eq:1}
    m \leq \frac{n(n-1)}{2}
\end{equation}
existuje prostý neorientovaný graf bez smyček s $n$ vrcholy a $m$ hranami. (To znamená, že popíšete způsob, jak byste
takový graf zkonstruovali.)
\end{exercise}

\begin{solution}
Mějme tedy $V=\bc{v_1, v_2, \dots, v_n}$. \\
V prostém neorientovaném grafu s $n$ vrcholy je maximální možný počet hran roven počtu všech možných dvojic vrcholů.
Taková situace odpovídá úplnému grafu a jeho počet hran je
\begin{equation}
    |E_{\max{}}| = \binom{n}{2} = \frac{n(n-1)}{2}
\end{equation}
Pro vytvoření $G \in \S$ stačí vybrat libovolných $m$ různých hran. Protože z předpokladu \eqref{eq:1} máme zaručeno, že
$m$ není větší, než celkový počet možných hran, tento výběr je vždy proveditelný.

Výsledný graf $G = (V,E)$, kde $E$ je námi vybraná množina $m$ hran, má zjevně $n$ vrcholů a přesně $m$ hran. Protože
jsme hrany vybírali jako páry různých vrcholů z úplného grafu, je výsledný graf z definice prostý (tj. každou dvojici
vrcholů spojuje nejvýše jedna hrana) a bez smyček (hrany spojují vždy dva různé vrcholy).
\end{solution}

\begin{exercise}
Je dán prostý neorientovaný graf $G = (V,E)$ bez smyček. Definujme jeho doplňovký graf $G^{dopl} =
(V, E^{dopl})$ takto: pro $u \not= v$ je
\[
    \bc{u, v} \in E^{dopl} \hspace{3mm} \text{ právě tehdy, když } \hspace{3mm} \bc{u, v} \not\in E.
\]
Existuje prostý neorientovaný graf $G$ bez smyček takový, že $G$ a $G^{dopl}$ jsou isomorfní (tj. liší se pouze
pojmenováním vrcholů)? Jestliže takový graf existuje, uveďte příklad takového grafu; jestliže takový graf neexistuje,
zdůvodněte to.
\end{exercise}

\begin{solution}
Mějme graf $G=(V,E)$, kde $V=\bc{1,2,3,4}$ a $E=\bc{\bc{1,3}, \bc{3,2}, \bc{2,4}}$:
\begin{figure}[H]
    \centering
    \begin{tikzpicture}
        \graph []{
            1 -- 3 -- 2 -- 4;
        };
    \end{tikzpicture}
\end{figure}
Když ke grafu zkonstruujme doplňkový, tj. $G^{dopl} = (V, E^{dopl})$, $E^{dopl} = \bc{\bc{2,1}, \bc{1,4}, \bc{4,3}}$:
\begin{figure}[H]
    \centering
    \begin{tikzpicture}
        \graph []{
            2 -- 1 -- 4 -- 3;
        };
    \end{tikzpicture}
\end{figure}
Je očividné, že se jedná o grafy vzájemně isomorfní.
\end{solution}

\end{document}