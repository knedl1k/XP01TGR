\documentclass[11pt,a4paper]{article}
\usepackage[czech]{babel}
\usepackage[margin=1in]{geometry}

\makeatletter
\def\input@path{{../../configs/}{./}} 
\makeatother

\usepackage{logics}
\setcounter{section}{4}

\begin{document}

\title{\bb{Domácí úkol 4}}
\author{Jakub Adamec\\XP01TGR}

\maketitle

\begin{exercise}
Najděte příklad orientovaného grafu \bb{se dvěma tranzitivními redukcemi o různém počtu hran}, který má nejmenší počet vrcholů.
\end{exercise}
% \begin{solution}
% Ať graf $G$ má $n$ vrcholů, pak
% \vspace{-1em}
% \begin{itemize}
%     \item \ii{pro $n=1$ nebo $n=2$} platí, že jediný netriviální případ je graf $a \leftrightarrow b$, jehož unikátní redukce je on sám.
%     \item \ii{pro $n=3$} platí, že aby redukce nebyla unikátní, musí graf mít silně souvislou komponentu. Uvažme tedy úplný graf $K_3$. Jeho tranzitivní redukce jsou minimální grafy, jejichž uzávěr je $K_3$. Takové redukce existují přesně dvě:
%     \begin{figure}[H]
%         \centering
%         \begin{minipage}[c]{0.1\textwidth}
%             \begin{tikzpicture}[scale=1]
%                 \graph [layered layout]{
%                     a -> b -> c -> a;
%                 };
%             \end{tikzpicture}
%         \end{minipage}%
%         \hspace{0.05\textwidth}
%         \begin{minipage}[c]{0.1\textwidth}
%             \begin{tikzpicture}[scale=1]
%                 \graph [layered layout]{
%                     a -> c -> b -> a;
%                 };
%             \end{tikzpicture}
%         \end{minipage}
%     \end{figure}
%     \vspace{-1em}
%     Obě tyto redukce ale mají stejný počet hran, tedy nesplňují podmínku různého počtu hran.
%     \item \ii{pro $n=4$} zkusme vzít úplný graf $K_4$. Tento graf má 12 hran. Tranzitivní uzávěr $K_4$ je $K_4$ sám. Hledáme tedy minimální silně souvislé grafy na 4 vrcholech, jejižch uzávěr je $K_4$.
%     \begin{enumerate}[1)]
%         \item \ii{Tranzitivní redukce $\R_1$} je cyklus délky 4.
%         \begin{figure}[H]
%             \centering
%             \begin{tikzpicture}[scale=1]
%                 \graph [spring layout, node distance=1cm]{
%                     a -> b;
%                     b -> c;
%                     c -> d;
%                     d -> a;
%                 };
%             \end{tikzpicture}
%         \end{figure}
%         \vspace{-1em}
%         Tento graf je očividně minimální a jeho tranzitivním uzávěrem je $K_4$.
%         \item \ii{Tranzitivní redukce $\R_2$} je graf tvořený cyklem délky 3 a cyklem délky 2, které sdílejí jeden vrchol.
%         \vspace{-1em}
%         \begin{figure}[H]
%             \centering
%             \begin{tikzpicture}[scale=1]
%                 \graph [spring layout, node distance=1cm]{
%                     a -> {b, d};
%                     c -> a;
%                     d -> a;
%                     b -> c;
%                 };
%             \end{tikzpicture}
%         \end{figure}
%         \vspace{-1em}
%         Tento graf je také silně souvislý a jeho uzávěrem je $K_4$. Je také minimální.
%     \end{enumerate}
%     Zjistili jsme tedy, že jakýkoliv graf $G$, jehož uzávěr je $K_4$ má dvě tranzitivní redukce $\R_1$ a $\R_2$ s počty hran $|E_1| = 4$ a $|E_2| = 5$. A protože jsme ukázali, že pro $n < 4$ takový graf neexistuje, a zároveň $4 \not= 5$, pak nejmenší možný počet vrcholů je 4.
% \end{itemize}
% \end{solution}

\begin{solution}
    Aby graf měl více než jednu tranzitivní redukci, musí obsahovat cykly, protože tranzitivní redukce orientovaného acyklického grafu je unikátní. Zaměřme se na silně souvislé komponenty.
    \vspace{-1em}
    \begin{itemize}
        \item \ii{Pro $n=1$ nebo $n=2$}:
        Jediné silně souvislé grafy jsou triviální vrchol nebo cyklus $a \leftrightarrow b$. V obou případech je redukce unikátní (samotný cyklus).
        
        \item \ii{Pro $n=3$}:
        Uvažujme silně souvislý graf na 3 vrcholech. Aby byl graf tranzitivní redukcí, nesmí obsahovat hranu, která je \enquote{zkratkou} pro cestu z jiných hran.
        Na 3 vrcholech je jediným minimálním silně souvislým grafem \ii{cyklus délky 3}. Jakýkoliv graf se 4 a více hranami na 3 vrcholech obsahuje \enquote{tětivu}, kterou lze vynechat, aniž by se porušila souvislost (a tedy tranzitivní uzávěr). Ať už má graf $G$ jakékoliv hrany, jeho tranzitivní redukce budou vždy cykly délky 3. Všechny redukce tedy mají stejný počet hran $|E|=3$. Podmínka zadání není splněna.
        \item \ii{Pro $n=4$}:
        Zkonstruujme graf $G$, jehož tranzitivní uzávěr je úplný graf $K_4$. Hledáme minimální silně souvislé podgrafy na 4 vrcholech.
        Zkusme dva strukturálně odlišné případy:
        \begin{enumerate}[1)]
            \item \ii{Hamiltonovský cyklus}: Graf tvořený jedním cyklem procházejícím všemi vrcholy.
            \begin{equation*}
                 R_1: a \to b \to c \to d \to a
            \end{equation*}
            Tento graf je minimální a má 4 hrany.
            \item \ii{Dva cykly sdílející vrchol}: Graf tvořený cyklem délky 3 ($a,b,c$) a cyklem délky 2 ($a,d$), které sdílejí vrchol $a$.
            \begin{equation*}
                 R_2: (a \to b \to c \to a) \cup (a \leftrightarrow d)
            \end{equation*}
             Tento graf je silně souvislý a je minimální, má $3 + 2 = 5$ hran.
        \end{enumerate}
        \begin{figure}[H]
            \centering
            \begin{minipage}[t]{0.45\textwidth}
                \centering
                \begin{tikzpicture}[auto, node distance=1.5cm, >=latex]
                    \node[] (a) {$a$};
                    \node[] (b) [right of=a] {$b$};
                    \node[] (c) [below of=b] {$c$};
                    \node[] (d) [below of=a] {$d$};
                    \draw[->] (a) -- (b);
                    \draw[->] (b) -- (c);
                    \draw[->] (c) -- (d);
                    \draw[->] (d) -- (a);
                    \node at (0.75, -2) {$|E|=4$};
                \end{tikzpicture}

                Redukce $R_1$
            \end{minipage}
            \hfill
            \begin{minipage}[t]{0.45\textwidth}
                \centering
                \begin{tikzpicture}[auto, node distance=1.5cm, >=latex]
                    \node[] (a) {$a$};
                    \node[] (b) [right of=a] {$b$};
                    \node[] (c) [below of=b] {$c$};
                    \node[] (d) [below of=a] {$d$};
                    \draw[->] (a) -- (b);
                    \draw[->] (b) -- (c);
                    \draw[->] (c) -- (a);
                    \draw[->, bend left=15] (a) to (d);
                    \draw[->, bend left=15] (d) to (a);
                    \node at (0.75, -2) {$|E|=5$};
                \end{tikzpicture}
                
                Redukce $R_2$
            \end{minipage}
        \end{figure}
        \vspace{-1em}
        Pokud zvolíme $G$ jako sjednocení $R_1 \cup R_2$, pak $R_1$ i $R_2$ jsou jeho tranzitivními redukcemi.
    \end{itemize}
    \vspace{-1em}
    Protože $4 \neq 5$, nalezli jsme hledaný příklad. Jelikož pro $n < 4$ takový graf neexistuje, je $n=4$ nejmenší možný počet vrcholů.
    \hspace{\fill}\qed
\end{solution}

\begin{exercise}
Je dán prostý neorientovaný graf $G = (V,E)$ bez smyček s $n$ vrcholy, kde $n$ je sudé. Dokažte, nebo vyvraťte:

Jestliže každý vrchol grafu $G$ má stupeň $d = \frac{n}{2}$, pak $G$ je úplný bipartitní graf se stranami o $\frac{n}{2}$ vrcholech.
\end{exercise}
\begin{solution}
Ověřme situaci pro $n=6$. Počet vrcholů je sudý. Požadovaný stupeň každého vrcholu $d = 3$, tedy hledáme 3-regulární graf na 6 vrcholech.

Tvrzení říká, že každo takový graf musí být $K_{\frac{n}{2}, \frac{n}{2}}$, takže v tomto případě $K_{3,3}$. Zkusme najít 3-regulární graf na 6 vrcholech, který \ii{není} $K_{3,3}$.

Ať $G = (V,E)$, kde $V = \bc{a,b, \dots, f}$ a $E = \{\bc{a,b}, \bc{a,d}, \bc{b,c}, \bc{b,e}, \bc{c,a}, \bc{c,f}, \bc{d,e}, $ \\ $ \bc{e,f}, \bc{f,d}\}$.
\vspace{-1em}
\begin{figure}[H]
    \centering
    \begin{tikzpicture}[scale=1]
        \graph [spring layout, node distance=2cm]{
            a -- {b, d};
            b -- {c, e};
            c -- {a, f};
            d -- e;
            e -- f;
            f -- d;
        };
    \end{tikzpicture}
\end{figure}
\vspace{-1em}
Ověřme, že graf $G$ splňuje podmínky.
\vspace{-1em}
\begin{enumerate}[1)]
    \item Nemá smyčky ani paralelní hrany.
    \item Má sudý počet vrcholů, $n = 6$.
    \item Ověřme stupně jednotlivých vrcholů:
    \vspace{-0.5em}
    \begin{itemize}
        \item $d(a) = 3$
        \item $d(b) = 3$
        \item $d(c) = 3$
        \item $d(d) = 3$
        \item $d(e) = 3$
        \item $d(f) = 3$
    \end{itemize}
    \vspace{-0.5em}
    Každý vrchol má stupeň $d = 3 = \frac{6}{2}$.
\end{enumerate}
\vspace{-1em}
Pak by $G$ měl být $K_{3,3}$, ověřme. Z definice bipartitního grafu víme, že \ii{nesmí} obsahovat kružnici liché délky. Náš graf $G$ obsahuje dokonce dvě kružnice délky tři, například
\begin{figure}[H]
    \centering
    \begin{tikzpicture}[scale=1]
        \graph [spring layout, node distance=1cm]{
            d -- e;
            e -- f;
            f -- d;
        };
    \end{tikzpicture}
\end{figure}
Takže $G$ není bipartitní, tudíž nemůže být ani úplným bipartitním grafem $K_{3,3}$.
\hspace{\fill}\qed
\end{solution}
\newpage
\begin{exercise}
Je dán prostý souvislý graf $G = (V,E)$ bez smyček s $n \geq 3$ vrcholy. Nechť $x$ a $y$ jsou dva vrcholy grafu, které nejsou spojeny hranou (tj. $\bc{x,y} \not\in E$) a takové, že $d_G(x) + d_G(y) \geq n$. Dokažte, nebo vyvraťte:

V $G$ existuje hamiltonovská kružnice právě tehdy, když v $G + \bc{x,y}$ existuje hamiltonovská kružnice. \\
(Graf $G + \bc{x,y}$ má stejnou množinu vrcholů jako $G$ a množinu hran rovnu $E \cup \bc{\bc{x,y}}$.)
\end{exercise}
\begin{solution}
\titlebreak

\enquote{$\Rightarrow$}: Předpokládejme, že $G = (V,E)$ má hamiltonovskou kružnici $H$. \\
Hamiltonovská kružnice $H$ je podgraf $G$, který obsahuje všechny vrcholy $V$ a cyklus tvořený hranami z $E$. Graf $G + \bc{x,y}$ má stejnou množinu vrcholů $V$ a množinu hran $E^\prime = E \cup \bc{\bc{x,y}}$. Takže každá hrana kružnice $H$, která je v $E$, je také v $E^\prime$. A proto $H$ je také hamiltonovskou kružnicí v grafu $G + \bc{x,y}$.

\enquote{$\Leftarrow$}: Předpokládejme, že $G + \bc{x,y}$ má hamiltonovskou kružnici $H^\prime$. \\
Rozdělme na dvě situace.
\vspace{-1em}
\begin{enumerate}[1)]
    \item \ii{Kružnice $H^\prime$ neobsahuje hranu $\bc{x,y}$}. Pak všechny její hrany musí pocházet z původní množiny $E$. A protože $H^\prime$ obsahuje všechny vrcholy $V$, je $H^\prime$ hamiltonovskou kružnicí i v grafu $G$.
    \item \ii{Kružnice $H^\prime$ obsahuje hranu $\bc{x,y}$}. V takovém případě, když hranu $\bc{x,y}$ z $H^\prime$ odebereme, dostaneme hamiltonovskou cestu $P$ v grafu $G$. Tato cesta vede z $x$ do $y$, respektive naopak, a navštíví všechny vrcholy $G$. Označme vrcholy na této cestě $P = (v_1, v_2, \dots, v_n)$, kde $v_1 = x$ a $v_n = y$. Všechny hrany $\bc{v_i, v_{i+1}}$, pro $i=1, \dots, n-1$, leží v $E$.

    Definujme množiny indexů sousedů $x$ a $y$ v $G$:
    \begin{align}
        S &= \bc{i \, \middle| \, 1 \leq i \leq n-1 \text{ a } \bc{v_1, v_{i+1} \in E}} \\
        T &= \bc{i \, \middle| \, 1 \leq i \leq n-1 \text{ a } \bc{v_i, v_n \in E}}
    \end{align}
    Takže velikost $S$ je počet sousedů $v_1$. Protože $\bc{v_1, v_n} \not\in E$, $v_n$, tak všichni sousedé $x$ musí být na cestě $P$. Počet takových sousedů je $|S| = d_G(x)$. Obdobně platí, že velikost $T$ je počet sousedů $v_n$, a protože $\bc{v_n, v_1} \not\in E$, tak $|T| = d_G(y)$.

    $S$ i $T$ jsou podmnožinami množiny indexů $\bc{1, 2, \dots, n-1}$, která má $n-1$ prvků. Hledáme $i$ takové, že $i \in S$ a také $i \in T$, tj. $i \in S \cap T$. \\ Předpokládejme spor, takže $S \cap T = \emptyset$. Víme, že platí
    \begin{equation}
        |S \cup T| = |S| + |T| - |S \cap T|.
    \end{equation}
    Dosazením získáme 
    \begin{equation}
        |S \cup T| = d_G(x) + d_G(y) - 0 = d_G(x) + d_G(y).
    \end{equation}
    Ze zadání víme, že $d_G(x) + d_G(y) \geq n$, z čehož plyne $|S \cup T| \geq n$. Což je nutně spor, protože $S \cup T$ je podmnožinou množiny $\bc{1, 2, \dots, n-1}$, která má $n-1$ prvků. Tudíž nutně platí $S \cap T \not= \emptyset$. Existuje tedy alespoň jeden index $i$, který je v obou množinách. Díky tomuto indexu $i$ jsme našli způsob, jak \enquote{překřížit} a spojit konce hamiltonovské cesty $P$ pomocí hran, které zaručeně leží v $G$, čímž jsme vytvořili kružnici $H$.
\end{enumerate}
Obě dvě strany ekvivalence jsou pravdivé, a tedy tvrzení je také pravdivé.
\hspace{\fill}\qed
\end{solution}

\end{document}
