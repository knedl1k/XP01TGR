\documentclass[11pt,a4paper]{article}
\usepackage[czech]{babel}
\usepackage[margin=1in]{geometry}

\usepackage{../cfg}
\setcounter{section}{4}

\begin{document}

\title{\bb{Domácí úkol 4}}
\author{Jakub Adamec\\XP01TGR}

\maketitle

\begin{exercise}
    Najděte příklad orientovaného grafu \bb{se dvěma tranzitivními redukcemi o různém počtu hran}, který má nejmenší 
    počet vrcholů.
\end{exercise}
\begin{solution}
    % Již víme, že graf $G$ se dvěma tranzitivními redukcemi o různém počtu hran musí obsahovat cyklus, protože acyklické 
    % grafy mají tranzitivní redukce vždy unikátní.
    Ať graf $G$ má $n$ vrcholů, pak
    \vspace{-1em}
    \begin{itemize}
        \item \ii{pro $n=1$ nebo $n=2$} platí, že jediný netriviální případ je graf $a \leftrightarrow b$, jehož 
        unikátní redukce je on sám.
        \item \ii{pro $n=3$} platí, že aby redukce nebyla unikátní, musí graf mít silně souvislou komponentu. Uvažme 
        tedy plný graf $K_3$. Jeho tranzitivní redukce jsou minimální grafy, jejichž uzávěr je $K_3$. Takové redukce 
        existují přesně dvě:
        \begin{figure}[H]
            \centering
            \begin{minipage}[c]{0.1\textwidth}
                \begin{tikzpicture}[scale=1]
                    \graph [layered layout]{
                        a -> b -> c -> a;
                    };
                \end{tikzpicture}
            \end{minipage}%
            \hspace{0.05\textwidth}
            \begin{minipage}[c]{0.1\textwidth}
                \begin{tikzpicture}[scale=1]
                    \graph [layered layout]{
                        a -> c -> b -> a;
                    };
                \end{tikzpicture}
            \end{minipage}
        \end{figure}
        \vspace{-1em}
        Obě tyto redukce ale mají stejný počet hran, tedy nesplňují podmínku různého počtu hran.
        \item \ii{pro $n=4$} zkusme vzít plný graf $K_4$. Tento graf má 12 hran. Tranzitivní uzávěr $K_4$ je $K_4$ sám.
        Hledáme tedy minimální silně souvislé grafy na 4 vrcholech, jejižch uzávěr je $K_4$.
        \begin{enumerate}[1)]
            \item \ii{Tranzitivní redukce $\R_1$} je cyklus délky 4.
            \begin{figure}[H]
                \centering
                \begin{tikzpicture}[scale=1]
                    \graph [spring layout, node distance=1cm]{
                        a -> b;
                        b -> c;
                        c -> d;
                        d -> a;
                    };
                \end{tikzpicture}
            \end{figure}
            \vspace{-1em}
            Tento graf je očividně minimální a jeho tranzitivním uzávěrem je $K_4$.
            \item \ii{Tranzitivní redukce $\R_2$} je graf tvořený cyklem délky 3 a cyklem délky 2, které sdílejí jeden 
            vrchol.
            \vspace{-1em}
            \begin{figure}[H]
                \centering
                \begin{tikzpicture}[scale=1]
                    \graph [spring layout, node distance=1cm]{
                        a -> {b, d};
                        c -> a;
                        d -> a;
                        b -> c;
                    };
                \end{tikzpicture}
            \end{figure}
            \vspace{-1em}
            Tento graf je také silně souvislý a jeho uzávěrem je $K_4$. Je také minimální.
        \end{enumerate}
        Zjistili jsme tedy, že jakýkoliv graf $G$, jehož uzávěr je $K_4$ má dvě tranzitivní redukce $\R_1$ a $\R_2$ s 
        počty hran $|E_1| = 4$ a $|E_2| = 5$. A protože jsme ukázali, že pro $n < 4$ takový graf neexistuje, a zároveň 
        $4 \not= 5$, pak nejmenší možný počet vrcholů je 4.
    \end{itemize}
\end{solution}

\begin{exercise}
    Je dán prostý neorientovaný graf $G = (V,E)$ bez smyček s $n$ vrcholy, kde $n$ je sudé. Dokažte, nebo vyvraťte:

    Jestliže každý vrchol grafu $G$ má stupeň $d = \frac{n}{2}$, pak $G$ je úplný bipartitní graf se stranami o 
    $\frac{n}{2}$ vrcholech.
\end{exercise}
\begin{solution}
    
\end{solution}

\begin{exercise}
    Je dán prostý souvislý graf $G = (V,E)$ bez smyček s $n \geq 3$ vrcholy. Nechť $x$ a $y$ jsou dva vrcholy grafu, 
    které nejsou spojeny hranou (tj. $\bc{x,y} \not\in E$) a takové, že $d(x) + d(y) \geq n$. Dokažte, nebo vyvraťte:

    V $G$ existuje hamiltonovská kružnice právě tehdy, když v $G + \bc{x,y}$ existuje hamiltonovská kružnice. \\
    (Graf $G + \bc{x,y}$ má stejnou množinu vrcholů jako $G$ a množinu hran rovnu $E \cup \bc{\bc{x,y}}$.)
\end{exercise}
\begin{solution}

\end{solution}

\end{document}