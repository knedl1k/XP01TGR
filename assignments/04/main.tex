\documentclass[11pt,a4paper]{article}
\usepackage[czech]{babel}
\usepackage[margin=1in]{geometry}

\usepackage{../cfg}
\setcounter{section}{4}

\begin{document}

\title{\bb{Domácí úkol 4}}
\author{Jakub Adamec\\XP01TGR}

\maketitle

\begin{exercise}
    Najděte příklad orientovaného grafu \bb{se dvěma tranzitivními redukcemi o různém počtu hran}, který má nejmenší 
    počet vrcholů.
\end{exercise}
\begin{solution}

\end{solution}

\begin{exercise}
    Je dán prostý neorientovaný graf $G = (V,E)$ bez smyček s $n$ vrcholy, kde $n$ je sudé. Dokažte, nebo vyvraťte:

    Jestliže každý vrchol grafu $G$ má stupeň $d = \frac{n}{2}$, pak $G$ je úplný bipartitní graf se stranami o 
    $\frac{n}{2}$ vrcholech.
\end{exercise}
\begin{solution}
    
\end{solution}

\begin{exercise}
    Je dán prostý souvislý graf $G = (V,E)$ bez smyček s $n \geq 3$ vrcholy. Nechť $x$ a $y$ jsou dva vrcholy grafu, 
    které nejsou spojeny hranou (tj. $\bc{x,y} \not\in E$) a takové, že $d(x) + d(y) \geq n$. Dokažte, nebo vyvraťte:

    V $G$ existuje hamiltonovská kružnice právě tehdy, když v $G + \bc{x,y}$ existuje hamiltonovská kružnice. \\
    (Graf $G + \bc{x,y}$ má stejnou množinu vrcholů jako $G$ a množinu hran rovnu $E \cup \bc{\bc{x,y}}$.)
\end{exercise}
\begin{solution}

\end{solution}

\end{document}