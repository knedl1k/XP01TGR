\documentclass[11pt,a4paper]{article}
\usepackage{polyglossia}
\setdefaultlanguage{czech}
\usepackage[margin=1in]{geometry}

\usepackage{../cfg}
\setcounter{section}{6}

\begin{document}

\title{\bb{Domácí úkol 6}}
\author{Jakub Adamec\\XP01TGR}

\maketitle

\begin{exercise}
    Je dán bipartitní graf $G$ se stranami $X$ a $Y$ a v něm maximální párování $P_{\max}$. Pro každou hranu $e \in P_{\max}$ vybereme jeden její krajní vrchol do množiny $A$ takto:

    Pro hranu $e = \bc{x,y} \in P_{\max} (x \in X, \, y \in Y)$ do $A$ vybereme vrchol $y$, jestliže existuje cesta $a_1, e_1, a_2, e_2, \dots, e_{2k+1}, y$, kde
    \vspace{-1em}
    \begin{itemize}
        \item $a_1 \in X$ je volný v párování $P_{\max}$,
        \item pro $i > 0$ je $e_{2i-1} \not\in P_{\max}$, $i > 0$, $e_{2i} \in P_{\max}$.
    \end{itemize}
    \vspace{-1em}
    Jestliže taková cesta neexistuje, do $A$ vybereme vrchol $x$.\\
    Dokažte nebo vyvraťte: Množina $A$ (zkonstruovaná výše) je vrcholovým pokrytím grafu $G$.
\end{exercise}
\begin{solution}
    Definujme si \ii{dosažitelné} vrcholy jako takové, ke kterým existuje střídavá cesta (přesně ta ze zadání) z nějakého volného vrcholu $X$.

    Mějme libovolnou hranu $e = \bc{u,v}$, kde $u \in X$ a $v \in Y$. Musíme dokázat, že $e$ je pokryta, tedy buď $u \in A$, nebo $v \in A$. Mohou nastat dva případy:
    \vspace{-1em}
    \begin{enumerate}[(a)]
        \item Hrana $e$ \ii{leží} v párování $P_{\max}$. Konstrukce množiny $A$ vybírá z každé hrany $P_{\max}$ právě jeden vrchol. Takže hrana je určitě pokryta.
        \item Hrana $e$ \ii{neleží} v párování $P_{\max}$. Sporem. Předpokládejme, že hrana $\bc{u, v}$ není pokryta. To by znamenalo, že $u \not\in A$ a zároveň $v \not\in A$. Takže:
        \vspace{-0.5em}
        \begin{itemize}
            \item Pokud je $u$ \ii{volný}, pak je triviálně dosažitelný (je počátkem střídavé cesty délky 0). Zároveň platí $u \not\in A$, protože $A$ obsahuje pouze vrcholy z párování.
            \item Pokud je $u$ \ii{spárovaný} s nějakým $y' \in Y$ (hrana $\bc{u, y'} \in P_{\max}$), pak z předpokladu $u \not\in A$ plyne, že do $A$ musel být vybrán vrchol $y'$. To podle definice $A$ nastane jen tehdy, je-li $y'$ dosažitelný. Pokud je $y'$ dosažitelný, pak prodloužením cesty přes hranu $\bc{y', u}$ je dosažitelný i vrchol $u$.
        \end{itemize}
        V obou případech jsme ukázali, že
        \begin{equation}
            u \not\in A \implies u \text{ je dosažitelný}.
        \end{equation}
        Využijeme hranu $e=\bc{u,v}$, která neleží v párování:
        Jelikož $u$ je dosažitelný a $\bc{u,v} \notin P_{\max}$, můžeme střídavou cestu končící v $u$ prodloužit o hranu $\bc{u,v}$ do vrcholu $v$. Tedy i $v$ je dosažitelný. Kdyby byl $v$ volný, našli bychom střídavou cestu z volného do volného vrcholu (zlepšující cestu), což je spor s maximalitou $P_{\max}$. Vrchol $v$ je tedy \ii{nutně} spárovaný s nějakým $x'$. Protože je $v$ dosažitelný a spárovaný, podmínka konstrukce $A$ říká, že pro hranu $\bc{x', v}$ vybereme do $A$ vrchol $v$.
        
        A to je spor s předpokladem $v \not\in A$. Tedy alespoň jeden z vrcholů $u,v$ musí ležet v $A$.
    \end{enumerate}
    \vspace{-1em}
    Množina $A$ je vrcholovým pokrytím.
    \hspace{\fill}\qed
\end{solution}

\begin{exercise}
    Dokažte, že v každém bipartitním grafu $G$ platí
    \begin{equation}
        \alpha_1(G) = \beta_0(G).
    \end{equation}
    Hint: Použijte Königovu větu pro určení počtu hran v maximálním párování v bipartitním grafu.
\end{exercise}
\begin{solution}
    $\alpha_1(G)$ je počet hran v maximálním párování, tj. $\alpha_1(G) = |P_{\max}|$ a $\beta_0(G)$ je počet vrcholů v minimálním vrcholovém pokrytí. Abychom mohli použit \ii{Königovu} větu, musíme ukázat
    \begin{equation}
        \beta_0(G) = \min_{A \subseteq X}(|X \setminus A| + |V_G(A)|).
    \end{equation}
    Vezměme libovolnou podmnožinu $A \subseteq X$. Sestavme množinu vrcholů $K$, která se skládá ze dvou částí:
    \vspace{-1em}
    \begin{enumerate}[(a)]
        \item vrcholy z $X$, které \ii{nejsou} v $A$ (tj. $X \setminus A$),
        \item sousedé vrcholů z $A$ v množině $Y$ (tj. $V_G(A)$).
    \end{enumerate}
    \vspace{-1em}
    Takže
    \begin{equation}
        K = (X \setminus A) \cup V_G(A).
    \end{equation}
    Velikost této množiny je pak $|K| = |X \setminus A| + |V_G(A)|$.

    Musíme ověřit, že $K$ pokrývá každou hranu v grafu. Mějme libovolnou hranu $e = \bc{x,y}$, kde $x \in X$ a $y \in Y$. Mohou nastat pouze tyto situace:
    \vspace{-1em}
    \begin{enumerate}[(a)]
        \item Vrchol $x$ \ii{nepatří} do $A$. Pak nutně $x \in (X \setminus A)$. Tím pádem je $x$ v naší množině $K$. Hrana je pokryta.
        \item Vrchol $x$ \ii{patří} do $A$. Protože $x \in A$ a $y$ je jeho soused, musí dle definice platit, že $y \in V_G(A)$. Tím pádem je $y$ v naší množině $K$. Hrana je pokryta.
    \end{enumerate}
    \vspace{-1em}
    V obou případech je alespoň jeden vrchol hrany v množině $K$. Takže $K$ je vrcholové pokrytí.

    Teď stačí ukázat
    \begin{equation}
        |P_{\max}| = |K|.
    \end{equation}
    Protože $K$ je vrcholové pokrytí, pak nejmenší takové $K$ odpovídá velikosti minimálního vrcholového pokrytí, tedy $\beta_0(G)$.
    Takže platí
    \begin{equation}
        \min_{A \subseteq X}(|X \setminus A| + |V_G(A)|) = \beta_0(G).
    \end{equation}
    Což odpovídá přesnému znění \ii{Königovy} věty:
    \begin{equation}
        \alpha_1(G) = |P_{\max}| = \min_{A \subseteq X}(|X \setminus A| + |V_G(A)|) = \beta_0(G)
    \end{equation}
    \hspace{\fill}\qed
\end{solution}

\end{document}        
