\documentclass[11pt,a4paper]{article}
\usepackage{polyglossia}
\setdefaultlanguage{czech}
\usepackage[margin=1in]{geometry}

\makeatletter
\def\input@path{{../../configs/}{./}} 
\makeatother

\usepackage{logics}
\setcounter{section}{6}

\begin{document}

\title{\bb{Domácí úkol 6}}
\author{Jakub Adamec\\XP01TGR}

\maketitle

\begin{exercise}
    Je dán bipartitní graf $G$ se stranami $X$ a $Y$ a v něm maximální párování $P_{\max}$. Pro každou hranu $e \in P_{\max}$ vybereme jeden její krajní vrchol do množiny $A$ takto:

    Pro hranu $e = \bc{x,y} \in P_{\max} (x \in X, \, y \in Y)$ do $A$ vybereme vrchol $y$, jestliže existuje cesta $a_1, e_1, a_2, e_2, \dots, e_{2k+1}, y$, kde
    \vspace{-1em}
    \begin{itemize}
        \item $a_1 \in X$ je volný v párování $P_{\max}$,
        \item pro $i > 0$ je $e_{2i-1} \not\in P_{\max}$, $i > 0$, $e_{2i} \in P_{\max}$.
    \end{itemize}
    \vspace{-1em}
    Jestliže taková cesta neexistuje, do $A$ vybereme vrchol $x$.\\
    Dokažte nebo vyvraťte: Množina $A$ (zkonstruovaná výše) je vrcholovým pokrytím grafu $G$.
\end{exercise}
\begin{solution}
    Definujme si \ii{dosažitelné} vrcholy jako takové, ke kterým existuje střídavá cesta (přesně ta ze zadání) z nějakého volného vrcholu $X$.

    Mějme libovolnou hranu $e = \bc{u,v}$, kde $u \in X$ a $v \in Y$. Musíme dokázat, že $e$ je pokryta, tedy buď $u \in A$, nebo $v \in A$. Mohou nastat dva případy:
    \vspace{-1em}
    \begin{enumerate}[(a)]
        \item Hrana $e$ \ii{leží} v párování $P_{\max}$. Konstrukce množiny $A$ vybírá z každé hrany $P_{\max}$ právě jeden vrchol. Takže hrana je určitě pokryta.
        \item Hrana $e$ \ii{neleží} v párování $P_{\max}$. Sporem. Předpokládejme, že hrana $\bc{u, v}$ není pokryta. To by znamenalo, že $u \not\in A$ a zároveň $v \not\in A$. Takže:
        \vspace{-0.5em}
        \begin{itemize}
            \item Pokud je $u$ \ii{volný}, pak je triviálně dosažitelný (je počátkem střídavé cesty délky 0). Zároveň platí $u \not\in A$, protože $A$ obsahuje pouze vrcholy z párování.
            \item Pokud je $u$ \ii{spárovaný} s nějakým $y' \in Y$ (hrana $\bc{u, y'} \in P_{\max}$), pak z předpokladu $u \not\in A$ plyne, že do $A$ musel být vybrán vrchol $y'$. To podle definice $A$ nastane jen tehdy, je-li $y'$ dosažitelný. Pokud je $y'$ dosažitelný, pak prodloužením cesty přes hranu $\bc{y', u}$ je dosažitelný i vrchol $u$.
        \end{itemize}
        V obou případech jsme ukázali, že
        \begin{equation}
            u \not\in A \implies u \text{ je dosažitelný}.
        \end{equation}
        Využijeme hranu $e=\bc{u,v}$, která neleží v párování:
        Jelikož $u$ je dosažitelný a $\bc{u,v} \notin P_{\max}$, můžeme střídavou cestu končící v $u$ prodloužit o hranu $\bc{u,v}$ do vrcholu $v$. Tedy i $v$ je dosažitelný. Kdyby byl $v$ volný, našli bychom střídavou cestu z volného do volného vrcholu (zlepšující cestu), což je spor s maximalitou $P_{\max}$. Vrchol $v$ je tedy \ii{nutně} spárovaný s nějakým $x'$. Protože je $v$ dosažitelný a spárovaný, podmínka konstrukce $A$ říká, že pro hranu $\bc{x', v}$ vybereme do $A$ vrchol $v$.
        
        A to je spor s předpokladem $v \not\in A$. Tedy alespoň jeden z vrcholů $u,v$ musí ležet v $A$.
    \end{enumerate}
    \vspace{-1em}
    Množina $A$ je vrcholovým pokrytím.
    \hspace{\fill}\qed
\end{solution}

\begin{exercise}
    Dokažte, že v každém bipartitním grafu $G$ platí
    \begin{equation}
        \alpha_1(G) = \beta_0(G).
    \end{equation}
    Hint: Použijte Königovu větu pro určení počtu hran v maximálním párování v bipartitním grafu.
\end{exercise}
\begin{solution}
    Označme $\alpha_1(G)$ velikost maximálního párování a $\beta_0(G)$ velikost minimálního vrcholového pokrytí.
    
    Nejprve si uvědomme, že pro libovolný graf platí nerovnost:
    \begin{equation}
        \alpha_1(G) \leq \beta_0(G).
    \end{equation}
    Je to proto, že každá hrana z maximálního párování musí být pokryta alespoň jedním vrcholem a tyto hrany jsou disjunktní.

    Nyní musíme dokázat opačnou nerovnost, tedy $\alpha_1(G) \geq \beta_0(G)$. K tomu využijeme \ii{Königovu} větu v jejím alternativním tvaru, která říká:
    \begin{equation}
        \alpha_1(G) = \min_{A \subseteq X}(|X \setminus A| + |V_G(A)|).
    \end{equation}
    
    Uvažujme libovolnou podmnožinu $A \subseteq X$ a sestrojme množinu vrcholů $K$ takto:
    \begin{equation}
        K = (X \setminus A) \cup V_G(A).
    \end{equation}
    Velikost této množiny je $|K| = |X \setminus A| + |V_G(A)|$.

    Ověříme, že $K$ je vrcholové pokrytí. Mějme libovolnou hranu $e = \bc{x,y}$, kde $x \in X$ a $y \in Y$.
    \vspace{-1em}
    \begin{enumerate}[(a)]
        \item Pokud $x \notin A$, pak $x \in (X \setminus A) \subseteq K$. Hrana je pokryta.
        \item Pokud $x \in A$, pak z definice sousedství $y \in V_G(A) \subseteq K$. Hrana je pokryta.
    \end{enumerate}
    \vspace{-1em}
    Množina $K$ je tedy vždy vrcholovým pokrytím. 
    
    Protože $\beta_0(G)$ je velikost \textit{minimálního} vrcholového pokrytí, musí pro každé takto sestrojené $K$ platit:
    \begin{equation}
        |K| \geq \beta_0(G).
    \end{equation}
    Tato nerovnost platí pro libovolné $A$, tedy i pro to, které minimalizuje výraz z Königovy věty. Dostáváme:
    \begin{equation}
        \alpha_1(G) = \min_{A \subseteq X}(|X \setminus A| + |V_G(A)|) \geq \beta_0(G).
    \end{equation}
    
    Máme tedy $\alpha_1(G) \leq \beta_0(G)$ a zároveň $\alpha_1(G) \geq \beta_0(G)$. Z toho nutně plyne:
    \begin{equation}
        \alpha_1(G) = \beta_0(G).
    \end{equation}
    \hspace{\fill}\qed
\end{solution}

\end{document}        
