\documentclass[11pt,a4paper]{article}
\usepackage[czech]{babel}
\usepackage[margin=1in]{geometry}

\usepackage{../cfg}
\setcounter{section}{6}

\begin{document}

\title{\bb{Domácí úkol 6}}
\author{Jakub Adamec\\XP01TGR}

\maketitle

\begin{exercise}
    Je dán bipartitní graf $G$ se stranami $X$ a $Y$ a v něm maximální párování $P_{\max}$. Pro každou hranu $e \in 
    P_{\max}$ vybereme jeden její krajní vrchol do množiny $A$ takto:

    Pro hranu $e = \bc{x,y} \in P_{\max} (x \in X, \, y \in Y)$ do $A$ vybereme vrchol $y$, jestliže existuje cesta 
    $a_1, e_1, a_2, e_2, \dots, e_{2k+1}, y$, kde
    \vspace{-1em}
    \begin{itemize}
        \item $a_1 \in X$ je volný v párování $P_{\max}$,
        \item pro $i > 0$ je $e_{2i-1} \not\in P_{\max}$, $i > 0$, $e_{2i} \in P_{\max}$.
    \end{itemize}
    Jestliže taková cesta neexistuje, do $A$ vybereme vrchol $x$.

    Dokažte nebo vyvraťte: Množina $A$ (zkonstruovaná výše) je vrcholovým pokrytím grafu $G$.
\end{exercise}
\begin{solution}

\end{solution}

\begin{exercise}
    Dokažte, že v každém bipartitním grafu $G$ platí
    \begin{equation}
        a_1(G) = \beta_0(G).
    \end{equation}
    Hint: Použijte Königovu větu pro určení počtu hran v maximálním párování v bipartitním grafu.
\end{exercise}
\begin{solution}

\end{solution}
\end{document}