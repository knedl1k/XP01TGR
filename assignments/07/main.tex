\documentclass[11pt,a4paper]{article}
\usepackage{polyglossia}
\setdefaultlanguage{czech}
\usepackage[margin=1in]{geometry}

\usepackage{../cfg}
\setcounter{section}{7}

\begin{document}

\title{\bb{Domácí úkol 7}}
\author{Jakub Adamec\\XP01TGR}

\maketitle

\begin{exercise}
    Dokažte:

    Každý řez souvislého neorientovaného grafu je disjuktiním sjednocením cutsetů.
\end{exercise}
\begin{solution}
    Nechť $G = (V,E)$ je souvislý neorientovaný graf. Víme, že symetrická diference dvou řezů je opět řez. Důkaz provedeme indukcí dle velikosti řezu $|K|$. Nechť $K$ je libovolný řez.
    \begin{enumerate}[1)]
        \item \ii{Základní krok}. Je-li $K$ minimální (vzhledem k inkluzi), pak je $K$ z definice cutset a tvrzení triviálně platí (sjednocení jedné množiny).
        \item \ii{Indukční předpoklad}. Předpokládejme, že $K$ není minimální. Pak existuje vlastní podmnožina $C \subset K$, která je cutsetem.
        \item \ii{Indukční krok}. Protože $C \subset K$, platí $K \setminus C = K \oplus C$. Jelikož symetrická diference dvou řezů je řez, je i $K'$ řezem.

        Uvažujme množinu $K' = K \setminus C$. Protože $C \subset K$, platí $K \setminus C = K \oplus C$. Jelikož $K$ i $C$ jsou řezy a symetrická diference řezů je řez, je i $K'$ řezem.
        
        Získali jsme rozklad $K = C \cup K'$, kde $C$ a $K'$ jsou disjunktní. Protože $|K'| < |K|$, můžeme na $K'$ aplikovat indukční předpoklad. Tedy $K'$ lze rozložit na sjednocení cutsetů $C_2 \cup \dots \cup C_k$.
    \end{enumerate}
    Celkově pak $K = C \cup C_2 \cup \dots \cup C_k$, což je disjunktní sjednocení cutsetů.
    \hspace{\fill}\qed
\end{solution}

\begin{exercise}
    Dokažte, nebo vyvraťte:

    Množina $K \subseteq E$ souvislého grafu $G$ s množinou hran $E$ je kružnice právě tehdy, když je to minimální množina hran, pro kterou platí
    \begin{equation}
        K \cap (E \setminus T) \not= \emptyset
    \end{equation}
    pro každou kostu $T$ grafu $G$.
\end{exercise}
\begin{solution}
    \titlebreak
    \begin{itemize}
        \item[$\Rightarrow$:] Předpokládejme, že množina $K \subseteq E$ souvislého grafu $G$ s množinou hran $E$ je kružnice.

        Kostra $T$ je z definice strom, což znamená, že je acyklickým grafem. Protože $K$ je kružnice, nemůže být podmnožinou žádné kostry. Tím pádem musí platit, že $K$ má s doplňkem kostry $(E \setminus T)$ neprázdný průnik. Dále víme, že kružnice bez jedné hrany tvoří cestu (případně les), což je acyklický graf. Každý acyklický podgraf v souvislém grafu lze doplnit na kostru. Existuje tedy kostra $T^\prime$, která obsahuje celou množinu $K^\prime = K \setminus \bc{e}$, kde $e$ je libovolná hrana (tj. $K^\prime \subseteq T^\prime$). To znamená, že $K^\prime \cap (E \setminus T^\prime) = \emptyset$. Takže množina $K^\prime$ podmínku nesplňuje, $K$ je proto minimální.
        \item[$\Leftarrow$:] Předpokládejme, že $K$ je minimální a splňuje podmínku $K \not\subseteq T$ pro každou kostru $T$.

        Víme, že množina hran $E$ je podmnožinou nějaké kostry právě tehdy, když $E$ neobsahuje kružnici (tj. je acyklická). Jelikož $K$ není podmnožinou žádné kostry, musí $K$ obsahovat alespoň jednu kružnici $C$. Jinak by totiž šla doplnit na kostru, což by byl spor s podmínkou. Takže $C \subseteq K$. A protože $C$ je kružnice, podle první části důkazu víme, že $C$ sama o sobě splňuje podmínku (průnik s doplňkem každé kostry je neprázdný). Kdyby $K$ obsahovala kromě kružnice $C$ ještě nějaké další hrany, nebyla by minimální, protože už její vlastní podmnožina $C$ podmínku splňuje. My ale předpokládáme minimální $K$, tudíž $K = C$.   
    \end{itemize}
    Množina $K$ je skutečně kružnicí právě tehdy, když je minimální množinou, která má neprázdný průnik s doplňkem každé kostry grafu $G$.
    \hspace{\fill}\qed
\end{solution}
\end{document}
