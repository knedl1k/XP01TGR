\documentclass[11pt,a4paper]{article}
\usepackage[czech]{babel}
\usepackage[margin=1in]{geometry}

\usepackage{../cfg}
\setcounter{section}{5}

\begin{document}

\title{\bb{Domácí úkol 5}}
\author{Jakub Adamec\\XP01TGR}

\maketitle

\begin{exercise}
    Dokažte: Je dána síť $G=(V,E)$, se zdrojem $s$, spotřebičem $t$ a omezeními $l$ a $c$. Dále je dán přípustný tok $f$ 
    a množina $A \subset V$ taková, že $s \in A$, $t \not\in A$. Pak platí
    \begin{equation}
        \vel(f) = \sum_{e \in W^+(A)}f(e) - \sum_{e \in W^-(A)}f(e).
    \end{equation}
    Postupujte indukcí podle počtu vrcholů množiny $A$.
\end{exercise}
\begin{solution}
    
\end{solution}

\begin{exercise}
    Je dán prostý souvislý neorientovaný graf $G$ bez smyček a se sudým počtem vrcholům, který neobsahuje $K_{1,3}$ jako 
    indukovaný podgraf. Pak v $G$ existuje perfektní párování.
\end{exercise}
\begin{solution}
    
\end{solution}

\begin{exercise}
    Je dán prostý 3-regulární graf $G$ bez smyček, tj. každý vrchol grafu $G$ má stupeň 3. Dokažte, nebo vyvraťte:
    \vspace{-1em}
    \begin{center}
        Jestliže $G$ nemá most, pak v něm existuje perfektní párování.    
    \end{center}
    \vspace{-1em}
    Hint: K důkazů použijte Tutteho větu.
\end{exercise}
\begin{solution}
    
\end{solution}
\end{document}