\documentclass[11pt,a4paper]{article}
\usepackage[czech]{babel}
\usepackage[margin=1in]{geometry}

\usepackage{../cfg}
\setcounter{section}{5}

\begin{document}

\title{\bb{Domácí úkol 5}}
\author{Jakub Adamec\\XP01TGR}

\maketitle

\begin{exercise}
    Dokažte: Je dána síť $G=(V,E)$, se zdrojem $s$, spotřebičem $t$ a omezeními $l$ a $c$. Dále je dán přípustný tok $f$ a množina $A \subset V$ taková, že $s \in A$, $t \not\in A$. Pak platí
    \begin{equation}\label{5.1}
        \vel(f) = \sum_{e \in W^+(A)}f(e) - \sum_{e \in W^-(A)}f(e).
    \end{equation}
    Postupujte indukcí podle počtu vrcholů množiny $A$, $|A| = k$.
\end{exercise}
\begin{solution}
    \titlebreak
    \begin{enumerate}[1)]
        \item \ii{Základní krok}. Nechť $k=1$. Protože $s \in A$, musí platit, že $A = \bc{s}$. V tomto případě je množina $W^+(A)$ tvořena všemi hranami vycházejícimi ze zdroje $s$ a množina $W^-(A)$ všemi hranami vcházejícimido zdroje $s$. Podle definice velikosti toku $\vel(f)$ platí
        \begin{equation}
            \vel(f) = \sum_{v \in V} f(s,v) - \sum_{v \in V}f(v,s).
        \end{equation}
        Což přesně odpovídá definici:
        \begin{equation}
            \vel(f) = \sum_{e \in W^+(\bc{s})}f(e) - \sum_{e \in W^-(\bc{s})}f(e)
        \end{equation}
        Tvrzení tedy pro $k=1$ platí.
        \item \ii{Indukční předpoklad}. Předpokládejme, že tvrzení platí pro libovolnou množinu $A$ velikosti $k$ obsahující $s$ a neobsahující $t$.
        \item \ii{Indukční krok}. Mějme množinu $A'$ velikosti $k+1$, která obsahuje $s$ a neobsahuje $t$. Zvolme libovolný vrchol $v \in A'$ takový, že $v \neq s$. Takový vrchol jistě existuje, protože $|A'| \ge 2$. Položme $A = A' \setminus \{v\}$. Množina $A$ má velikost $k$, obsahuje $s$ a neobsahuje $t$, proto pro ni dle indukčního předpokladu platí vztah \eqref{5.1}.
        
        Nyní zkoumejme rozdíl mezi toky přes řezy určené množinami $A$ a $A'$. Přidáním vrcholu $v$ do množiny $A$ se změní sumy následovně:
        \begin{itemize}
            \item Hrany z $A$ do $v$, které byly původně v $W^+(A)$, se stanou vnitřními hranami $A'$, tedy ze sumy zmizí (odečteme tok vtékající do $v$ z $A$).
            \item Hrany z $v$ do $A$, které byly původně v $W^-(A)$, se stanou vnitřními, tedy ze sumy zmizí (přičteme tok vytékající z $v$ do $A$, protože v $W^-$ byl s minusem).
            \item Hrany z $v$ do $V \setminus A'$, které dříve nebyly v řezu, se stanou součástí $W^+(A')$ (přičteme tok vytékající z $v$ ven).
            \item Hrany z $V \setminus A'$ do $v$, které dříve nebyly v řezu, se stanou součástí $W^-(A')$ (odečteme tok vtékající do $v$ zvenčí).
        \end{itemize}
        
        Matematicky vyjádřeno:
        \begin{align}
            \begin{split}
            \sum_{e \in W^+(A')}f(e) - \sum_{e \in W^-(A')}f(e) &= \left( \sum_{e \in W^+(A)}f(e) - 
            \sum_{u \in A} f(u,v) + \sum_{w \notin A'} f(v,w) \right) \\
            &\quad - \left( \sum_{e \in W^-(A)}f(e) - \sum_{u \in A} f(v,u) + \sum_{w \notin A'} f(w,v) \right)
            \end{split}
        \end{align}
        
        Seskupíme-li členy odpovídající původnímu toku a členy týkající se vrcholu $v$, dostáváme:
        \begin{equation}
            \text{Tok}(A') = \vel(f) + \underbrace{\left( \sum_{u \in A} f(v,u) + 
            \sum_{w \notin A'} f(v,w) \right)}_{\text{celkový tok z } v} - \underbrace{\left( \sum_{u \in A} f(u,v) + 
            \sum_{w \notin A'} f(w,v) \right)}_{\text{celkový tok do } v}
        \end{equation}
        
        Jelikož $v \neq s$ a $v \neq t$ (protože $t \notin A'$), platí pro vrchol $v$ Kirchhoffův zákon zachování toku:
        \begin{equation}
            \sum_{z \in V} f(v,z) - \sum_{z \in V} f(z,v) = 0.
        \end{equation}
        
        Členy v závorkách výše pokrývají všechny hrany incidentní s $v$ (buď vedou z/do $A$, anebo z/do $V \setminus A'$). Rozdíl v závorce je tedy roven nule. Proto platí:
        \begin{equation}
            \sum_{e \in W^+(A')}f(e) - \sum_{e \in W^-(A')}f(e) = \vel(f).
        \end{equation}
        \hspace{\fill}\qed
    \end{enumerate}
\end{solution}

\begin{exercise}
    Je dán prostý souvislý neorientovaný graf $G$ bez smyček a se sudým počtem vrcholů, který neobsahuje $K_{1,3}$ jako indukovaný podgraf. Pak v $G$ existuje perfektní párování.
\end{exercise}
\begin{solution}
    \ii{Sporem}. Z Tutteovy věty víme, že pokud graf nemá perfektní párování, existuje podmnožina vrcholů $S \subset V$ taková, že počet lichých komponent grafu $G \setminus S$ je větší než počet vrcholů v $S$. Tedy předpokládáme:
    \begin{equation}\label{5.8}
        g(G \setminus S) > |S|,
    \end{equation}
    kde $g(G \setminus S)$ je počet komponent souvislosti $G \setminus S$, které mají lichý počet vrcholů.

    Víme, že celkový počet vrcholů $|V|$ je sudý. Platí vztah:
    \begin{equation}
        |V| = |S| + \sum_{C \in \text{komponenty } G \setminus S} |V(C)|
    \end{equation}
    Protože $|V(C)|$ je liché pro liché komponenty a sudé pro sudé, parita $|V|$ je stejná jako parita $|S| + g(G \setminus S)$. Aby byl součet sudý, musí mít $|S|$ a $g(G \setminus S)$ stejnou paritu. Z nerovonosti \eqref{5.8} pak vyplývá silnější podmínka
    \begin{equation}\label{5.10}
        g(G \setminus S) \geq |S| + 2.
    \end{equation}

    Označme $k = g(G \setminus S)$ a liché komponenty jako $C_1, C_2, \dots, C_k$. Uvažujme bipartitní graf $H$, kde jedna strana je množina $S$ a druhá strana jsou komponenty $\mathcal{C} = \bc{C_1, \dots, C_k}$. Hrana mezi $s \in S$ a $C_i \in \mathcal{C}$ v grafu $H$ existuje právě tehdy, když v původním grafu $G$ existuje hrana mezi vrcholem $s$a nějakým vrcholem $v \in C_i$.
    \begin{itemize}
        \item Protože $G$ je souvislý, každá komponenta $C_i$ musí být spojena s alespoň jedním vrcholem z $S$. Tedy $d_H(C_i) \geq 1$ pro všechna $i$.
        \item Zkoumejme stupně vrcholů $s \in S$ v grafu $H$. Předpokládejme, že existuje vrchol $s \in S$, který je spojen s třemi (nebo více) různými komponentami, \ii{BÚNO} řekněme $C_1, C_2, C_3$.
        \item To znamená, že existují vrcholy $v_1 \in C_1, v_2 \in C_2, v_3 \in C_3$, které jsou sousedy $s$ v $G$.
        \item Protože $v_1, v_2, v_3$ leží v různých komponentách grafu $G \setminus S$, neexistuje mezi nimi žádná hrana. Množina $\{v_1, v_2, v_3\}$ je tedy nezávislá.
        \item Podgraf indukovaný množinou $\{s, v_1, v_2, v_3\}$ by pak tvořil $K_{1,3}$ (se středem $s$). To je ale ve sporu s předpokladem, že $G$ neobsahuje $K_{1,3}$.
    \end{itemize}
    Takže musí platit, že každý vrchol $s \in S$ sousedí s nejvýše dvěma lichými komponentami. Tedy $d_H(s) \leq 2$.

    Alespoň části grafu $H$, které spojují jednotlivé komponenty, jsou souvislé. Zároveň vrcholy z $\mathcal{C}$ mají stupeň alespoň 1 a vrcholy z $S$ nejvýše 2. Takový bipartitní graf je tvořen cestami a cykly, kde se střídají vrcholy z $S$ a $\mathcal{C}$. Abychom maximalizovali počet vrcholů z $\mathcal{C}$ vůči $S$, musí $H$ vypadat jako cesta začínající a končící vrcholem z $\mathcal{C}$, například
    \begin{figure}[H]
        \centering
        \begin{tikzpicture}
            \graph []{
                "$C_1$" -- "$s_1$" -- "$C_2$" -- "$s_2$" -- "$C_3$";
            };
        \end{tikzpicture}
    \end{figure}
    V tomto maximálním případě platí
    \begin{equation}
        |\mathcal{C}| \leq |S| + 1.
    \end{equation}
    Což je ale ve sporu s podmínkou \eqref{5.10}. Předpoklad, že neexistuje perfektní párování, byl chybný. Graf $G$ tedy perfektní párování má.
    \hspace{\fill}\qed
\end{solution}

\begin{exercise}
    Je dán prostý 3-regulární graf $G$ bez smyček, tj. každý vrchol grafu $G$ má stupeň 3. Dokažte, nebo vyvraťte:
    \vspace{-1em}
    \begin{center}
        Jestliže $G$ nemá most, pak v něm existuje perfektní párování.    
    \end{center}
    \vspace{-1em}
    Hint: K důkazů použijte Tutteho větu.
\end{exercise}
\begin{solution}
    \ii{Sporem}. Předpokládejme, že $G$ nemá perfektní párování. Z Tutteovy věty víme, že pokud graf nemá perfektní párování, existuje podmnožina vrcholů $S \subset V$ taková, že počet lichých komponent grafu $G \setminus S$ je větší než počet vrcholů v $S$. Tedy předpokládáme:
    \begin{equation}
        g(G \setminus S) > |S|,
    \end{equation}
    kde $g(G \setminus S)$ je počet komponent souvislosti $G \setminus S$, které mají lichý počet vrcholů. Pro pohodlnost označme $k = g(G \setminus S)$.

    Nechť $C_1, C_2, \dots C_k$ jsou liché komponenty vzniklé odebráním množiny $S$ od $G$. Pro libovolnou komponentu $C_i$ označme $E(C_i, S)$ množinu hran spojujících tuto komponentu s $S$. Počet těchto hran označme $m_i = |E(C_i, S)|$.
    
    Sečtěme stupně všech vrcholů uvnitř komponenty $C_i$. Protože graf je 3-regulární, nutně platí
    \begin{equation}
        \sum_{v \in V(C_i)} d_G(v) = 3 |V(C_i)|.
    \end{equation}
    Zároveň můžeme tuto sumu vyjádřit pomocí vnitřních hran a hran vedoucích ven z komponenty do $S$ jako
    \begin{equation}
        \sum_{v \in V(C_i)} d_G(v) = 2 |E(C_i)| + m_i.
    \end{equation}
    Protože $G$ je souvislý a $m_i$ je liché, musí platit $m_i \geq 1$. Pokud by $m_i = 1$, pak by tato jediná hrana byla mostem. Ze zadání ale víme, že $G$ nemá mosty, takže nutně platí
    \begin{equation}
        m_i \geq 3 \quad \forall i \in \bc{1, \dots, k}.
    \end{equation}
    Celkový počet hran vedoucích z lichých komponent do $S$ je
    \begin{equation}
        M = \sum_{i=1}^k m_i \geq \sum_{i=1}^k 3 = 3k.
    \end{equation}
    Tyto hrany musí být incidentní s vrcholy v $S$. Jelikož každý vrchol v $S$ má v grafu $G$ stupeň 3, nemůže do $S$ vcházet více než $3|S|$ hran, i kdyby všechny hrany z $S$ vedly do lichých komponent a žádné mezi vrcholy $S$ navzájem nebo do sudých komponent. Takže platí
    \begin{equation}
        M \leq 3|S|.
    \end{equation}
    Spojením nerovností dostáváme
    \begin{align}
        \begin{split}
            3k &\leq M \leq 3|S| \\
            3k &\leq 3|S| \\
            k &\leq |S|
        \end{split}
    \end{align}
    To je však ve sporu s předpokladem pro neexistenci perfektního párování, graf $G$ tedy perfektní párování má.
    \hspace{\fill}\qed
\end{solution}
\end{document}
