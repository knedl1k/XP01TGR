\documentclass[11pt,a4paper]{article}
\usepackage[czech]{babel}
\usepackage[margin=1in]{geometry}

\usepackage{../cfg}
\setcounter{section}{3}

\begin{document}

\title{\bb{Domácí úkol 3}}
\author{Jakub Adamec\\XP01TGR}

\maketitle

\begin{exercise}
    Je dáno číslo $n \geq 5$. Je možné pro každé takové $n$ zkonstruovat 2-souvislý prostý neorientovaný graf $G$ bez 
    smyček, který má
    \vspace{-1em}
    \begin{itemize}[noitemsep]
        \item průměr $\diam(G)$ roven 2,
        \item a má $2n-5$ hran?
    \end{itemize}
    \vspace{-1em}
    Jestliže ano, pro každé $n$ takový graf zkonstruujte; jestliže ne, zdůvodněte, proč takový graf nemůže existovat.
\end{exercise}
\begin{solution}
    Nejdříve ověřme, že tvrzení platí pro $n=5$. Zvolme $C_5$ jako kružnici na 5 vrcholech. Protože je kružnicí, tak je 
    určitě 2-souvislá, protože nemá artikulaci a je souvislá. Má 5 vrcholů a 5 hran, což odpovídá $2 \cdot 5 - 5 = 5$ 
    počtu hran. Průměr $\diam(C_5) = \floor*{\frac{5}{2}} = 2$ také sedí zadání.

    Pro $n \geq 6$ zkonstruujme graf $G_n$ tak, že vezmeme 5 vrcholů $v_1, v_2, \dots, v_5$ a spojíme je do kružnice 
    $C_5$. Máme tedy 5 vrcholů a 5 hran. Označme si zbývající počet vrcholů jako $k = n - 5$. Přidejme zbývající vrcholy
    $u_1, u_2, \dots, u_k$. Každý z těchto $k$ vrcholů připojíme dvěma hranami k jedné konkrétní dvojici vrcholů, na 
    $C_5$. Tato dvojice vrcholů nesmí být vzájemně sousední. Vyberme například dvojici $v_1$ a $v_3$. Pak pro 
    $i=1, 2, \dots, k$ přidáme hrany $(u_i, v_1)$ a $(u_i, v_3)$. Teď máme $n$ vrcholů a $5+2k$ hran.

    Ověřme, zda takto zkonstruovaný graf splňuje požadavky.
    \vspace{-1em}
    \begin{enumerate}[1)]
        \item \bb{Počet hran}. Máme $5+2k$ hran, tedy $5 + 2(n-5) = 2n-5$ hran, to jsme přesně chtěli.
        \item \bb{2-souvislost}. Tzn. musíme ukázat, že graf nemá artikulaci. Ověřme tedy indukcí: Odeberme $u_i$; graf 
        $G_{n-1}$, který zbyde, má stejnou strukturu, jen o jeden připojený vrchol méně. Postupujme dále až k $n=6$, to
        nám zbyde původní $C_5$ s jedním $u_1$, který je očividně souvislý. Odebrání $u_i$ tedy graf nerozpojí.

        Zkusme odebrat $v_1$: Zbytek $C_5$ je cesta $v_2, v_3, v_4, v_5$. Všechny vrcholy $u_i$ jsou stále připojeny k
        $v_3$. Celý zbytek grafu $G_n \setminus v_1$ je tedy souvislý. Analogicky se ukáže pro $v_3$.
        
        Anebo odeberme $v_2$, $v_4$, nebo $v_5$. BÚNO vyberme $v_2$. Graf $G_n \setminus v_2$ je cesta 
        $v_1, v_5, v_4, v_3$. Všechny vrcholy $u_i$ jsou připojeny k $v_1$ i $v_3$. Tedy $G_n \setminus v_2$ je 
        souvislý.

        Graf je tedy určitě 2-souvislý, protože je souvislý a nemá artikulaci.
        \item \bb{Průměr $\diam(G)=2$}. Dva vrcholy jsou buď vzájemní sousedé, anebo mají společného souseda. 
        
        Jestliže oba vrcholy leží na $C_5$, tak jsme již v případě $n=5$ dokázali, že $\diam(C_5) = 2$. 
        Jestliže oba vrcholy jsou z připojených vrcholů $u_i$, pak mají zaručeně společné sousedy $v_1$ a $v_3$, tedy
        $d(u_i, u_j)=2$, pro $i \not= j$. Pokud je jeden z $C_5$ a druhý z $u_i$, pak:
        \vspace{-0.8em}
        \begin{itemize}
            \item je jeden z vrcholů $v_1$, respektive $v_3$, pak jsou sousedé, $d=1$.
            \item je jeden z vrcholů $v_2$, pak mají společné sousedy $v_1$ a $v_3$, $d(v_2, u_i) = 2$.
            \item je jeden z vrcholů $v_4$, pak mají společného souseda $v_3$, $d(v_4, u_i) = 2$. Obdobně pro $v_5$.
        \end{itemize}
        Takže průměr grafu je 2.
    \end{enumerate}
    Ano, pro každé $n \geq 5$ je možné takový graf zkonstruovat.
    \hspace{\fill}\qed
\end{solution}

\begin{exercise}
    Dokažte nebo vyvraťe:\\
    Je dán prostý souvislý neorientovaný graf $G$ bez smyček s $n \geq 4$ vrcholy, který neobsahuje jako \bb{indukovaný 
    podgraf} úplný bipartitní graf $K_{1,3}$. Pak v $G$ existují dva sousední vrcholy $x,y$ takové, že graf $G \setminus
    \bc{x,y}$ je také souvislý. (Graf $G \setminus \bc{x,y}$ je podgraf $G$, ze kterého jsme odstranili vrcholy $x$ a 
    $y$, nejen hranu s krajními vrcholy $x$ a $y$.)

    ($K_{1,3}$ je úplný bipartitní graf se stranami o 1 a 3 vrcholech.)
\end{exercise}
\begin{solution}
    Sporem. Ať existuje graf $G$, který je prostý, souvislý, neorientovaný, $n \geq 4$ a neobsahuje indukovaný 
    $K_{1,3}$, pro který platí, že pro \ii{každou} dvojici sousedních vrcholů $u,v$ je graf $G \setminus \bc{u,v}$ 
    nesouvislý. 
    
    Ať $(x,y)$ je libovolná hrana v $G$. Dle našeho předpokladu je $G \setminus \bc{x,y}$ nesouvislý. To znamená, že 
    množinu vrcholů $V \setminus \bc{x,y}$ lze rozdělit na dvě neprázdné, disjunktní množiny $A$ a $B$ tak, že mezi $A$ 
    a $B$ nevedou žádné hrany. Zaměřme se na situaci, kdy existuje nějaká hrana \ii{uvnitř} $A$ nebo $B$. 
    Předpokládejme, že v $A$ existuje hrana $(a_1, a_2)$. Dle našeho hlavního předpokladu musí platit, že 
    $G \setminus \bc{a_1, a_2}$ je také nesouvislý. Graf $G \setminus \bc{a_1, a_2}$ ale obsahuje vrcholy $x$, $y$, 
    které jsou spojeny hranou, a celou neprázdnou množinu $B$. Protože původní graf $G$ byl souvislý, každý vrchol 
    $b \in B$ musel být v $G$ spojen s $\bc{x,y}$, protože nemohl být spojen s $A$. Stejně tak každý vrchol 
    $a \in A \setminus \bc{a_1, a_2}$ musí být spojen s $\bc{x,y}$. $(x,y)$ je tedy most. To znamená, že 
    $G \setminus \bc{a_1, a_2}$ je souvislý. Což je spor s hlavním předpokladem, takže pro libovolnou hranu $(x,y) 
    \in E(G)$ musí platit, že $G \setminus \bc{x,y}$ je nezávislá množina.

    Teď ověřme, že tento důsledek může koexistovat s našimi podmínkami $n \geq 4$ a že $G$ neobsahuje indukovaný 
    $K_{1,3}$. Protože $G$ je souvislý a $n \geq 4$, musí nutně obsahovat alespoň jeden vrchol $x$ se stupněm 
    $\deg(x) \geq 2$. Vyberme dva různé sousedy $x$, označme $y$ a $z$. Mějme hrany $(x,y)$ a $(x,z)$. Co když hrana 
    $(y,z)$ (ne)existuje?
    \vspace{-1em}
    \begin{enumerate}[1)]
        \item Hrana neexistuje. Protože $n \geq 4$, musí existovat alespoň jeden další vrchol $w \not= x,y,z$. Dle 
        prvního důkazu již víme, že $(z,w) \not\in E$ a $(y,w) \not\in E$. Protože $G$ je souvislý, $w$ musí mít alespoň 
        jednoho souseda. Jeho jediný možný soused je $x$, tedy $(x,w) \in E$. To ale znamená, že máme indukovaný 
        $K_{1,3}$ graf s centrem $x$, což je spor s předpokladem, že $G$ je bez $K_{1,3}$. Hrana tedy musí existovat.
        \item Hrana existuje. Již víme, že toto je pravda. Zároveň víme, že $G \setminus \bc{y,z}$ \ii{musí} být 
        nezávislá množina. Protože $n \geq 4$, musí existovat 4. vrchol $w$. A protože $G$ je souvislý, musí $w$ být 
        spojen s nějakým z vrcholů $x,y,z$. Pokud je $w$ spojen s $y$ nebo $z$, je $G \setminus \bc{y,z}$ stále souvislý
        (přes $x$). Předpokládejme tedy, že $w$ je spojen \ii{pouze} s $x$. Ale vrcholy $x$ i $w$ jsou obsaženy v 
        $G \setminus \bc{y,z}$, hrana $(x,w)$ je tedy hrana v $G \setminus \bc{y,z}$, a to je spor s tím, že 
        $G \setminus \bc{y,z}$ \ii{musí} být nezávislá množina.
    \end{enumerate}
    Ve všech případech jsme došli ke sporu. Náš původní předpoklad, že tvrzení neplatí, musí být nepravdivý.
    \hspace{\fill}\qed
\end{solution}

\begin{exercise}
    Je dán prostý orientovaný graf $G$ bez smyček s $n$ vrcholy a $m$ hranami.\\
    Dokažte nebo vyvraťte: Je-li $G$ souvislý, ale ne silně souvislý, pak platí
    \begin{equation*}
        n - 1 \leq m \leq (n-1)^2.
    \end{equation*}
    Buď tvrzení dokažte, nebo najděte protipříklad.
\end{exercise}
\begin{solution}

\end{solution}

\end{document}