\documentclass[11pt,a4paper]{article}
\usepackage[czech]{babel}
\usepackage[margin=1in]{geometry}

\usepackage{../cfg}
\setcounter{section}{3}

\begin{document}

\title{\bb{Domácí úkol 3}}
\author{Jakub Adamec\\XP01TGR}

\maketitle

\begin{exercise}
    Je dáno číslo $n \geq 5$. Je možné pro každé takové $n$ zkonstruovat 2-souvislý prostý neorientovaný graf $G$ bez 
    smyček, který má
    \vspace{-1em}
    \begin{itemize}[noitemsep]
        \item průměr $\diam(G)$ roven 2,
        \item a má $2n-5$ hran?
    \end{itemize}
    \vspace{-1em}
    Jestliže ano, pro každé $n$ takový graf zkonstruujte; jestliže ne, zdůvodněte, proč takový graf nemůže existovat.
\end{exercise}
\begin{solution}

\end{solution}

\begin{exercise}
    Dokažte nebo vyvraťe:\\
    Je dán prostý souvislý neorientovaný graf $G$ bez smyček s $n \geq 4$ vrcholy, který neobsahuje jako \bb{indukovaný 
    podgraf} úplný bipartitní graf $K_{1,3}$. Pak v $G$ existují dva sousední vrcholy $x,y$ takové, že graf $G \setminus
    \bc{x,y}$ je také souvislý. (Graf $G \setminus \bc{x,y}$ je podgraf $G$, ze kterého jsme odstranili vrcholy $x$ a 
    $y$, nejen hranu s krajními vrcholy $x$ a $y$.)

    ($K_{1,3}$ je úplný bipartitní graf se stranami o 1 a 3 vrcholech.)
\end{exercise}
\begin{solution}
    
\end{solution}

\begin{exercise}
    Je dán prostý orientovaný graf $G$ bez smyček s $n$ vrcholy a $m$ hranami.\\
    Dokažte nebo vyvraťte: Je-li $G$ souvislý, ale ne silně souvislý, pak platí
    \begin{equation*}
        n - 1 \leq m \leq (n-1)^2.
    \end{equation*}
    Buď tvrzení dokažte, nebo najděte protipříklad.
\end{exercise}
\begin{solution}
    
\end{solution}

\end{document}