\documentclass[11pt,a4paper]{article}
\usepackage[czech]{babel}
\usepackage[margin=1in]{geometry}

\usepackage{../cfg}
\setcounter{section}{3}

\begin{document}

\title{\bb{Domácí úkol 3}}
\author{Jakub Adamec\\XP01TGR}

\maketitle

\begin{exercise}
    Je dáno číslo $n \geq 5$. Je možné pro každé takové $n$ zkonstruovat 2-souvislý prostý neorientovaný graf $G$ bez 
    smyček, který má
    \vspace{-1em}
    \begin{itemize}[noitemsep]
        \item průměr $\diam(G)$ roven 2,
        \item a má $2n-5$ hran?
    \end{itemize}
    \vspace{-1em}
    Jestliže ano, pro každé $n$ takový graf zkonstruujte; jestliže ne, zdůvodněte, proč takový graf nemůže existovat.
\end{exercise}
\begin{solution}
    Nejdříve ověřme, že tvrzení platí pro $n=5$. Zvolme $C_5$ jako kružnici na 5 vrcholech. Protože je kružnicí, tak je 
    určitě 2-souvislá, protože nemá artikulaci a je souvislá. Má 5 vrcholů a 5 hran, což odpovídá $2 \cdot 5 - 5 = 5$ 
    počtu hran. Průměr $\diam(C_5) = \floor*{\frac{5}{2}} = 2$ také sedí zadání.

    Pro $n \geq 6$ zkonstruujme graf $G_n$ tak, že vezmeme 5 vrcholů $v_1, v_2, \dots, v_5$ a spojíme je do kružnice 
    $C_5$. Máme tedy 5 vrcholů a 5 hran. Označme si zbývající počet vrcholů jako $k = n - 5$. Přidejme zbývající vrcholy
    $u_1, u_2, \dots, u_k$. Každý z těchto $k$ vrcholů připojíme dvěma hranami k jedné konkrétní dvojici vrcholů, na 
    $C_5$. Tato dvojice vrcholů nesmí být vzájemně sousední. Vyberme například dvojici $v_1$ a $v_3$. Pak pro 
    $i=1, 2, \dots, k$ přidáme hrany $(u_i, v_1)$ a $(u_i, v_3)$. Teď máme $n$ vrcholů a $5+2k$ hran.

    Ověřme, zda takto zkonstruovaný graf splňuje požadavky.
    \vspace{-1em}
    \begin{enumerate}[1)]
        \item \bb{Počet hran}. Máme $5+2k$ hran, tedy $5 + 2(n-5) = 2n-5$ hran, to jsme přesně chtěli.
        \item \bb{2-souvislost}. Tzn. musíme ukázat, že graf nemá artikulaci. Ověřme tedy indukcí: Odeberme $u_i$; graf 
        $G_{n-1}$, který zbyde, má stejnou strukturu, jen o jeden připojený vrchol méně. Postupujme dále až k $n=6$, to
        nám zbyde původní $C_5$ s jedním $u_1$, který je očividně souvislý. Odebrání $u_i$ tedy graf nerozpojí.

        Zkusme odebrat $v_1$: Zbytek $C_5$ je cesta $v_2, v_3, v_4, v_5$. Všechny vrcholy $u_i$ jsou stále připojeny k
        $v_3$. Celý zbytek grafu $G_n \setminus v_1$ je tedy souvislý. Analogicky se ukáže pro $v_3$.
        
        Anebo odeberme $v_2$, $v_4$, nebo $v_5$. BÚNO vyberme $v_2$. Graf $G_n \setminus v_2$ je cesta 
        $v_1, v_5, v_4, v_3$. Všechny vrcholy $u_i$ jsou připojeny k $v_1$ i $v_3$. Tedy $G_n \setminus v_2$ je 
        souvislý.

        Graf je tedy určitě 2-souvislý, protože je souvislý a nemá artikulaci.
        \item \bb{Průměr $\diam(G)=2$}. Dva vrcholy jsou buď vzájemní sousedé, anebo mají společného souseda. 
        
        Jestliže oba vrcholy leží na $C_5$, tak jsme již v případě $n=5$ dokázali, že $\diam(C_5) = 2$. 
        Jestliže oba vrcholy jsou z připojených vrcholů $u_i$, pak mají zaručeně společné sousedy $v_1$ a $v_3$, tedy
        $d(u_i, u_j)=2$, pro $i \not= j$. Pokud je jeden z $C_5$ a druhý z $u_i$, pak:
        \vspace{-0.8em}
        \begin{itemize}
            \item je jeden z vrcholů $v_1$, respektive $v_3$, pak jsou sousedé, $d=1$.
            \item je jeden z vrcholů $v_2$, pak mají společné sousedy $v_1$ a $v_3$, $d(v_2, u_i) = 2$.
            \item je jeden z vrcholů $v_4$, pak mají společného souseda $v_3$, $d(v_4, u_i) = 2$. Obdobně pro $v_5$.
        \end{itemize}
        Takže průměr grafu je 2.
    \end{enumerate}
    Ano, pro každé $n \geq 5$ je možné takový graf zkonstruovat.
    \hspace{\fill}\qed
\end{solution}

\begin{exercise}
    Dokažte nebo vyvraťe:\\
    Je dán prostý souvislý neorientovaný graf $G$ bez smyček s $n \geq 4$ vrcholy, který neobsahuje jako \bb{indukovaný 
    podgraf} úplný bipartitní graf $K_{1,3}$. Pak v $G$ existují dva sousední vrcholy $x,y$ takové, že graf $G \setminus
    \bc{x,y}$ je také souvislý. (Graf $G \setminus \bc{x,y}$ je podgraf $G$, ze kterého jsme odstranili vrcholy $x$ a 
    $y$, nejen hranu s krajními vrcholy $x$ a $y$.)

    ($K_{1,3}$ je úplný bipartitní graf se stranami o 1 a 3 vrcholech.)
\end{exercise}
\begin{solution}
    
\end{solution}

\begin{exercise}
    Je dán prostý orientovaný graf $G$ bez smyček s $n$ vrcholy a $m$ hranami.\\
    Dokažte nebo vyvraťte: Je-li $G$ souvislý, ale ne silně souvislý, pak platí
    \begin{equation*}
        n - 1 \leq m \leq (n-1)^2.
    \end{equation*}
    Buď tvrzení dokažte, nebo najděte protipříklad.
\end{exercise}
\begin{solution}
    \titlebreak
    \begin{enumerate}[1)]
        \item $m \geq n-1$. Protože $G$ je souvislý, znamená to, že v něm ignorujeme orientaci hran. Z definice víme, že
        jakýkoli souvislý neorientovaný graf s $n$ vrcholy musí mít alespoň $n-1$ hran. Tedy nutně platí $m \geq n-1$.
        \item $m \leq (n-1)^2$.
    \end{enumerate}
\end{solution}

\end{document}