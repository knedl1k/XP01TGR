\documentclass[11pt,a4paper]{article}
\usepackage[czech]{babel}
\usepackage[margin=1in]{geometry}

\usepackage{../cfg}
\setcounter{section}{3}

\begin{document}

\title{\bb{Domácí úkol 3}}
\author{Jakub Adamec\\XP01TGR}

\maketitle

\begin{exercise}
    Je dáno číslo $n \geq 5$. Je možné pro každé takové $n$ zkonstruovat 2-souvislý prostý neorientovaný graf $G$ bez smyček, který má průměr $\diam(G)$ roven 2  a má $2n-5$ hran?

    Jestliže ano, pro každé $n$ takový graf zkonstruujte; jestliže ne, zdůvodněte, proč takový graf nemůže existovat.
\end{exercise}
\begin{solution}
    Nejdříve ověřme, že tvrzení platí pro $n=5$. Zvolme $C_5$ jako kružnici na 5 vrcholech. Protože je kružnicí, tak je určitě 2-souvislá, protože nemá artikulaci a je souvislá. Má 5 vrcholů a 5 hran, což odpovídá $2 \cdot 5 - 5 = 5$ počtu hran. Průměr $\diam(C_5) = \floor*{\frac{5}{2}} = 2$ také sedí zadání.

    Pro $n \geq 6$ zkonstruujme graf $G_n$ tak, že vezmeme 5 vrcholů $v_1, v_2, \dots, v_5$ a spojíme je do kružnice $C_5$. Máme tedy 5 vrcholů a 5 hran. Označme si zbývající počet vrcholů jako $k = n - 5$. Přidejme zbývající vrcholy $u_1, u_2, \dots, u_k$. Každý z těchto $k$ vrcholů připojíme dvěma hranami k jedné konkrétní dvojici vrcholů, na $C_5$. Tato dvojice vrcholů nesmí být vzájemně sousední. Vyberme například dvojici $v_1$ a $v_3$. Pak pro $i=1, 2, \dots, k$ přidáme hrany $(u_i, v_1)$ a $(u_i, v_3)$. % Teď máme $n$ vrcholů a $5+2k$ hran.

    Ověřme, zda takto zkonstruovaný graf splňuje požadavky.
    \vspace{-1em}
    \begin{itemize}
        \item \ii{Počet hran}. Máme $5+2k$ hran, tedy $5 + 2(n-5) = 2n-5$ hran, to jsme přesně chtěli.
        \item \ii{2-souvislost}. Tzn. musíme ukázat, že graf nemá artikulaci. Ověřme tedy indukcí: Odeberme $u_i$; graf $G_{n-1}$, který zbyde, má stejnou strukturu, jen o jeden připojený vrchol méně. Postupujme dále až k $n=6$, to nám zbyde původní $C_5$ s jedním $u_1$, který je očividně souvislý. Odebrání $u_i$ tedy graf nerozpojí.

        Zkusme odebrat $v_1$: Zbytek $C_5$ je cesta $v_2, v_3, v_4, v_5$. Všechny vrcholy $u_i$ jsou stále připojeny k $v_3$. Celý zbytek grafu $G_n \setminus v_1$ je tedy souvislý. Analogicky se ukáže pro $v_3$.
        
        Anebo odeberme $v_2$, $v_4$, nebo $v_5$. \ii{BÚNO} vyberme $v_2$. Graf $G_n \setminus v_2$ je cesta $v_1, v_5, v_4, v_3$. Všechny vrcholy $u_i$ jsou připojeny k $v_1$ i $v_3$. Tedy $G_n \setminus v_2$ je souvislý.

        Graf je tedy určitě 2-souvislý, protože je souvislý a nemá artikulaci.
        \item \ii{Průměr $\diam(G)=2$}. Vrcholy jsou buď vzájemní sousedé, anebo mají společného souseda. 
        
        Jestliže oba vrcholy leží na $C_5$, tak jsme již v případě $n=5$ dokázali, že $\diam(C_5) = 2$. Jestliže oba vrcholy jsou z připojených vrcholů $u_i$, pak mají zaručeně společné sousedy $v_1$ a $v_3$, tedy $d(u_i, u_j)=2$, pro $i \not= j$. Pokud je jeden z $C_5$ a druhý z $u_i$, pak:
        \vspace{-0.8em}
        \begin{itemize}
            \item je jeden z vrcholů $v_1$, respektive $v_3$, pak jsou sousedé, $d=1$.
            \item je jeden z vrcholů $v_2$, pak mají společné sousedy $v_1$ a $v_3$, $d(v_2, u_i) = 2$.
            \item je jeden z vrcholů $v_4$, pak mají společného souseda $v_3$, $d(v_4, u_i) = 2$. Obdobně pro $v_5$.
        \end{itemize}
        \vspace{-0.8em}
        Takže průměr grafu je 2.
    \end{itemize}
    \vspace{-1em}
    Ano, pro každé $n \geq 5$ je možné takový graf zkonstruovat.
    \hspace{\fill}\qed
\end{solution}

\begin{exercise}
    Dokažte nebo vyvraťe:\\
    Je dán prostý souvislý neorientovaný graf $G$ bez smyček s $n \geq 4$ vrcholy, který neobsahuje jako \bb{indukovaný podgraf} úplný bipartitní graf $K_{1,3}$. Pak v $G$ existují dva sousední vrcholy $x,y$ takové, že graf $G \setminus \bc{x,y}$ je také souvislý. (Graf $G \setminus \bc{x,y}$ je podgraf $G$, ze kterého jsme odstranili vrcholy $x$ a $y$, nejen hranu s krajními vrcholy $x$ a $y$.)

    ($K_{1,3}$ je úplný bipartitní graf se stranami o 1 a 3 vrcholech.)
\end{exercise}
\begin{solution}
    Mějme v grafu $G$ nejdelší možnou cestu $P$. Její vrcholy označme jako $v_1, v_2, \dots, v_{k-1}, v_k$, kde $v_k$ je konec cesty. Dále uvažme hranu na samém konci cesty $\bc{v_{k-1}, v_k}$. Chceme dokázat, že po odstranění této dvojice zůstane graf souvislý.

    Sporem. Předpokládejme, že po odstranění $\bc{v_{k-1}, v_k}$ se graf stane nesouvislým. To znamená, že v grafu existuje nějaký vrchol (případně skupina vrcholů) $w$, který ztratil spojení se zbytkem grafu. Vrchol $w$ tedy musel být připojen \ii{pouze} k odstraněným vrcholům $v_{k-1}$ nebo $v_k$ a k ničemu jinému ze zbytku grafu.

    Vrchol $w$ nemůže být součástí původní cesty $P$, protože $v_{k-2}$ a dřívější vrcholy v grafu zůstaly, takže pokud by $w$ byl na cestě, zůstal by připojen. $w$ je tedy \enquote{nový} vrchol mimo cestu. Vyšetřeme situace, kam může být $w$ připojen:
    \begin{enumerate}[a)]
        \item $w$ je spojen s koncovým vrcholem $v_k$. Pokud vede hrana $\bc{v_k, w}$, můžeme nejdelší cestu $P$ jednoduše prodloužit
        \begin{figure}[H]
            \centering
            \begin{tikzpicture}[
                node distance=1.5cm,
            ]
                \node (v1) {$v_1$};
                \node (dots) [right of=v1] {$\dots$};
                \node (vk-1) [right of=dots] {$v_{k-1}$};
                \node (vk) [right of=vk-1] {$v_k$};
                \node (w) [right of=vk] {$w$};
        
                \draw (v1) -- (dots) -- (vk-1) -- (vk);
                \draw(vk) -- (w);
            \end{tikzpicture}
        \end{figure}
        To je ale spor s tím, že $P$ byla nejdelší cesta, takže situace nemůže nastat.
        \item $w$ je spojen s předposledním vrcholem $v_{k-1}$. Zde využijeme vlastnost, že graf neobsahuje $K_{1,3}$. Zaměřme se na sousedy vrcholu $v_{k-1}$. Má minimálně 3 sousedy:
        \begin{enumerate}[1)]
            \item $v_{k-2}$, předchozí vrchol je na cestě, který v komponentě zůstal,
            \item $v_k$,
            \item $w$, vrchol, který se má odpojit.
        \end{enumerate}
        Aby tyto vrcholy netvořily zakázaný $K_{1,3}$ se středem v $v_{k-1}$, musí být některé z vrcholů $\bc{v_{k-2}, v_k, w}$ navzájem propojeny hranou. Zkoumejme, které to mohou být:
        \begin{itemize}
            \item Kdyby existovala hrana $\bc{w, v_{k-2}}$, pak by po odstranění dvojice $\bc{v_{k-1}, v_k}$ vrchol $w$ nezůstal izolovaný, protože by byl spojen s $v_{k-2}$, což je spor s naším předpokladem, že se graf rozpadl.
            \item Jediná možnost, která nevede na $K_{1,3}$, je hrana $\bc{v_k, w}$. Ale pokud existuje hrana $\bc{v_k, w}$, jsme ve stejné situaci, jako v a). Cestu lze prodloužit o $w$. A to je opět spor s tím, že cesta $P$ je nejdelší.
            \begin{figure}[H]
                \centering
                \begin{tikzpicture}[
                    node distance=2cm,
                    missing/.style={red, decorate, decoration={crosses, segment length=4pt}}
                ]
                    \node (vk-1) {$v_{k-1}$};

                    \node (vk-2) [left of=vk-1] {$v_{k-2}$};
                    \node (vk) [right of=vk-1] {$v_k$};
                    \node (w) [above of=vk-1] {$w$};

                    \draw (vk-2) -- (vk-1);
                    \draw (vk) -- (vk-1);
                    \draw (w) -- (vk-1);

                    \draw[missing] (w) -- (vk-2);
                    \draw[missing] (w) -- (vk);
                    \draw[missing, bend right=40] (vk-2) to (vk);
                \end{tikzpicture}
            \end{figure}
        \end{itemize}
    \end{enumerate}
    Takže neexistuje žádný vrchol $w$, který by se odstraněním koncové hrany $\bc{v_{k-1}, v_k}$ odpojil. Zbytek grafu tedy zůstává souvislý.
    \hspace{\fill}\qed
\end{solution}

\newpage
\begin{exercise}
    Je dán prostý orientovaný graf $G$ bez smyček s $n$ vrcholy a $m$ hranami.\\
    Dokažte nebo vyvraťte: Je-li $G$ souvislý, ale ne silně souvislý, pak platí
    \begin{equation}
        n - 1 \leq m \leq (n-1)^2.
    \end{equation}
    Buď tvrzení dokažte, nebo najděte protipříklad.
\end{exercise}
\begin{solution}
    Dokažme jednotlivé meze.
    \begin{enumerate}[1)]
        \item \ii{Dolní mez $m \geq n-1$}. To již plyne z toho, že $G$ je souvislý, respektive, že $G^\prime$, který vznikl tím, že jsme odstranili orientaci z hran $G$, je souvislý. Graf $G^\prime$ má stejný počet vrcholů a hran jako graf $G$. A z definice víme, že jakýkoliv neorientovaný souvislý graf s $n$ vrcholy musí mít alespoň $n-1$ hran.
        \item \ii{Horní mez $m \geq (n-1)^2$}. Využijme toho, že $G$ není silně souvislý. To totiž znamená, že jeho vrcholy $V$ lze rozdělit na dvě neprázdné, disjuktní podmnožiny, nazvěme je $A$ a $B$, takové, že \ii{neexistuje žádná hrana vedoucí z $B$ do $A$}. Chceme najít maximální možný počet hran $m$ takového grafu. Počet hran maximalizujeme, pokud přidáme všechny hrany, které nejsou zakázané, tj. hrany z $B$ do $A$. Nechť $|A| = p$ a $|B| = q$, kde $p+q = n$ a $p,q \geq 1$.

        Maximální počet hran je součtem:
        \begin{itemize}
            \item \ii{Hran uvnitř $A$}: V $G$ nejsou smyčky, maximální počet hran v $A$ je $p(p-1)$.
            \item \ii{Hran uvnitř $B$}: Obdobně jako pro $A$ $\rightarrow$ $q(q-1)$.
            \item \ii{Hran z $A$ do $B$}: Maximální počet hran je $p \cdot q$.
            \item \ii{Hran z $B$ do $A$}: Podle našeho předpokladu $0$.
        \end{itemize}
        Celkový maximální počet hran $m_{\max}$ pro graf, který není silně souvislý, je tedy:
        \begin{equation}
            m \leq p(p-1) + q(q-1) + pq
        \end{equation}
        Upravme dosazením $q = n-p$:
        \begin{align}
            m &\leq p(p-1) + (n-p)(n-p-1) + p(n-p) \\
            m &\leq p^2 - p + (n^2 -np -n -np + p^2 + p) + np - p^2 \\
            m &\leq (p^2 + p^2 - p^2) + (-p + p) + (-np -np + np) + n^2 - n \\
            m &\leq p^2 -np + n^2 -n
        \end{align}
        Protože maximalizujeme funkci $f(p) = p^2 -np + n^2 -n$ pro $p \in \bc{1, 2, \dots, n-1}$, která je konvexní kvadratickou funkcí, svého maxima nabývá v krajních bodech.
        \begin{itemize}
            \item \ii{Pro $p=1$}: 
            \begin{equation}
                m \leq 1^2 - n(1) + n^2 -n = 1 -n+ n^2 -n = n^2 -2n + 1 = (n-1)^2
            \end{equation}
            \item \ii{Pro $p=n-1$}:
            \begin{align}
                m &\leq (n-1)^2 - n(n-1) + (n^2 - n)m \\
                &\leq (n^2 -2n + 1) - (n^2 - n) + (n^2 -n)m \\
                &\leq n^2 -2n + 1 = (n-1)^2
            \end{align}
            Takže maximální možný počet hran v grafu, který není silně souvislý, je $(n-1)^2$. Z toho plyne, že 
            $m \geq (n-1)^2$.
        \end{itemize}
    \end{enumerate}
    \hspace{\fill}\qed
\end{solution}

\end{document}
