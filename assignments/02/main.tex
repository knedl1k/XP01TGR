\documentclass[11pt,a4paper]{article}
\usepackage[czech]{babel}
\usepackage[margin=1in]{geometry}

\usepackage{../cfg}
\setcounter{section}{2}

\begin{document}

\title{\bb{Domácí úkol 2}}
\author{Jakub Adamec\\XP01TGR}

\maketitle

\begin{exercise}
Je dán prostý neorientovaný souvislý graf $G = (V,E)$, který má most $e = \bc{u,v}$. Určete zda alespoň jeden z vrcholů
$u,v$ musí být artikulace anebo oba vrcholy $u,v$ musí být artikulace.\\
Odpověď pečlivě zdůvodněte.
\end{exercise}
\begin{solution}
Pomocné lemma. \ii{Vrchol $u$ není artikulace $\iff$ $d(u)=1$.}\\
Důkaz. Vrchol $u$ není artikulace, pokud graf $G \setminus u$ zůstane souvislý. Graf $G \setminus u$ vytvoříme tak, že z 
$G$ odstraníme vrchol $u$ a všechny hrany, které z něj vycházejí (včetně mostu $e = \bc{u,v}$). Graf $G \setminus u$ se 
bude skládat ze dvou (potenciálně prázdných) částí:
\begin{enumerate}[(1), noitemsep]
    \item Zbytek komponenty $C_u$ po odstranění $u$, tj. $C_u \setminus \bc{u}$.
    \item Celá komponenta $C_v$.
\end{enumerate}
Mezi těmito dvěma částmi nevede žádná hrana, protože jediná hrana, která je spojovala, most $e$, byla odstraněna spolu s 
vrcholem $u$. Aby byl graf $G \setminus u$ souvislý, musí být jedna z těchto dvou částí prázdná.  Část $C_v$ nemůže být 
prázdná, protože obsahuje alespoň vrchol $v$.  Tedy část $C_u \setminus \bc{u}$ musí být nutně prázdná. A to platí právě 
tehdy, když komponenta $C_u$ obsahovala \ii{pouze} vrchol $u$. To znamená, že vrchol $u$ neměl v $G$ žádného jiného 
souseda, než $v$. Tedy $u$ není artikulace $\iff$ $d(u)=1$. 
\hspace{\fill}\qed 

Najděme protipříklad. Hledejme souvislý graf $G$ s mostem $e = \bc{u,v}$, kde $d(u)=d(v)=1$. 
Jediný takový graf je graf se dvěma vrcholy ($u,v$) a jedinou hranou ($e=\bc{u,v}$).
Ověřme, že tento graf má všechny námi požadované vlastnosti:
\begin{enumerate}[(a), noitemsep]
    \item \ii{Je graf prostý, neorientovaný, souvislý?} Ano.
    \item \ii{Je $e=\bc{u,v}$ most?} Ano. $G$ je souvislý a $G \setminus e$ sestává ze dvou izolovaných vrcholů $u,v$, 
    takže má 2 komponenty souvislosti. Počet komponent se zvýšil.
    \item \ii{Je $u$ artikulace?} Ne. $d(u)=1$. Graf $G \setminus u$ je pouze vrchol $v$. Počet komponent se nezvýšil.
    \item \ii{Je $v$ artikulace?} Ne. Obdobná situace jako pro $u$.
\end{enumerate}
Našli jsme tedy graf $G$, která má most $e = \bc{u,v}$, ale \ii{ani jeden} z vrcholů $u,v$ není artikulace. Takže ani 
jeden vrchol nemusí být artikulace.
\hspace{\fill}\qed
\end{solution}

\begin{exercise}
Dokažte nebo vyvraťte: Každý prostý neorientovaný graf $G$ bez smyček s alespoň dvěma vrcholy obsahuje alespoň dva 
vrcholy, které nejsou artikulacemi.
\end{exercise}
\begin{solution}
Musíme problém rozdělit na dva případy, když graf $G$ je (ne)souvislý.
\begin{enumerate}[(a)]
    \item \ii{Graf $G$ je souvislý.} Máme $|V| \geq 2$. Hledáme dva vrcholy $u,v$ takové, že $G \setminus u$ a 
    $G \setminus v$ budou mít právě jednu komponentu souvislosti.
    \begin{enumerate}
        \item Protože $G$ je souvislý a má $|V| \geq 2$ vrcholů, můžeme v něm sestrojit kostru $T$. $T$ je strom a má 
        stejný počet vrcholů jako $G$.
        \item Každý strom s alespoň dvěma vrcholy má alespoň dva listy. Vyberme si dva různé listy kostry $T$ a nazvěme 
        je $u$ a $v$. Teď stačí dokázat, že $u$, respektive $v$, není artikulace. Tedy, že $G \setminus u$ bude 
        souvislý.
    \end{enumerate}
\end{enumerate}
\end{solution}

\begin{exercise}
Dokažte nebo vyvraťte: Prostý souvislý neorientovaný graf $G$ bez smyček s alespoň dvěma hranami je 2-souvislý právě 
tehdy, když každé dvě hrany grafu $G$ leží na společné kružnici.
\end{exercise}
\begin{solution}

\end{solution}

\end{document}