\documentclass[11pt,a4paper]{article}
\usepackage[czech]{babel}
\usepackage[margin=1in]{geometry}

\usepackage{../cfg}
\setcounter{section}{1}

\begin{document}

\title{\bb{Domácí úkol 2}}
\author{Jakub Adamec\\XP01TGR}

\maketitle

\begin{exercise}
Je dán prostý neorientovaný souvislý graf $G = (V,E)$, který má most $e = \bc{u,v}$. Určete zda alespoň jeden z vrcholů
$u,v$ musí být artikulace anebo oba vrcholy $u,v$ musí být artikulace.\\
Odpověď pečlivě zdůvodněte.
\end{exercise}

\begin{exercise}
Dokažte nebo vyvraťte: Každý prostý neorientovaný graf $G$ bez smyček s alespoň dvěma vrcholy obsahuje alespoň dva 
vrcholy, které nejsou artikulacemi.
\end{exercise}

\begin{exercise}
Dokažte nebo vyvraťte: Prostý souvislý neorientovaný graf $G$ bez smyček s alespoň dvěma hranami je 2-souvislý právě 
tehdy, když každé dvě hrany grafu $G$ leží na společné kružnici.
\end{exercise}

\end{document}