\documentclass[11pt,a4paper]{article}
\usepackage[czech]{babel}
\usepackage[margin=1in]{geometry}

\usepackage{../cfg}
\setcounter{section}{2}

\begin{document}

\title{\bb{Domácí úkol 2}}
\author{Jakub Adamec\\XP01TGR}

\maketitle

\begin{exercise}
Je dán prostý neorientovaný souvislý graf $G = (V,E)$, který má most $e = \bc{u,v}$. Určete zda alespoň jeden z vrcholů
$u,v$ musí být artikulace anebo oba vrcholy $u,v$ musí být artikulace.\\
Odpověď pečlivě zdůvodněte.
\end{exercise}
\begin{solution}
    \begin{lemma}
        Vrchol $u$ není artikulace $\iff$ $d(u)=1$.
    \end{lemma}
    \begin{prof}
        Vrchol $u$ není artikulace, pokud graf $G \setminus u$ zůstane souvislý. Graf $G \setminus u$ vytvoříme tak, že 
        z $G$ odstraníme vrchol $u$ a všechny hrany, které z něj vycházejí (včetně mostu $e = \bc{u,v}$). Graf 
        $G \setminus u$ se bude skládat ze dvou (potenciálně prázdných) částí:
        \vspace{-1em}
        \begin{enumerate}[1), noitemsep]
            \item Zbytek komponenty $C_u$ po odstranění $u$, tj. $C_u \setminus \bc{u}$.
            \item Celá komponenta $C_v$.
        \end{enumerate}
        \vspace{-1em}
        Mezi těmito dvěma částmi nevede žádná hrana, protože jediná hrana, která je spojovala, most $e$, byla 
        odstraněna spolu s vrcholem $u$. Aby byl graf $G \setminus u$ souvislý, musí být jedna z těchto dvou částí 
        prázdná.  Část $C_v$ nemůže být prázdná, protože obsahuje alespoň vrchol $v$. Proto část $C_u \setminus \bc{u}$ 
        musí být nutně prázdná. A to platí právě tehdy, když komponenta $C_u$ obsahovala \ii{pouze} vrchol $u$. To 
        znamená, že vrchol $u$ neměl v $G$ žádného jiného souseda, než $v$. Tedy $u$ není artikulace $\iff$ $d(u)=1$. 
        \hspace{\fill}\qed 
    \end{prof}
Najděme protipříklad. Hledejme souvislý graf $G$ s mostem $e = \bc{u,v}$, kde $d(u)=d(v)=1$. 
Jediný takový graf je graf se dvěma vrcholy ($u,v$) a jedinou hranou ($e=\bc{u,v}$).
Ověřme, že tento graf má všechny námi požadované vlastnosti:
\begin{enumerate}[a), noitemsep]
    \item \ii{Je graf prostý, neorientovaný, souvislý?} Ano.
    \item \ii{Je $e=\bc{u,v}$ most?} Ano. $G$ je souvislý a $G \setminus e$ sestává ze dvou izolovaných vrcholů $u,v$, 
    takže má 2 komponenty souvislosti. Počet komponent se zvýšil.
    \item \ii{Je $u$ artikulace?} Ne. $d(u)=1$. Graf $G \setminus u$ je pouze vrchol $v$. Počet komponent se nezvýšil.
    \item \ii{Je $v$ artikulace?} Ne. Obdobná situace jako pro $u$.
\end{enumerate}
Našli jsme tedy graf $G$, která má most $e = \bc{u,v}$, ale \ii{ani jeden} z vrcholů $u,v$ není artikulace. Takže ani 
jeden vrchol nemusí být artikulace.
\hspace{\fill}\qed
\end{solution}

\newpage

\begin{exercise}
Dokažte nebo vyvraťte: Každý prostý neorientovaný graf $G$ bez smyček s alespoň dvěma vrcholy obsahuje alespoň dva 
vrcholy, které nejsou artikulacemi.
\end{exercise}
\begin{solution}
Rozdělme problém na dva případy, dle souvislosti $G$.
\vspace{-1em}
\begin{enumerate}[1)]
    \item \ii{Pokud $G$ není souvislý, pak má alespoň 2 komponenty souvislosti.} V takovém případě nám nastavají dvě situace:
    \begin{enumerate}[a)]
        \item $G$ se skládá pouze z izolovaných vrcholů. Protože $|V(G)| \geq 2$, $G$ má alespoň dva vrcholy, $v_1$ a 
        $v_2$. $G$ má $n$ komponent souvislosti. Graf $G \setminus v_1$ má $n-1$ izolovaných vrcholů, tedy i komponent 
        souvislosti. Takže $v_1$, respektive $v_2$, určitě \ii{není} artikulace. 
        \item $G$ má alespoň jednu komponentu souvislosti $C_i$ s $|V(C_i)| \geq 2$. Nechť $v$ je vrchol v komponentě 
        $C_i$. Odebrání $v$ může ovlivnit pouze komponentu $C_i$. Zadefinujme si $k(G)$ jako počet komponent souvislosti 
        grafu $G$. 
        \begin{equation}
            k(G) = k(C_i) + \sum_{j \not= i}k(C_j) = 1 + \sum_{j \not= i}1
        \end{equation} 
        Počet komponent $G \setminus v$ je 
        \begin{equation}
            k(G \setminus v) = k(C_i \setminus v) + \sum_{j \not= i}k(C_j) = k(C_i \setminus v) + k(G) - 1 > k(G)
        \end{equation}
        právě tehdy, když $k(C_i \setminus v) > 1$.  
        
        To znamená, že $v$ je artikulací $G$ právě tehdy, když je artikulací své komponenty $C_i$. Tudíž stačí dokázat 
        tvrzení pro souvislou komponentu $C_i$. Pokud má $C_i$ alespoň dvě \bb{ne}artikulace, pak tyto dva vrcholy 
        nejsou ani artikulacemi vůči $G$.
    \end{enumerate}
    \item \ii{Pokud $G$ je souvislý, pak má právě jednu komponentu souvislosti.} \\
    Protože $G$ je souvislý a $|V(G)| \geq 2$, můžeme v něm sestrojit kostru $T$. Kostra je strom a má stejný počet
    vrcholů jako $G$. Každý strom s alespoň dvěma vrcholy má alespoň dva listy. Vyberme si dva různé listy kostry $T$, 
    $u$ a $v$. Teď stačí dokázat, že $u$, respektive $v$, není artikulací grafu $G$.

    $G \setminus u$ musí být souvislý graf. Protože odebrání listu z netriviálního stromu zachovává souvislost, 
    tak $T \setminus u$ je souvislý. Poznamenejme, že $T \setminus u$ je podgrafem $G \setminus u$, protože obsahuje
    všechny vrcholy $V(G) \setminus \bc{u}$ a některé hrany z $G \setminus u$. No ale to nutně znamená, že 
    $G \setminus u$ je také souvislý. Takže $u$ určitě není artikulací.
    
    Stejný argument platí i pro druhý list $v$. Tím jsme v souvislém grafu našli alespoň dva vrcholy, které nejsou 
    artikulacemi.
    \hspace{\fill}\qed
\end{enumerate}
\vspace{-1em}
\end{solution}

\newpage

\begin{exercise}
Dokažte nebo vyvraťte: Prostý souvislý neorientovaný graf $G$ bez smyček s alespoň dvěma hranami je 2-souvislý právě 
tehdy, když každé dvě hrany grafu $G$ leží na společné kružnici.
\end{exercise}
\begin{solution}
\titlebreak

\enquote{$\Rightarrow$}: Graf je 2-souvislý. \\
Nechť $e_1 = \bc{u,v}$ a $e_2 = \bc{x,y}$ jsou dvě libovolné různé hrany v $G$. Vytvořme nový graf $G^\prime$ z $G$ tak, 
že rozpůlíme obě hrany:
\vspace{-1em}
\begin{itemize}[noitemsep]
    \item Hranu $e_1$ nahradíme cestou $u \rightarrow w_1 \rightarrow v$.
    \item Hranu $e_2$ nahradíme cestou $x \rightarrow w_2 \rightarrow y$.
\end{itemize}
\vspace{-1em}
Přidali jsme vrcholy $w_1, w_2$. Tímto zachováváme souvislost, tedy i $G^\prime$ je 2-souvislý. Víme, že v 2-souvislém 
grafu každé dva vrcholy leží na společné kružnici. Využijme toho: Existuje kružnice $C^\prime$ v $G^\prime$, která 
obsahuje $w_1$ i $w_2$.

Protože vrcholy $w_1$ a $w_2$ mají v $G^\prime$ stupeň 2, každá kružnice $C^\prime$, která jimi prochází, \ii{musí} 
obsahovat obě hrany s nimi incidentní.
\vspace{-1em}
\begin{itemize}[noitemsep]
    \item Kružnice $C^\prime$ musí obsahovat sled $u \rightarrow w_1 \rightarrow v$ (případně v opačném pořadí).
    \item Kružnice $C^\prime$ musí obsahovat sled $x \rightarrow w_2 \rightarrow y$ (případně v opačném pořadí).
\end{itemize}
\vspace{-1em}
A tedy kružnice $C^\prime$ obsahuje původní hrany $e_1$ a $e_2$.
\hspace{\fill}\qed

\enquote{$\Leftarrow$}: $G$ je prostý graf s $|E| \geq 2$, ve kterém každé dvě hrany leží na společné kružnici.\\
Aby graf byl 2-souvislý, musí být souvislý a nemít artikulaci. Ukažme, že $G$ nemá artikulaci. \\ 
\bb{Sporem.} Ať $G$ \ii{není} 2-souvislý, tedy že má artikulaci $v$.

Pokud $v$ je artikulace, pak graf $G \setminus v$ je nesouvislý. Nechť $C_1$ a $C_2$ jsou dvě z jeho komponent 
souvislosti. Protože $G$ je souvislý, vrchol $v$ musí mít sousedy v obou těchto komponentách:
\vspace{-1em}
\begin{itemize}[noitemsep]
    \item Existuje vrchol $u \in V(C_1)$ takový, že $e_1 = \bc{u,v} \in E(G)$.
    \item Existuje vrchol $w \in V(C_2)$ takový, že $e_2 = \bc{w,v} \in E(G)$.
\end{itemize}
\vspace{-1em}
Jelikož $u \in C_1$ a $w \in C_2$, je $u \not= w$, tak $e_1$ a $e_2$ jsou dvě různé hrany. Použijme předpoklad: hrany 
$e_1$ a $e_2$ musí ležet na společné kružnici $K$. Tato kružnice musí nutně obsahovat sled 
$u \rightarrow v \rightarrow w$. Aby byla kružnice uzavřena, musí existovat cesta $P$ z $w$ zpět do $u$, která již 
\ii{neobsahuje} vrchol $v$, protože $G$ je prostý a kružnice nemůže vrchol opakovat mimo počátku.

To znamená, že cesta $P$ musí celá ležet v grafu $G \setminus v$. Existence cesty $P$ z $w$ do $u$ v $G \setminus v$ ale 
znamená, že $u$ a $w$ leží ve stejné komponentě souvislosti $G \setminus v$. Což je \bb{spor} s tím, jak jsme definovali 
$u$ a $w$.

Graf $G$ je souvislý a \ii{nemá} artikulaci, takže je 2-souvislý.
\hspace{\fill}\qed
\end{solution}

\end{document}