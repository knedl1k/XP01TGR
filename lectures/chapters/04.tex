\section{Extremální teorie}
\subsection{Věta o souvislosti vrcholů a hran (Mantel)}
Máme $G \in \S$ s $n$ vrcholy, $m$ hranami, který nemá $K_3$. Pak $m \leq \frac{n^2}{4}$.

\bb{Důkaz}.\\
\bb{Definice}. Množina $A$ je nezávislá $A \subseteq V(G)$ pokud pro každou $e = \bc{u, v}$, jestliže $u \in A$, platí 
$v \not\in e$. \ii{Množina $A$ je nezávislá právě tehdy, když v ní žádné dva vrcholy nejsou spojené hranou.}

Ať $A$ je nejpočetnější nezávislá množina a $B = V \setminus A$. $G$ nemá $K_3$: každá množina sousedů vrcholů $v \in V$
je nezávislá množina.
\begin{equation}
    m \leq \sum_{N \in B}d(v) \leq \underbrace{(n-k)}_{|B|} \cdot \underbrace{k}_{|A|}
\end{equation}
Každá hrana má alespoň 1 krajní vrchol v $B$. Pro které $k$ je $(n-k)k$ největší?
\begin{align*}
    f(x) &= (n-x)x\\
    f^\prime(x) &= n-2x \implies f^\prime(x) = 0 \iff x = \frac{n}{2}\\
    f^{\prime \prime}(x) &= -2\\
\end{align*}
Protože jsme v $\N$, tak $\ceil*{\frac{n}{2}} \cdot \floor*{\frac{n}{2}} = \frac{n+1}{2} \frac{n-1}{2} = \frac{n^2-1}{4} 
\leq \frac{n^2}{4}$.
\hspace{\fill}\qed

\subsection{Věta o souvislosti hran a úplném grafu}
Máme $G \in \S$, který neobsahuje $K_{r+1}$ (úplný graf na $r+1$ vrcholech), $r \geq 2$. Pak
\begin{equation}
    m \leq \frac{r-1}{r} \frac{n^2}{2}.
\end{equation}
\bb{Důkaz}. Vezměme graf $G$ bez $K_{r+1}$ s nejméně hranami (přidáním hrany by vznikl $K_{r+1}$). Tedy $G$ má $K_r$.
Ať $A$ je množina vrcholů $K_r$ a $B$ je $V(G) \setminus A$, $|B| = n -r$. Každý vrchol $v \in B$ má max $r-1$ sousedů v
$A$ (jinak by $A \cup \bc{u}$ tvořil $K_{r+1}$).

$m$ rozdělíme na \colorbox{blue!30}{hrany v $A$ (hrany úplného grafu)}, \colorbox{black!30}{hrany mezi $A$ a $B$} a 
\colorbox{black!20!yellow}{hrany v $B$}. % TODO: dodělat nákres
\begin{equation}
    m = m_A + m_{A-B} + m_B \leq \frac{r(r-1)}{2} + (n-r)(r-1) + m_B
\end{equation}
a graf indukovaný $B$ neobsahuje $K_{r+1}$ a má maximální počet hran.
\begin{align*}
    m_B &< m \\
    n-r = |B| &< n
\end{align*}
Použijme tedy silnou indukci, dle počtu vrcholů $n = |V(G)|$.
\begin{itemize}
    \item Základní krok. $n=1, 2,  \dots, r$. 
        \begin{align*}
            m &\leq \frac{n(n-1)}{2} \\
            \frac{n(n-1)}{2}&\overset{?}{\leq}\frac{r-1}{r}\frac{n^2}{2} \\
            \frac{r-1}{r}\frac{n^2}{2} - \frac{n(n-1)}{2} &= \frac{n}{2}\left(n \frac{r-1}{r}-(n-1)\right) \\
            &= \frac{n}{2}\frac{nr -n -nr +r}{r} = \frac{n}{2}\frac{\overbrace{r-n}^{\geq 0}}{\underbrace{r}_{\geq 0}}
            \geq 0.
        \end{align*}
    \item Když budeme mít indukční předpoklad pro $G_B$, pak:
        \begin{align}
            m &\leq \frac{r(r-1)}{2} + (n-r)(r-1) + \frac{r-1}{n}\frac{(n-r)^2}{2} \\
            &= \frac{r-1}{r}\left(\frac{r^2}{2} + r(n-r) + \frac{(n-r)^2}{2}\right) \\
            &= \frac{r-1}{r}\left(\frac{r^2 + 2rn + n^2 -2nr + r^2}{2}\right) = \frac{r-1}{r} \frac{n^2}{2}.
        \end{align}
\end{itemize}
\hspace{\fill}\qed

\subsection{Turánovy grafy}\label{turan}
Pro $n$, $r < n$. $T(n,r)$ je $r$-partitní úplný graf. Označíme-li strany $S_1, \dots, S_r$, pak $|S_i - S_j| \leq 1$,
$|S_i| = \floor*{\frac{n}{r}}$ nebo $\ceil*{\frac{n}{r}}$. Takový graf má potom
\[
    \frac{r(r-1)}{2}\left(\frac{n}{r}\right)^2 = \frac{r-1}{r}\frac{n^2}{2}, \, n=k \cdot r,
\]
hran.

\subsection{Tvrzení počtu hran a Turánově grafu}
Každý $G = (v,E) \in \S$ bez $K_{r+1}$ s největším počtem hran je $T(n,r)$.

Důkaz. Na $V$ definujme $\R$: $u\R v \iff \bc{u,v} \not\in E$.\\
$\R$ je reflexivní, protože nemáme smyčky. $\R$ je symetrické, protože se jedná o neorientovaný graf. Teď je potřeba 
ověřit tranzitivitu, tj. $\left(\bc{x,y} \not\in E, \bc{y,z} \not\in E\right) \implies \bc{x,z} \not\in E$. \\
Dokažme sporem. Kdyby $\bc{x,y} \not\in E$ a $\bc{y,z} \not\in E$ a $\bc{x,z} \in E$.
\begin{enumerate}[1)]
    \item $d(y) \geq d(x)$ (obdobně $d(y) \geq d(z)$). Sporem. Kdyby $d(y) < d(x)$.
        \begin{figure}[H]
            \centering
            \begin{tikzpicture}[scale=1, dot/.style={circle, fill=black, inner sep=1pt, minimum size=4pt}]
                \node[dot, label=left:$x$] (x) at (0,0) {};
                \node[dot, label=below:$y$] (y) at (1,-1) {};
                \node[dot, label=right:$z$] (z) at (2,0) {};

                \node[dot] (a) at (-1, -0.5) {};
                \node[dot] (b) at (-0.5, -0.9) {};
                \node[dot] (c) at (-0.2, -1.4) {};
                \node[dot] (d) at (0.6, -1.7) {};
                \node[dot] (e) at (1.4, -1.7) {};

                \draw [black!30!yellow] (x) to (a);
                \draw [black!30!yellow] (x) to (b);
                \draw [black!30!yellow] (x) to (c);

                \draw [black!30!green] (y) to (b);
                \draw [black!30!green] (y) to (c);
                \draw [black!30!yellow] (y) to (d);
                \draw [black!30!yellow] (y) to (e);

                \draw[thick, white!50!black, decorate, decoration={crosses, segment length=4pt}] (x) to (y);
                \draw[thick, white!50!black, decorate, decoration={crosses, segment length=4pt}] (z) to (y);
                \draw[thick, black, bend left] (x) to (z);
            \end{tikzpicture}
        \end{figure}
        \textcolor{black!30!yellow}{Neighbourhood} $N(v) = \bc{u \mid \bc{u,v} \in E}$. \\ \textcolor{black!30!green}{Z 
        $G$ odstraníme hrany $\bc{y,t}, t \in N(y)$ a přidáme $\bc{y,u}, u \in N(x)$. Tím dostaneme $G^\prime$, to má 
        více hran jak $G$.} \\
        $G^\prime$ nemá $K_{r+1}$, protože ani původní graf nebyl $K_{r+1}$. Což je spor.
        \hspace{\fill}\qed
    \item $G^{\prime \prime} = G \setminus \bc{x,y,z}$. $m(G) \leq m(G^\prime) + d(x) + d(y) + d(z) - 1$ ($-1$ za hranu 
        $\bc{x,z}$).
        \begin{figure}[H]
            \centering
            \begin{tikzpicture}[scale=1, dot/.style={circle, fill=black, inner sep=1pt, minimum size=4pt}]
                \node[dot, label=left:$x$] (x) at (0,0) {};
                \node[dot, label=below:$y$] (y) at (1,-1) {};
                \node[dot, label=right:$z$] (z) at (2,0) {};

                \draw[thick, white!50!black, decorate, decoration={crosses, segment length=4pt}] (x) to (y);
                \draw[thick, white!50!black, decorate, decoration={crosses, segment length=4pt}] (z) to (y);
                \draw[thick, black, bend left] (x) to (z);
            \end{tikzpicture}
        \end{figure}
        $G^{\prime \prime \prime}$ z $G$ odstraníme hrany $\bc{x,t}, t \in N(x)$ a $\bc{z,v}, v \in N(z)$ a přidáme 
        hrany $\bc{x,u}, u \in N(y)$ a $\bc{z,u}, u \in N(y)$.
        \begin{equation*}
            m(G^{\prime \prime \prime}) = m(G^{\prime \prime}) + 3d(y) > m(G^{\prime \prime}) + d(x) + d(y) + d(z) - 1
            \geq m(G)
        \end{equation*}
        $G^{\prime \prime}$ nemá $K_{r+1}$, což je spor.
        \hspace{\fill}\qed
\end{enumerate}
$\R$ je tedy ekvivalence. Třídy ekvivalence $\R$ jsou maximální množiny, že graf jimi indukovaný nemá hranu. $G$ má 
nejvíce hran, tj. $G$ má $K_r$, stran má $r$, je tedy úplný $r$-partitní graf.

Potřebujeme $||S_i| - |S_j|| \leq 1$. Dokažme sporem. Kdyby ne, tak $|S_1| \geq |S_2| + 2$. Označme $|S_1| = n_1$ a 
$|S_2| = n_2$.

Graf měl původně $n_1 \cdot n_2$ hran. Nově má
\begin{equation*}
    (n_1 - 1)(n_2 + 1) = n_1n_2 \underbrace{\underbrace{- n_2 + n_1}_{\geq 2} - 1}_{\geq 1}.
\end{equation*}
A to je \hyperref[turan]{Turánův} graf.
\hspace{\fill}\qed