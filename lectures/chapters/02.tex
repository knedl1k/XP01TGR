\section{Souvislé grafy}

\subsection{\texorpdfstring{$k$}{k}-souvislost}
$G = (V,E) \in \S$. Řekněme, že $G$ je $k$-souvislý, pokud $|V| > k$ a pro každou $X \subseteq V$, $|X| = k-1$ je 
$G \setminus X$ souvislý.\\
Mějmě
\begin{figure}[H]
    \centering
    \begin{minipage}[c]{0.3\textwidth}
        \begin{figure}[H]
            \begin{tikzpicture}[scale=1]
                \graph [layered layout, orient=west]{
                    1 -- {2, 3};
                    2 -- 4;
                    3 -- 4;
                    4 -- 5;
                };
            \end{tikzpicture}
        \end{figure}
        Je souvislý, ale ne 2-souvislý.
    \end{minipage}%
    \hspace{0.1\textwidth}
    \begin{minipage}[c]{0.3\textwidth}
        \begin{figure}[H]
            \begin{tikzpicture}[scale=1]
                \graph [tree layout, orient=west]{
                    1 -- {2, 4, 5};
                    2 -- 3;
                    3 -- 4;
                    4 -- 5;
                };
            \end{tikzpicture}
        \end{figure}
        Je 2-souvislý.
    \end{minipage}
\end{figure}
Každý graf je $0$-souvislý, i nesouvislý graf je $0$-souvislý.\\
$1$-souvislý je každý souvislý graf.

\subsection{Souvislost v grafu}
Souvislost v grafu $G$ je největší $k$ takové, že $G$ je $k$-souvislý. Značíme $\kappa(G)$. \\ Úplný graf má 
$\kappa(G) = |V|-1$.

\subsection{Vrcholový řez}\label{vrchRez}
Vrcholový řez grafu $G \in \S$ je množina vrcholů $X \subsetneqq V$, že $G \setminus X$ je nesouvislý.

\subsection{Vztah neúplnosti a vrcholového řezu}
Je-li $G \in S$, $G$ není úplný, pak $\kappa(G)=k$ právě tehdy, když nemá vrcholový řez o $k-1$ vrcholech a má vrcholový 
řez o $k$ vrcholech.

\subsection{Věta o vztahu podgrafu a souvislosti}
Mějme $G \in \S$, $G = (V,E)$, splňující $d(G) \geq 4k$. Pak $G$ obsahuje podgraf, který je $k$-souvislý.

\bb{Důkaz}. 
\begin{itemize}
    \item Pro $k=0$ triviální. Všechny grafy jsou $0$-souvislé.
    \item Pro $k=1$: Pokud $\frac{2m}{n} \geq 4k$, tedy $m \geq 1$ (takže má hranu), tak sama hrana je $1$-souvislý 
    podgraf.
    \item Pro $k \geq 2$: tj. $\frac{2m}{n} \geq 4k$
    \begin{align*}
        2m &\geq 4kn \\
        m &\geq 2kn \\
        m &\geq 4n \, (\text{dosazeno } k \geq 2)
    \end{align*}
\end{itemize}
Průběh důkazu $d(G) \geq 4k, k\geq 2$ $\xRightarrow{\text{Lemma 1}} (\text{i}), (\text{ii}) \xRightarrow{\text{Lemma 2}} 
G \text{ má } k \text{-souislost.}$
\subsubsection{Pomocné lemma 1}
Pokud $k \geq 2$ a $d(G) \geq 4k$, pak
\begin{enumerate}[(i), noitemsep]
    \item $n \geq 2k-1$
    \item $m \geq (2k-3)(n-k+1)+1$
\end{enumerate}
\bb{Důkaz}. (i) Kdyby ne, tak $n<2k-1$.
\begin{align*}
    n+1 &< 2k\\
    \frac{n+1}{2} &<k
\end{align*}
Teď použijme předpoklad $m \geq 2kn > (n+1)n$ % TODO: dodělat argumentaci, jaký předpoklad? nahrávka

(ii) Mějme
\begin{align*}
    m \geq 2kn - ((2k-3)(n-k+1)+1) &= 2kn - (2kn - 2k^2 + 2k - 3n+ 3k-3+1)\\
    &= 2k^2 - 5k + 3n + 2
\end{align*}
Teď aplikujme již dokázané (i):
\begin{align*}
    2k^2 - 5k + 3n + 2 \geq 2k^2 - 5k + 6k - 3 + 2 = 2k^2 + k - 1 
\end{align*}
\begin{figure}[H]
    \begin{minipage}[c]{0.4\textwidth}
        Vyšetřeme průběh funkce
        \begin{figure}[H]
            \begin{tikzpicture}[>=latex, scale=1]
                % x axis
                \draw[->] (-2.5,0) -- (2.5,0) node[below] {$k$};
                % y axis
                \draw[->] (0,-1.5) -- (0,3.5) node[left] {$y$};

                \draw[thick, red, domain=-1.5:1.2, samples=200] 
                    plot (\x, {2*\x*\x + \x - 1});

                \foreach \x in {-2, -1, 0.5, 1, 2}
                    \draw[shift={(\x,0)}] (0pt,2pt) -- (0pt,-2pt) node[below] {\footnotesize $\x$};
                \foreach \y in {-1, 1, 2, 3}
                    \draw[shift={(0,\y)}] (2pt,0pt) -- (-2pt,0pt) node[left] {\footnotesize $\y$};
                \node[below left] at (0,0) {\footnotesize $0$};
            \end{tikzpicture}
        \end{figure}
    \end{minipage}%
    \hspace{0.05\textwidth}
    \begin{minipage}[c]{0.5\textwidth}
        Funkce je očividně konvexní, a protože nás zajímá průběh funkce na $k \geq 2$, můžeme prohlásit, že 
        $2k^2 + k - 1 > 0$. \hspace{\fill} \qed
    \end{minipage}
\end{figure}

\subsubsection{Pomocné lemma 2}
Pokud $G$ splňuje (i) a (ii), tak $G$ má $k$-souvislý podgraf.

\bb{Důkaz}. $G$ není $k$-souvislý.\\
Indukcí podle $|V| = n$.

Základní krok: n $\stackrel{\text{(i)}}{=}2k-1$, $m \geq (2k-3)(n-k+1)+1$.\\
Dosaďme $k = \frac{n+1}{2}$:
\begin{equation}
    m \geq (n+1-3)\left(n-\frac{n+1}{2}+1\right)+1 = \frac{(n-2)(n+1)}{2}+1 = \frac{n(n-1)}{2}
\end{equation}
A tedy graf je úplný na $n$ vrcholech. Teď potřebujeme $n>k$.
\begin{equation}
    n = 2k-1 = k + \underbrace{k-1}_{\geq 1} \geq k+1
\end{equation}
Indukční krok: Každý graf $G^\prime$ splňující (i) a (ii) s méně než $n$ vrcholy (s alespoň $2k-1$ vrcholy) má 
$k$-souvislý podgraf.

Vezmeme $G$ splňující (i) a (ii) s $n$ vrcholy.
\begin{enumerate}[(a)]
    \item Kdyby $\delta(G) \leq 2k-3$, tak $v \in V$ s $d_G(v) \leq 2k-3$.\\
    $G \setminus v = G_1$, $n_1=n-1$,
    \begin{equation*}
        m_1 \geq m-(2k-3) \geq (2k-3)(n-k+1)+1 - (2k-3) = (2k-3)(\underbrace{n-1}_{n_1}-k+1)+1
    \end{equation*}
    Tudíž $G_1$ má $k$-souvislý podgraf, tedy i ho má $G$.
    \item Ať $\delta(G) > 2k-3m$, $\delta(G) \geq 2k-2$; $\forall v \in G, d_G(v)\geq 2k-2$.\\
    $G$ není $k$-souvislý, tj. $X \subseteq V$, $|X|=k-1$ a $X$ je \hyperref[vrchRez]{vrcholový řez}.
\end{enumerate}

% TODO: dodělat diagram

$G_1$ graf indukovaný $C_1$ v $X$ má alespoň $2k-1$ vrcholů.

Kdyby $G_1$ i $G_2$ nesplňovaly (ii), $G_i$ má $n_i$ vrcholů a $m_i$ hran, $i=1,2$.
\begin{equation}
    m_i \not\geq (2k-3)(n_i-k+1)+1, \hspace{1.5em} \text{tj. } m_i \leq (2k-3)(n_i-k+1)
\end{equation}
$m_1+m_2 \geq m$ víme. $n_1+n_2 = n+(k-1)$, počítali jsme vrcholy v $X$ dvakrát.
\begin{align*}
    m \leq n_1 + n_2 \leq (2k-3)(n_1-k+1)+(2k-3)(n_2-k+1) &= (2k-3)(n_1+n_2-2k+2) \\
    &= (2k-3)(n+(k-1)-2k+2) \\
    &= (2k-3)(n-k+1)
\end{align*}
Tedy spor s (ii). 
\hspace{\fill} \qed

\newpage
\subsection{Artikulace}\label{artikulace}
Vrchol $v$ grafu $G$ se nazývá artikulace, jestliže $G \setminus v$ má více komponent souvislosti, než $G$.\\
\bb{Platí.} $G \in \S$ s alespoň 3 vrcholy je 2-souvislý $\iff$ je 1-souvislý a nemá artikulaci.
\begin{figure}[H]
    \centering
    \begin{minipage}[c]{0.3\textwidth}
        \begin{figure}[H]
            \begin{tikzpicture}[scale=1]
                \graph [tree layout, orient=south]{ 
                    1 -- {2,3};
                    2 -- 3;
                    3 -- {4,5};
                    4 -- {5,6};
                    5 -- 6;
                };
            \end{tikzpicture}
        \end{figure}
        Není 2-souvislý.
    \end{minipage}%
    \hspace{0.1\textwidth}
    \begin{minipage}[c]{0.3\textwidth}
        \begin{figure}[H]
            \begin{tikzpicture}[scale=1]
                \graph [spring layout, node distance=1cm]{
                    1 -- {2, 5, 6};
                    2 -- {3, 6};
                    3 -- 4;
                    4 -- {5, 6};
                    5 -- 6;        
                };
            \end{tikzpicture}
        \end{figure}
        Je 2-souvislý.
    \end{minipage}
\end{figure}

\subsection{Operace nad 2-souvislými grafy}
Mějme operace
\begin{enumerate}[(a)]
    \item $G \in \S$ a $e \in \bc{u,v}$; $u, v \in V(G)$, $e \not\in E(G)$, pak
    $G + e$ je graf s $V(G)$ a $E(G) \cup \bc{e}$. \\ Je-li $G$ 2-souvislý, tak $G+e$ je 2-souvislý.
    \item $G \in \S$, $G = (V,E)$, $e \in E$, pak
    $G \% e = \left(V \cup \bc{w}, (E \setminus \bc{e}) \cup \bc{e_1, e_2}\right)$. \\
    \enquote{Do hrany $e$ vložíme vrchol se stupněm 2.}
\end{enumerate}

\subsection{Tvrzení o 2-souvislých grafech a kružnicích}
Každý 2-souvislý graf obsahuje kružnici. 

\bb{Důkaz.} Každý 2-souvislý graf je souvislý. Kdyby souvislý neobsahoval kružnici, jedná se o strom. A každý strom s 
alespoň 3 vrcholy má artikulaci. Protože stromy nemohou být 2-souvislé, a zároveň všechny ostatní souvislé grafy 
obsahují kružnici, i každý 2-souvislý graf obsahuje kružnici.
\hspace{\fill}\qed

\subsection{Věta o vrcholech na společné kružnici}
$G \in \S$, $G = (V,E)$ je 2-souvislý právě tehdy, když každé 2 vrcholy $u \not= v$ leží na společné kružnici.

\bb{Důkaz.}\\
\enquote{$\Leftarrow$}: Předpokládejme, že pro každé $u \not= v$ existuje kružnice $K$, která je obsahuje.

To znamená, že graf je souvislý. Musíme ještě dokázat, že v něm neexistuje \hyperref[artikulace]{artikulace}.
Kdyby graf měl artikulaci $v$: 
\begin{figure}[H]
    \centering
    \begin{tikzpicture}[scale=1]
        \node[ellipse, draw, thick, minimum width=2.5cm, minimum height=1.5cm, 
            label={[xshift=-0.5cm, yshift=-0.8cm]$C_1$}, label={[xshift=0cm, yshift=-1.5cm]$x$}] (C1) at (-3,0) {};
        \node[ellipse, draw, thick, minimum width=2.5cm, minimum height=1.5cm, 
            label={[xshift=0.5cm, yshift=-0.8cm]$C_2$}, label={[xshift=0cm, yshift=-1.5cm]$y$}] (C2) at (3,0) {};
        \node[circle, fill, inner sep=1.5pt, label=below:$v$] (v) at (0,0) {};

        \draw (C1.east) -- (v);
        \draw (C2.west) -- (v);

        \node[anchor=west] at (1, 2.5) {komponenty souvislosti};
        \node[anchor=west] at (2.2, 2) {$G \setminus v$};

        \draw[-Stealth, thick] (1.5, 2.2) -- (-2, 0.8);

        \node[anchor=west, text width=5cm] at (5, -0.5) {alespoň jeden vrchol};
        \node[anchor=west, text width=5cm] at (5.2, -1.0) {$u \in C_1$ a $C_2$.};
    \end{tikzpicture}
\end{figure}
Znamenalo by to, že v jedné komponentě souvislosti by ležely alespoň 2 vrcholy (protože máme minimálně 3 vrcholy). 
Zároveň ale vrchol $x \in C_1$ a $y \in C_2$ rozhodně neleží na společné kružnici, tudíž graf nemůže mít artikulaci, 
takže $G$ je 2-souvislý. 
\hspace{\fill}\qed

\enquote{$\Rightarrow$}: Předpokládejme, že $G$ je 2-souvislý. Dokažme indukcí podle vzdálenosti $d(u,v)$.
\begin{enumerate} [(a)]
    \item Základní krok: $u, v$ s $d(u,v)=1$. \\
    Budeme se snažit ukázat, že když zrušíme hranu, souvislost zůstane. 
    \begin{enumerate}[(1)]
        \item $G \setminus e$ je souvislý. Kdyby ne, tak
        \begin{figure}[H]
            \centering
            \begin{tikzpicture}[scale=1]
                \node[ellipse, draw, thick, minimum width=2.5cm, minimum height=1.5cm, 
                    label={[xshift=-0.5cm, yshift=-0.8cm]$C_1$}, label={[xshift=0cm, yshift=-1.5cm]$x$}] (C1) at (-3,0) {};
                \node[ellipse, draw, thick, minimum width=2.5cm, minimum height=1.5cm, 
                    label={[xshift=0.5cm, yshift=-0.8cm]$C_2$}, label={[xshift=0cm, yshift=-1.5cm]$y$}] (C2) at (3,0) {};

                \coordinate (StartEdge) at (-1.5, 0);
                \coordinate (EndEdge) at (1.5, 0);

                \draw[thick, dashed, gray] (C1.east) -- (C2.west);

                \coordinate (CrossCenter) at (0, 0);
                
                \draw[very thick, red] (CrossCenter) ++ (-0.5, 0.5) -- ++ (1, -1);
                \draw[very thick, red] (CrossCenter) ++ (-0.5, -0.5) -- ++ (1, 1);
            \end{tikzpicture}
        \end{figure}
        Přitom $G$ má alespoň 3 vrcholy, tedy v jedné komponentě leží alespoň 2 vrcholy. \ii{BÚNO} exsistuje 
        $x \in C_1$, $x \not= u$, tj. $u$ je \hyperref[artikulace]{artikulace}. Což je spor. % TODO: s čím? doplnit
        Takže $G \setminus e$ je souvislý. Tedy existuje cesta $P$ z $u$ do $v$. Pak $P$ je kružnice obsahující $u,v$. 
    \end{enumerate}
    \item Indukční předpoklad: Pro každé $x,y$ s $d(x, y) = n \geq 1$ existuje kružnice obsahující $x,y$.
    \item Indukční krok: Vezměme libovolné $u,v$ s $d(u,v) = n+1$. Vyberme nejkratší cestu:
    \begin{figure}[H]
        \centering
        \begin{tikzpicture}[scale=1]
            \node[circle, draw, fill=white, inner sep=1.5pt, label=left:$u$] (u_node) at (-4, 1) {};
            \node[circle, draw, fill=white, inner sep=1.5pt, label=right:$v$] (v_node) at (4, 1) {};
            \node[circle, fill, inner sep=1.5pt, label=below:$x_1$] (x1_node) at (-1.5, 1) {};
            \node[circle, fill, inner sep=1.5pt, label=below:$x_n$] (xn_node) at (1.5, 1) {};
            \node[circle, draw, fill=green!50!black, inner sep=1.5pt, label=below:$w$] (w_node) at (0, -2) {};
            \draw[thick] (u_node) -- (x1_node);
            \node at (0, 1) {$\dots$};
            \draw[thick] (xn_node) -- (v_node);
            \draw[red, thick] (u_node.north east) .. controls (-2.5, 1.5) and (-2.5, 1.5) .. (x1_node.north west);
            \draw[red, thick] (v_node.north) .. controls (2, 1.5) and (1, 1.5) .. (x1_node.north);
            \draw[red, very thick, -] (u_node.south west) .. controls (-4.5, -0.5) and (-2.5, -2) .. (w_node.west);
            \node[below left=0.1cm of u_node, color=green!70!black] {$P$};
            \draw[green!70!black, very thick, -] (u_node.south) .. controls (-3, -1) and (-1.5, -2) .. (w_node.north west);
            \draw[orange, very thick] (v_node.north) .. controls (2, 2) and (0, 2) .. (x1_node.north);
            
            \draw[orange, very thick] (x1_node.south) -- (w_node.north);
            \draw[green!70!black, very thick, -] (w_node.north east) .. controls (1.5, -2) and (3, -1) .. (v_node.south);
            \draw[orange, very thick, -] (w_node.north east) .. controls (1.5, -2) and (2, -1) .. (v_node.south);
            \node[below right=0.1cm of v_node, color=orange] {$K_1$};
            \draw[red, thick, -] (w_node.east) .. controls (1.5, -2.5) and (3.5, -2) .. (v_node.south);
        \end{tikzpicture}
    \end{figure}
    Použijme I.P.: tj. existuje kružnice $K_1$ obsahující $x_1, v$. $x_1$ není \hyperref[artikulace]{artikulace}, tj. 
    existuje cesta $P$ z $u$ do $v$ neobsahující $x_1$. $w$ je prvním vrcholem cesty $P$, který leží v $K_1$ 
    % TODO: dodělat
\end{enumerate}
\hspace{\fill}\qed

\subsection{Tvrzení o 2-souvislých grafech a \% operaci}
$G \in \S$ je 2-souvislý právě tehdy, když $G \% e$, $e \in E(G)$ je 2-souvislý.

\bb{Důkaz.}\\
\enquote{$\Rightarrow$}: Předpokládejme, že $G$ je 2-souvislý, tj. souvislý a nemá \hyperref[artikulace]{artikulaci}.

Vrchol $w$, který vložíme do hrany $e$, není artikulace. A žádný jiný se nemohl stát artikulací, to by už musely být 
artikulací předtím, a tedy by se v prvé řadě nejednalo o 2-souvislý.
\hspace{\fill}\qed

\enquote{$\Leftarrow$}: Předpokládejme, že $G \% e$ je 2-souvislý, tj. každé 2 vrcholy leží na společné kružnici.
\[
    x, y \in V(G) \dots \text{existuje } K \text{ v } G \% e \text{ obsahující } x, y
    \begin{cases}
        K \text{ neobsahuje } e_1, e_2 & K \text{ je kružnice } G. \\
        \\
        K \text{ obsahuje } e_1, e_2 & \text{ z } K \text{ odstraníme } e_1, e_2, \\ 
        & \text{ nahradíme } e \text{ a máme } K^\prime.
    \end{cases}
\]
$K^\prime$ je kružnice v $G$. 
\hspace{\fill}\qed

\subsection{Algoritmus sestrojení 2-souvislého grafu}
Každý 2-souvislý graf $G \in \S$, $G = (V,E)$ je možné sestrojit postupem:
\[ G_0 \coloneq K \text{ je nějaká kružnice}\]
Máme-li $G_i$, že $G_i \not= G$, tak $G_{i+1}$ je $G_i$, ke kterému přidáme cestu $P$ (v $G$), která vede mezi 2 vrcholy
z $G_i$ a zároveň všechny vrcholy této cesty nejsou v $G_i$.

\subsection{Příklad sestrojení 2-souvislého grafu}
Mějme 2-souvislý graf, tj. bez artikulace: 
\begin{figure}[H]
    \centering
    \begin{tikzpicture}[scale=1]
        \graph [spring layout, node distance=1cm]{ 
            1 -- {2, 3, 4};
            2 -- {3, 5};
            3 -- {4, 6};
            4 -- 5;
            5 -- 6;
        };
    \end{tikzpicture}
\end{figure}
Začněme $G_0$:
\begin{figure}[H]
    \centering
    \begin{tikzpicture}[scale=1]
        \graph [spring layout, node distance=1cm, orient=north]{ 
            1 -- {2, 3};
            2 -- 3;
        };
    \end{tikzpicture}
\end{figure}
Přidáme cestu z 1 do 2, tedy $G_1$:
\begin{figure}[H]
    \centering
    \begin{tikzpicture}[scale=1]
        \graph [spring layout, node distance=1cm, orient=north]{ 
            1 -- {2, 3, 4};
            2 -- {3, 5};
            4 -- 5;
        };
    \end{tikzpicture}
\end{figure}
Teď přidáme cestu z 3 do 4, $G_2$:
\begin{figure}[H]
    \centering
    \begin{tikzpicture}[scale=1]
        \graph [spring layout, node distance=1cm, orient=north]{ 
            1 -- {2, 3, 4};
            2 -- {3, 4, 5};
            4 -- 5;
        };
    \end{tikzpicture}
\end{figure}
A posledně z 3 do 5, $G_3 = G$:
\begin{figure}[H]
    \centering
    \begin{tikzpicture}[scale=1]
        \graph [spring layout, node distance=1cm]{ 
            1 -- {2, 3, 4};
            2 -- {3, 5};
            3 -- {4, 6};
            4 -- 5;
            5 -- 6;
        };
    \end{tikzpicture}
\end{figure}
% TODO: přidat i druhý postup z fotky

\subsection{Komponenty 2-souvislosti - blok}
Mějme $G \in \S$, $G = (V,E)$, pak $A \subseteq V(G)$ se nazývá \bb{blok}, jestliže je maximální podmnožina taková, že 
jí indukovaný podgraf je 2-souvislý. \\
\ii{Pozn. maximální v tomto kontextu neznamená nejpočetnější, nýbrž, že do takové podmnožiny již nelze přidat další 
vrchol.}
\begin{figure}[H]
    \centering
    \begin{tikzpicture}[scale=1] % TODO: dodělat kroužky ke komponentám souvislosti
        \graph [spring layout, node distance=1cm]{ 
            1 -- {2, 3};
            2 -- 3;
            3 -- {4, 6};
            4 -- {5,6};
            5 -- 6;
        };
    \end{tikzpicture}
\end{figure}
Když nejsou jednotlivé bloky vzájemně disjunktní, tak jejich průnik je \hyperref[artikulace]{artikulace}.
