\section{Hranově souvislé grafy}
\subsection{Hranový řez}\label{hranRez}
Množině $F \subseteq E$, že $G \setminus F$ je ne\hyperref[souvisly]{souvislá}, se říká hranový řez.

\subsection{Hranová souvislost}\label{hran-souvislost}
Máme $G \in \S$, $G = (V,E)$, pak $G$ je $k$-hranově souvislý, jestliže neexistuje $F \subseteq E$, $|F| \leq k-1$, 
taková, že $G \setminus F$ je ne\hyperref[souvisly]{souvislá}.

Hranová souvislost grafu $G$, značíme $\lambda(G)$, je největší $k$, že $G$ je $k$-hranově souvislý.\\
\ii{Pozn. největší znamená, že nemá \hyperref[hranRez]{hranový řez} s $\lambda(G)-1$ hranami, ale má s $\lambda(G)$ 
hranami.}

Mějme 2-hranově souvislý graf:
\begin{figure}[H]
    \centering
    \begin{tikzpicture}[
        vertex/.style={fill=black, circle, inner sep=0pt, outer sep=0pt}
    ]

        \graph [nodes={vertex}] {
            1 [at={(0, 0)}];
            2 [at={(1, 2)}];
            3 [at={(2, 2)}];
            4 [at={(3, 3)}];
            5 [at={(2, 3)}];
            6 [at={(3, 4)}];

            1 -- {2,3};
            2 -- 3;
            3 -- {4, 5, 6};
            4 -- 6;
            5 -- 6;
        };
    \end{tikzpicture}
\end{figure}

\subsection{Most}\label{most}
Nazvěme most hranu $e\in E(G)$, že $\bc{e}$ je \hyperref[hranRez]{hranový řez}.

\subsection{Souvislost krajních vrcholů a mostů}
Každý most má alespoň jeden krajní vrchol, který je \hyperref[artikulace]{artikulace}. 

\dukaz Nechť $e = \{u, v\}$ je most v souvislém grafu $G$, kde $|V(G)| \geq 3$. Dle definice mostu platí, že graf 
$G \setminus e$ není \hyperref[souvisly]{souvislý} a skládá se ze dvou komponent souvislosti. Označme $K_u$ komponentu 
obsahující vrchol $u$ a $K_v$ komponentu obsahující vrchol $v$.

Protože má graf $G$ alespoň 3 vrcholy, musí alespoň jedna z komponent $K_u$ nebo $K_v$ obsahovat více než jeden vrchol. 
\ii{BÚNO} předpokládejme, že $|V(K_u)| \geq 2$.

To znamená, že ve komponentě $K_u$ existuje vrchol $w$ různý od $u$ (tj. $w \in V(K_u), w \neq u$). Protože $e$ je 
\hyperref[most]{most}, jediná \hyperref[stc]{cesta} v grafu $G$ z vrcholu $w$ do vrcholu $v$ vede přes hranu $e$, a tedy 
nutně prochází vrcholem $u$.

Pokud z grafu $G$ odstraníme vrchol $u$, neexistuje žádná cesta mezi $w$ a $v$, protože jediná spojnice byla přerušena. 
Graf $G \setminus u$ tedy není souvislý (vrcholy $w$ a $v$ leží v různých komponentách).

Z toho plyne, že vrchol $u$ je \hyperref[artikulace]{artikulace}.
\hspace{\fill}\qed

\subsection{Základní vlastnosti hranově souvislých grafů}
\vspace{-1em}
\begin{flalign*}
    &G \text{ je \hyperref[hran-souvislost]{$0$-hranově souvislý} pro každé } G. && \\
    &G \text{ je } 1\text{-hranově souvislý} \iff G \text{ je \hyperref[souvisly]{souvislý}.} && \\
    &G \text{ je } 2\text{-hranově souvislý} \iff G \text{ je souvislý a nemá \hyperref[most]{most}.}&& \\
\end{flalign*}

\subsection{Tvrzení o hranové a vrcholové souvislosti}
Platí, že $\kappa(G) \leq \lambda(G)$.
% TODO: přepsat ze skript