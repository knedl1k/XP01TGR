\section{Hranově souvislé grafy}
\subsection{Hranový řez}\label{hranRez}
Množině $F \subseteq E$, že $G \setminus F$ je ne\hyperref[souvisly]{souvislá}, se říká hranový řez.

\subsection{Hranová souvislost}\label{hran-souvislost}
Máme $G \in \S$, $G = (V,E)$, pak $G$ je $k$-hranově souvislý, jestliže neexistuje $F \subseteq E$, $|F| \leq k-1$, taková, že $G \setminus F$ je ne\hyperref[souvisly]{souvislá}.

Hranová souvislost grafu $G$, značíme $\lambda(G)$, je největší $k$, že $G$ je $k$-hranově souvislý.\\
\ii{Pozn. největší znamená, že nemá \hyperref[hranRez]{hranový řez} s $\lambda(G)-1$ hranami, ale má s $\lambda(G)$ hranami.}

Mějme 2-hranově souvislý graf:
\begin{figure}[H]
    \centering
    \begin{tikzpicture}[
        vertex/.style={fill=black, circle, inner sep=0pt, outer sep=0pt}
    ]

        \graph [nodes={vertex}] {
            1 [at={(0, 0)}];
            2 [at={(1, 2)}];
            3 [at={(2, 2)}];
            4 [at={(3, 3)}];
            5 [at={(2, 3)}];
            6 [at={(3, 4)}];

            1 -- {2,3};
            2 -- 3;
            3 -- {4, 5, 6};
            4 -- 6;
            5 -- 6;
        };
    \end{tikzpicture}
\end{figure}

\subsection{Most}\label{most}
Nazvěme most hranu $e\in E(G)$, že $\bc{e}$ je \hyperref[hranRez]{hranový řez}.

\subsection{Souvislost krajních vrcholů a mostů}
Každý most má alespoň jeden krajní vrchol, který je \hyperref[artikulace]{artikulace}. 

\dukaz Nechť $e = \{u, v\}$ je most v souvislém grafu $G$, kde $|V(G)| \geq 3$. Dle definice mostu platí, že graf $G \setminus e$ není \hyperref[souvisly]{souvislý} a skládá se ze dvou komponent souvislosti. Označme $K_u$ komponentu obsahující vrchol $u$ a $K_v$ komponentu obsahující vrchol $v$.

Protože má graf $G$ alespoň 3 vrcholy, musí alespoň jedna z komponent $K_u$ nebo $K_v$ obsahovat více než jeden vrchol. \ii{BÚNO} předpokládejme, že $|V(K_u)| \geq 2$.

To znamená, že ve komponentě $K_u$ existuje vrchol $w$ různý od $u$ (tj. $w \in V(K_u), w \neq u$). Protože $e$ je \hyperref[most]{most}, jediná \hyperref[stc]{cesta} v grafu $G$ z vrcholu $w$ do vrcholu $v$ vede přes hranu $e$, a tedy nutně prochází vrcholem $u$.

Pokud z grafu $G$ odstraníme vrchol $u$, neexistuje žádná cesta mezi $w$ a $v$, protože jediná spojnice byla přerušena. Graf $G \setminus u$ tedy není souvislý (vrcholy $w$ a $v$ leží v různých komponentách).

Z toho plyne, že vrchol $u$ je \hyperref[artikulace]{artikulace}.
\hspace{\fill}\qed

\subsection{Základní vlastnosti hranově souvislých grafů}
\vspace{-1em}
\begin{flalign*}
    &G \text{ je \hyperref[hran-souvislost]{$0$-hranově souvislý} pro každé } G. && \\
    &G \text{ je } 1\text{-hranově souvislý} \iff G \text{ je \hyperref[souvisly]{souvislý}.} && \\
    &G \text{ je } 2\text{-hranově souvislý} \iff G \text{ je souvislý a nemá \hyperref[most]{most}.}&& \\
\end{flalign*}

\subsection{Tvrzení o hranové a vrcholové souvislosti}\label{hranVrchSouv}
Platí, že $\kappa(G) \leq \lambda(G)$.

\dukaz
\subsubsection{Pomocné lemma 1}
Pro každý \hyperref[prosty]{prostý} graf $G$ bez smyček a jeho libovolnou hranu $e$ platí
\begin{equation}
    \alpha(G) - 1 \leq \alpha(G \setminus e) \leq \alpha(G).
\end{equation}
\dukaz Jestliže hrana $e$ leží v některém neméně početném \hyperref[hranRez]{hranovém řezu} $F$, pak $F \setminus \bc{e}$ je hranový řez grafu $G \setminus e$. V opačném případě mají grafy $G$ a $G \setminus e$ stejnou \hyperref[hran-souvislost]{hranovou souvislost}.

\subsubsection{Pomocné lemma 2}
Pro každý \hyperref[prosty]{prostý} graf $G$ bez smyček a jeho libovolnou hranu $e$ platí
\begin{equation}
    \kappa(G) - 1 \leq \kappa(G \setminus e) \leq \kappa(G).
\end{equation}
\dukaz Stačí dokázat trošku jiné tvrzení: Pro každý prostý graf $H$ bez smyček platí
\begin{equation}
    \kappa(H + e) \leq \kappa(H) + 1.
\end{equation}
(Graf $H + e$ má stejnou množinou vrcholů a jednu hranu $e$ navíc)

Uvažujme některý nejméně početný \hyperref[vrchrez]{vrcholový řez} $A$ grafu $H$; tj. $H \setminus A$ je ne\hyperref[souvisly]{souvislý}; označme jeho komponenty souvislosti $C_1, C_2, \dots, C_r$. Navíc $\kappa(H) = |A|$.

Pak v následujících případech platí $\kappa(H + e) = \kappa(H)$:
\begin{itemize}
    \item hrana $e$ má alespoň jeden krajní vrchol v množině $A$;
    \item hrana $e$ leží uvnitř některé z komponent souvislosti $C_1, C_2, \dots, C_r$;
    \item hrana $e$ spojuje dvě komponenty souvislosti a $r > 2$.
\end{itemize}
Uvažujme případ $r=2$ a alespoň jedna komponenta $C_1$ a $C_2$ je alespoň dvouprvková. Označme $x$ vrchol incidentní s hranou $e$, který leží v komponentě s alespoň 2 vrcholy. Pak $A^\prime = A \cup \bc{x}$ je \hyperref[vrchRez]{vrcholový řez} grafu $H + e$ a $\kappa(H + e) \leq \kappa(H) + 1$.

Zbývá případ, kdy $H \setminus A$ obsahuje dvě komponenty, obě jsou jednoprvkové a hrana $e$ je spojuje. Pak ale platí
\begin{equation}
    \kappa(H + e) \leq |V| - 1 = |V| - 2 + 1 = |A| + 1 = \kappa(H) + 1.
\end{equation}


Nyní se vraťme k důkazu \ref{hranVrchSouv}.

Indukcí podle $m = |E(G)|$: Jestliže $m < |V(G)| - 1$, je $G$ nesouvislý graf a $\kappa(G) = 0 = \lambda(G)$.

Předpokládejme, že $\lambda(G) > 0$ a předpokládejme, že tvrzení platí pro všechny grafy s méně než $m$ hranami. Zvolme $F$ některou nejméně početnou množinu hran takovou, že $G \setminus F$ je nesouvislý graf. Dále vyberme hranu $e$ z $F$. Pak $G \setminus e$ má méně hran, proto splňuje nerovnost $\kappa(G \setminus e) \leq \lambda(G \setminus e)$. Odtud
\begin{equation}
    \kappa(G) - 1 \leq \kappa(G \setminus e) \leq \lambda(G \setminus e) = \lambda(G) - 1.
\end{equation}
\hspace{\fill}\qed
