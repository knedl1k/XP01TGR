\section{Barvení}
\subsection{Hranové obarvení}\label{hranObar}
Je dán graf bez smyček $G = (V,E, \eps)$ a konečná množina $C$ tzv. barev. \ii{Hranové obarvení} $G$ je přiřazení 
$c: E \rightarrow C$, že žádné dvě hrany se společným krajním vrcholem nemají stejnou barvu.

Lze také říci, že se snažíme graf pokrýt vzájemně disjunktními párováními.

\subsection{Hranová barevnost}\label{hranChrom}
Hranová barevnost, též \ii{chromatický index}, grafu $G$ je nejmenší počet barev, kterými lze 
\hyperref[hranObar]{hranově obarvit} graf $G$, značíme ji $\chi^\prime(G)$.

Nutně platí, že
\begin{equation}
    \chi^\prime(G) \geq \Delta(G).
\end{equation}

\subsection{Věta o souvislosti hranové barevnosti a maximálním stupni grafu}\label{vizing}
\ii{Vizing}. Je dán graf bez smyček $G = (V, E, \eps)$. Pak
\begin{equation}
    \Delta(G) \leq \chi^\prime (G) \leq \Delta(G) + 1.
\end{equation}
Bez důkazu.

\subsection{Vrcholové obarevní}\label{vrchObar}
Je dán prostý neorientovaný graf $G = (V,E)$ bez smyček a konečná množina $B$ (nazývaná též množina barev). \ii{Obarvení 
vrcholů} grafu (též \ii{vrcholové obarvení} grafu) je $b: V \rightarrow B$, že žádné dva vrcholy spojené hranou nemají 
stejnou barvu.

$G$ nazveme $k$-barevný, jestliže se dá obarvit $k$ barvami.

\subsection{Barevnost grafu}\label{vrchChrom}
Barevnost grafu $G$, též \ii{chromatické číslo} grafu $G$, je nejmenší $k$ takové, že $G$ je $k$-barevný, značíme ji 
$\chi(G)$.

\subsection{Tvrzení o dvoubarevném grafu}
Graf $G$ je dvoubarevný právě tehdy, když $G$ neobsahuje kružnici liché délky, tedy $G$ je bipartitní.

\dukaz Ať $G = (V,E)$, $V = A \cup B$, tak, že:
\begin{align*}
    v \in A \iff \text{má barvu } 1 \\
    v \in B \iff \text{má barvu } 2 \\
\end{align*}
Stačí ukázat, že nemá-li $G$ kružnici liché délky, dá se obarvit 2 barvami.

Dokažme algoritmem barvení. Vyberme si počáteční vrchol $v$. Následně aplikujme algoritmus BFS (breath-first-search), 
kterým graf rozdělíme do několika úrovní. Každé úrovni přiřadíme barvu tak, že
\begin{equation}
    b(v) = 
    \begin{cases}
        1 \iff v \in H_{2i}, \\
        2 \iff v \in H_{2i+1}, \, \text{pro } i \in \N.   
    \end{cases}
\end{equation}

Kdyby $b$ nebylo obarvení; existuje $\bc{x, y} \in E$, kde $x \in H_i$ a $y \in H_j$ tak, že $i+j = 2k$, $k \in \N$.

V našem grafu existuje cesta $P_i$ z $v$ do $x$ o $i$ hranách, a zároveň existuje cesta $P_j$ z $v$ do $y$ o $j$ 
hranách.

Když vezmeme $P_i$, $\bc{x,y}$, $P_j$ (pozpátku), jedná se o \hyperref[stc]{uzavřený sled} o $i+j+1=2k+1$ hranách.

Dále máme kružnici o $i+j+1-2\ell$ hranách, kde $\ell$ je počet hran ve \colorbox{green!80!black}{společném úseku $P_i, 
P_j$}.

TODO: AŽ SE SEM DOPLNÍ OBRÁZEK, BUDE TO HNED JASNÝ.
\hspace{\fill}\qed

\subsection{Tvrzení o vztahu barevnosti grafu a nezávislosti grafu}
Pro každý neorientovaný graf $G$ bez smyček o $n$ vrcholech a $m$ hranách platí
\begin{enumerate}[(a)]
    \item $\alpha_0(G) + \chi(G) \leq n+1$;
    \item $\chi(G) \cdot \alpha_0(G) \leq n$;
    \item $\chi(G) \leq \frac{1}{2} + \sqrt{2m+\frac{1}{4}}$;
\end{enumerate}
kde $\alpha_0(G)$ je počet $|N|$ tak, že $N$ je nejpočetnější \hyperref[nezavislost]{nezávislá}.

\dukaz
\begin{enumerate}[(a)]
    \item Ať $b$ je optimální \hyperref[vrchChrom]{obarvení}. To rozděluje $G$ na nezávislé množiny dle barev. A všech 
    vrcholů v množině je nejvýše $\alpha_0(G)$, protože to je ta nejpočetnější. Sjednocením těchto množin pak dostaneme
    $\chi(G) \cdot \alpha_0(G) \leq n$. \hspace{\fill}\qed
    \item Ať
    \begin{align}
        &v \in N \quad \dots \quad b(n) = 1, \\
        &\begin{rcases}
            x_1 \\
            \vdots \\
            x_{n-\alpha_0(G)}
        \end{rcases} b(x_i) = i +1.
    \end{align}
    Má tedy $1 + n - \alpha_0(G)$ barev. Upravme
    \begin{align}
        1+n-\alpha_0(G) &\geq \chi(G) \\
        \alpha_0(G) + \chi(G) &\leq n+1 
    \end{align}
    \hspace{\fill}\qed
    \item Ať $b$ je optimální obarvení. Mějme množiny $A_1, A_2, \dots, A_k$, kde $k = \chi(G)$. Mezi každými 2 
    množinami je alepsoň jedna hrana, tj.
    \begin{equation}
        m \geq \frac{k(k-1)}{2}.
    \end{equation}
    Upravme:
    \begin{align}
        2m &\geq k^2 - k\\
        0 &\geq 2k^2 - k - 2m
    \end{align}
    Určeme kvadratické kořeny:
    \begin{equation}
        k_{1,2} = \frac{1 \pm \sqrt{1 + 8m}}{2} = \frac{1}{2} \pm \sqrt{\frac{1}{4} + 2m}
    \end{equation}
    A po analýze funkce víme, že 
    \begin{equation}
        \chi(G) \leq \frac{1}{2} + \sqrt{\frac{1}{4} + 2m}.
    \end{equation}
    \hspace{\fill}\qed
\end{enumerate}

\subsection{Tvrzení o největším stupni vrcholu a barevnosti grafu}
Označme $\Delta$ největší stupeň vrcholu grafu $G \in \S$. Pak
\begin{equation}
    \chi(G) \leq \Delta + 1.
\end{equation}
\dukaz Algoritmem sekvenčního barvení.

Označme $B = \bc{1,2, \dots, \Delta, \Delta + 1}$. 
Iterujme:
\begin{enumerate}[1)]
    \item Očíslujeme vrcholy grafu $G$, tj. $v_1, v_2, \dots, v_n$.
    \item Barvíme vrcholy v tomto pořadí, že $b(v_i) = j \iff j$ je nejmenší barva, kterou nemá žádný z již obarvených 
    sousedů. 
\end{enumerate}
\hspace{\fill}\qed

\subsection{Příklad použití algoritmu sekvenčního barvení}
% TODO: dodělat příklady s nákresy

\subsection{Věta o souvislosti největšího stupně vrcholu a barevnosti grafu}
\ii{Brooks}. Nechť $G$ je prostý souvislý graf bez smyček, který není ani úplný ani kružnice liché délky. Pak
\begin{equation}
    \chi(G) \leq \Delta(G).
\end{equation}
\dukaz
Omezme se na $\Delta(G) \geq 3$. Indukcí podle $|V| = n$.
\begin{enumerate}[(a)]
    \item \ii{Základní krok}, $n=4$.
        \begin{figure}[H]
            \centering
            \begin{tikzpicture}[scale=1]
                \graph [tree layout, orient=west]{
                    1 -- {2, 3, 4};
                    2 -- {3, 4};
                    
                };
            \end{tikzpicture}
        \end{figure} % TODO: dodělat nákres
    \item \ii{Indukční předpoklad}. Předpokládejme, že každý souvislý $G^\prime \in \S$, $|V^\prime| = n$, který není $K_n$ a má $\Delta(G^\prime) \geq 3$ se dá obarvit $\Delta(G^\prime)$ barvami.
    \item \ii{Indukční krok}. $G = (V, E)$, $|V| = n + 1$. Zvolme $u_0 \in V$, pak $G \setminus u_0 = H$ je graf o $n$ vrcholech.
    % TODO: dodělat důkaz
\end{enumerate}
