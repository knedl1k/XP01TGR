\section{Hamiltonovské grafy}
\subsection{Cesta, kružnice, cyklus}\label{hamKru}
Cesta (kružnice, cyklus) je hamiltonovská(ský), jestliže prochází všemi vrcholy.

\subsection{(Ne)orientovaný graf}
(Ne)orientovaný graf je Hamiltonovský, jestliže obsahuje hamiltonovský(skou) cyklus (kružnici).

\subsection{Chvátalova věta o Hamiltonovském grafu}\label{chvatal}
Máme $G \in \S$, $n \geq 3$, se skóre $d_1 \leq d_2 \leq \dots \leq d_n$, $n = |V|$. Jestliže pro nějaké $k < 
\frac{n}{2}$ platí $d_k \leq k$, a pak $d_{n-k} \geq n-k$, tak $G$ je Hamiltonovský.

Důkaz. Sporem. Předpokládejme, že existuje $G \in \S$, $n \geq 3$, $G$ splňuje $k < \frac{n}{2}$ platí $d_k \leq k$, a 
pak $d_{n-k} \geq n-k$ a $G$ není Hamiltonovský.

Když ke $G$ splňující Chvátalovu podmínku přidáme hranu, která v $G$ není, tak nový graf stále splňuje Chvátalovu 
podmínku.

Zvolíme $G$ s Chvátalovou podmínkou maximální bez Hamiltonovské kružnice. Zvolme $x,y: \, \bc{x,y} \not\in E$, $d(x) + 
d(y)$ je největší mezi $\bc{u,v} \not\in E$, $d(x) \leq d(y)$. V $G$ existuje Hamiltonovská cesta z $x$ do $y$.
% TODO: nákres
Označme
\begin{align*}
    S &\coloneq \bc{ i \mid \bc{x, v_{i+1}} \in E} \in 1, \\
    T &\coloneq \bc{ j \mid \bc{v_j, y} \in E} \ni n-1,
\end{align*}
kde $n \not\in S$, $n \not\in T$.
\begin{align*}
    S \cup T \subset \bc{1, \dots, n-1} \\
    |S| = d(x) \\
    |T| = d(y)
\end{align*}
\bb{Pomocný důkaz}. Platí $S \cap T = \emptyset$.\\
Důkaz. Kdyby $i \in S \cap T$, tak se jedná o hamiltonovskou kružnici. Spor s velkým předpokladem.
\hspace{\fill}\qed \\
Takže
\begin{equation*}
    |S \cup T| = |S| + |T| = d(x) + d(y) \leq n-1.
\end{equation*}
Položme
\begin{alignat*}{2}
   k &\coloneq d(x) &\qquad\qquad d(x) &\leq d(y) \\
   k &< \frac{n}{2} &             2d(x) &\leq d(x) + d(y) \leq n-1
\end{alignat*}
A tedy $d(x) \leq \frac{n-1}{2} < \frac{n}{2}$.

Máme $d(x)$ vrcholů $u$, že $\bc{x,y} \in E$.
\begin{align*}
    u &= v_l \\
    \bc{v_{l-1}, y} &\not\in E \\
    d(v_{l-1}) + d(y) &\leq d(x) + d(y) \\
    d(v_{l-1}) &\leq d(x) = k
\end{align*}
Tj. máme $k$ vrcholů $v_{l-1}$ s $d(v_{l-1}) \leq k$. Platí $d_k \leq k$, protože máme alespoň $k$ vrcholů stupně 
$\leq k$. Z Chvátalovy podmínky víme, že $d_{n-k} \geq n-k$, tj. existuje $k+1$ vrcholů $w$ s $d(w) \geq n-k$. Ale $x$ 
má $d(x) = k$. Tedy existuje $w$ s $d(w) = n-k$, že $\bc{x,w} \not\in E$.

$d(x) + d(w) = k+n-k = n$, což je spor s volbou $x$ a $y$, protože $d(x) + d(y) \leq n-1$.
\hspace{\fill}\qed

\subsection{Věta o skóre grafu a hamiltonovské kružnici}
Jestliže posloupnost čísel
\[ 
    a_1 \leq a_2 \leq \dots \leq a_n \, \text{ nesplňuje \hyperref[chvatal]{Chvátalovu podmínku}},
\]
tak existuje
\[
    d_1 \leq d_2 \leq \dots \leq d_n, \, a_i \leq d_i,
\]
tak, že je skóre grafu, který nemá \hyperref[hamKru]{hamiltonovskou kružnici}.