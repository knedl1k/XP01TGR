\section{Hamiltonovské grafy}
\subsection{Cesta, kružnice, cyklus}\label{hamKru}
Cesta (kružnice, cyklus) je hamiltonovská(ský), jestliže prochází všemi vrcholy.

\subsection{(Ne)orientovaný graf}\label{hamGraf}
(Ne)orientovaný graf je Hamiltonovský, jestliže obsahuje hamiltonovský(skou) cyklus (kružnici).

\subsection{Chvátalova věta o Hamiltonovském grafu}\label{chvatal}
Máme $G \in \S$, $n \geq 3$, se \hyperref[skore]{skóre} $d_1 \leq d_2 \leq \dots \leq d_n$, $n = |V|$. Jestliže pro nějaké $k < \frac{n}{2}$ platí $d_k \leq k$, a pak $d_{n-k} \geq n-k$, tak $G$ je \hyperref[hamGraf]{Hamiltonovský}.

\dukaz Sporem. Předpokládejme, že existuje $G \in \S$, $n \geq 3$, $G$ splňuje $k < \frac{n}{2}$ platí $d_k \leq k$, a pak $d_{n-k} \geq n-k$ a $G$ není Hamiltonovský.

Když ke $G$ splňující Chvátalovu větu přidáme hranu, která v $G$ není, tak nový graf stále splňuje Chvátalovu větu.

Zvolíme $G$ s Chvátalovou podmínkou maximální bez \hyperref[hamKru]{hamiltonovské kružnice}. Zvolme $x,y: \, \bc{x,y} \not\in E$, $d(x) + d(y)$ je největší mezi $\bc{u,v} \not\in E$, $d(x) \leq d(y)$. V $G$ existuje Hamiltonovská cesta z $x$ do $y$.
Označme
\begin{align}
    S &\coloneq \bc{ i \mid \bc{x, v_{i+1}} \in E} \in 1, \\
    T &\coloneq \bc{ j \mid \bc{v_j, y} \in E} \ni n-1,
\end{align}
kde $n \not\in S$, $n \not\in T$.
\begin{align}
    S \cup T \subset \bc{1, \dots, n-1} \\
    |S| = d(x) \\
    |T| = d(y)
\end{align}
\begin{figure}[H]
    \centering
    \begin{tikzpicture}[scale=1, transform shape]
        \foreach \i in {1, ..., 8} {
            \node[circle, fill=black, inner sep=1.5pt] (v\i) at (\i*1.5, 0) {};
        }
        
        \node[below=0.1cm of v1] {$x=v_1$};
        \node[below=0.1cm of v2] {$v_2$};
        \node[below=0.1cm of v3] {$v_{i}$};
        \node[below=0.1cm of v4] {$v_{i+1}$};
        \node[below=0.1cm of v5] {$\dots$};
        \node[below=0.1cm of v6] {$v_j$};
        \node[below=0.1cm of v7] {$v_{n-1}$};
        \node[below=0.1cm of v8] {$y=v_n$};

        \draw[thick] (v1) -- (v8);
        \draw[thick, blue, bend left=45] (v1) to node[midway, above] {$i \in S$} (v4);
        \draw[thick, red, bend right=45] (v6) to node[midway, below] {$j \in T$} (v8);

        \node[anchor=west, blue] at (1, 1.5) {$\exists \{x, v_{i+1}\} \in E \implies i \in S$};
        \node[anchor=east, red] at (11, -1.5) {$\exists \{v_j, y\} \in E \implies j \in T$};
    \end{tikzpicture}
\end{figure}
\ii{Pomocný důkaz}. Platí $S \cap T = \emptyset$.\\
\dukaz Kdyby $i \in S \cap T$, tak se jedná o hamiltonovskou kružnici. Spor s velkým předpokladem.
\hspace{\fill}\qed \\
Takže
\begin{equation}
    |S \cup T| = |S| + |T| = d(x) + d(y) \leq n-1.
\end{equation}
Položme
\begin{alignat}{2}
   k &\coloneq d(x) &\qquad\qquad d(x) &\leq d(y) \\
   k &< \frac{n}{2} &             2d(x) &\leq d(x) + d(y) \leq n-1
\end{alignat}
A tedy $d(x) \leq \frac{n-1}{2} < \frac{n}{2}$.

Máme $d(x)$ vrcholů $u$, že $\bc{x,y} \in E$.
\begin{align}
    u &= v_l \\
    \bc{v_{l-1}, y} &\not\in E \\
    d(v_{l-1}) + d(y) &\leq d(x) + d(y) \\
    d(v_{l-1}) &\leq d(x) = k
\end{align}
Tj. máme $k$ vrcholů $v_{l-1}$ s $d(v_{l-1}) \leq k$. Platí $d_k \leq k$, protože máme alespoň $k$ vrcholů stupně $\leq k$. Z Chvátalovy věty víme, že $d_{n-k} \geq n-k$, tj. existuje $k+1$ vrcholů $w$ s $d(w) \geq n-k$. Ale $x$ má $d(x) = k$. Tedy existuje $w$ s $d(w) = n-k$, že $\bc{x,w} \not\in E$.

$d(x) + d(w) = k+n-k = n$, což je spor s volbou $x$ a $y$, protože $d(x) + d(y) \leq n-1$.
\hspace{\fill}\qed

\subsection{Věta o skóre grafu a hamiltonovské kružnici}
Jestliže posloupnost čísel
\begin{equation} 
    a_1 \leq a_2 \leq \dots \leq a_n \, \text{ nesplňuje \hyperref[chvatal]{Chvátalovu větu}},
\end{equation}
tak existuje
\begin{equation} 
    d_1 \leq d_2 \leq \dots \leq d_n, \, a_i \leq d_i,
\end{equation}
tak, že je \hyperref[skore]{skóre} grafu, který nemá \hyperref[hamKru]{hamiltonovskou kružnici}.

\dukaz Jestliže posloupnost nesplňuje \hyperref[chvatal]{Chvátalovu větu}, pak existuje číslo $k$ tak, že $k \leq \frac{n}{2}$, $d_k \leq k$ a přitom $d_{n-k} \leq n-k-1$.

Utvořme graf s množinou vrcholů $\bc{v_1, v_2, \dots, v_n}$ takto: $\bc{v_i, v_j}$ je hrana $G$ právě tehdy, když
\begin{itemize}
    \item buď $1 \leq i \leq k$ a $n-k+1 \leq j \leq n$,
    \item nebo $i \not=j$ a $k+1\leq i$, $j \leq n$.
\end{itemize}
Jinými slovy $G$ se skládá z úplného bipartitního grafu se stranami $X = \bc{v_1, \dots, v_k}$ a $Y = \bc{v_{n-k+1}, \dots, v_n}$ a úplného grafu na množině vrcholů $\bc{v_{k+1}, v_{k+2}, \dots, v_n}$.

Graf $G$ má tedy
\begin{equation}
    a_1 \leq a_2 \leq \dots \leq a_k \leq a_{k+1} \leq \dots \leq a_{n-k} \leq a_{n-k+1} \leq \dots \leq a_n 
\end{equation}
\begin{equation}
    \underbrace{d_1}_k \leq \underbrace{d_2}_k \leq \dots \leq \underbrace{d_k}_k \leq \underbrace{d_{k+1}}_{n-k+1} \leq \dots \leq \underbrace{d_{n-k}}_{n-k+1} \leq \underbrace{d_{n-k+1}}_{n-1} \leq \dots \leq \underbrace{d_n}_{n-1}
\end{equation}
a tedy majorizuje $a_1, a_2, \dots, a_n$. Není těžké nahlédnout, že v $G$ neexistuje hamiltonovská kružnice.
\hspace{\fill}\qed

% TODO: dodělat příklad

\subsection{Turnaj}\label{turnaj}
Prostý orientovaný graf $G$ bez smyček nazveme \ii{turnajem}, jestliže pro každé dva různé vrcholy $u,v$ buď $(u,v)$ je hrana grafu $G$, nebo $(v,u)$ je hrana $G$; nikdy ale ne oboje.

Jinými slovy, zapomeneme-li na orientaci hran v grafu $G$, dostaneme úplný graf.

\subsection{Vztah hamiltonovských cyklů a silné souvislosti}
Je dán \hyperref[turnaj]{turnaj} $G$ s $n \geq 3$ vrcholy. Pak v $G$ existuje \hyperref[hamKru]{hamiltonovský cyklus} právě tehdy, když je $G$ silně \hyperref[souvisly]{souvislý}.

\dukaz
\begin{itemize}
    \item[$\Rightarrow$:] Nechť v $G$ existuje hamiltonovský cyklus $C$. Pro libovolné dva vrcholy $u, v$ existuje cesta z $u$ do $v$ vedená po hranách \hyperref[kruznice]{cyklu} $C$. Graf $G$ je tedy silně souvislý.
    
    \item[$\Leftarrow$:] Nechť $G = (V,E)$ je turnaj, který je silně souvislý. Ukážeme, že v $G$ existuje cyklus délky $k$ pro každé $3 \leq k \leq n$. Tím bude pro $k=n$ dokázáno, že je $G$ hamiltonovský. Důkaz vedeme indukcí podle $k$.
    \begin{enumerate}[1)]
        \item \ii{Základní krok $k=3$}. Vezměme libovolný vrchol $v \in V$. Označme $N^+ = \bc{x \in V \mid \bc{v,x} \in E}$ a $N^- = \bc{y \in V \mid \bc{y,v} \in E}$. Protože je $G$ silně souvislý, jsou obě množiny $N^+$ a $N^-$ neprázdné. Navíc musí existovat hrana vedoucí z $N^+$ do $N^-$, tj. vrcholy $x \in N^+$ a $y \in N^-$ takové, že $\bc{x,y} \in E$. (Kdyby taková hrana neexistovala, neexistovala by žádná \hyperref[stc]{cesta} z $N^+$ do $v$, což by byl spor se silnou souvislostí). Trojice $\bc{v,x}, \bc{x,y}, \bc{y,v}$ pak tvoří hledaný cyklus délky 3.
        \item \ii{Indukční předpoklad}. Předpokládejme, že máme zkonstruován cyklus $C$ s vrcholy $u_1, u_2, \dots, u_k$, pro $3 \leq k < n$.
        \item \ii{Indukční krok}. Pro rozšíření cyklu definujme množiny:
        \begin{equation}
             S^+ = \bc{w \notin C \mid \exists i, \bc{u_i, w} \in E} \quad \text{a} \quad S^- = \bc{w \notin C \mid \exists j, \bc{w, u_j} \in E}.            
        \end{equation}
        Protože $G$ je silně \hyperref[souvisly]{souvislý} a $k < n$, jsou obě množiny neprázdné (vrcholy mimo cyklus nemohou být izolované od cyklu). Rozlišíme dva případy:

        \ii{Případ A: $S^+ \cap S^- \not= \emptyset$.}
        Existuje vrchol $w$ mimo cyklus, do kterého vede hrana z cyklu a z něhož vede hrana do cyklu. Tedy existují $i, j$ tak, že $\bc{u_i, w} \in E$ a $\bc{w, u_j} \in E$. 
        \begin{itemize}
            \item Pokud $j = i+1$ (vrcholy jsou na cyklu sousední), vložíme $w$ mezi ně a máme cyklus $u_i \to w \to u_{i+1}$, jehož délka je $k+1$.
            \item Pokud $u_i$ a $u_j$ nejsou sousední, uvažujme vrchol $u_{i+1}$. V \hyperref[turnaj]{turnaji} musí existovat hrana mezi $w$ a $u_{i+1}$.
            \begin{itemize}
                \item Je-li $\bc{w, u_{i+1}} \in E$, našli jsme sousední dvojici $u_i, u_{i+1}$, mezi kterou vložíme $w$ (nahradíme hranu $\bc{u_i, u_{i+1}}$ cestou $u_i \to w \to u_{i+1}$).
                \item Je-li $\bc{u_{i+1}, w} \in E$, pak máme novou dvojici $\bc{u_{i+1}, w}$ a $\bc{w, u_j}$, kde se vzdálenost indexů na cyklu zmenšila.
            \end{itemize}
            Opakováním tohoto postupu v konečně mnoha krocích nalezneme sousední vrcholy na \hyperref[kruznice]{cyklu}, mezi které lze $w$ vložit, a získáme cyklus délky $k+1$.
        \end{itemize}

        \ii{Případ B: $S^+ \cap S^- = \emptyset$.}
        Množiny jsou disjunktní. To v turnaji znamená, že pro každé $z \in S^+$ platí $\bc{u, z} \in E$ pro \ii{všechny} $u \in C$ (jinak by existovala hrana $\bc{z, u}$, což by znamenalo $z \in S^-$). Analogicky pro $t \in S^-$ platí $\bc{t, u} \in E$ pro všechny $u \in C$.
    
        Protože je $G$ silně souvislý, musí existovat hrana z množiny $S^+$ do množiny $S^-$ (jinak by nebyla cesta z $S^+$ do $S^-$). Nechť tedy existují $z \in S^+$ a $t \in S^-$ takové, že $\bc{z, t} \in E$.
    
        Vybereme na cyklu $C$ tři po sobě jdoucí vrcholy, řekněme $u_1, u_2, u_3$ (kde $u_3$ následuje $u_2$ a $u_2$ následuje $u_1$).
        Víme, že existují hrany $\bc{u_1, z}$ (protože $z \in S^+$) a $\bc{t, u_3}$ (protože $t \in S^-$).
        V původním cyklu $C$ nahradíme cestu délky 2 ($u_1 \to u_2 \to u_3$) cestou délky 3 ($u_1 \to z \to t \to u_3$).
    
        Tím jsme do cyklu přidali dva vrcholy ($z, t$) a jeden odebrali ($u_2$). Výsledný cyklus má délku $k - 1 + 2 = k+1$.
    \end{enumerate}
    \hspace{\fill}\qed
\end{itemize}
