\section{Neorientované grafy}

\subsection{Základní pojmy a definice}
Graf je soubor vrcholů, hran a vztahů incidence. Zapíšeme jako $G = (V, E, \eps)$, kde $V$ je neprázdná množina vrcholů, 
$E$ množina hran a $\eps$ říká \enquote{co hrany představují}, respektive
\begin{equation}
    \eps: E \rightarrow \bc{\bc{u, v} \mid u, v \in V}.
\end{equation}

Jestliže pro dvě hrany $e_1, e_2 \in E$ platí, že $\eps(e_1) = \eps(e_2)$, pak se hrany $e_1, e_2$ nazývají 
\bb{paralelní}. Pokud graf nemá paralelní hrany, nazýváme jej \bb{prostý}. V takovém případě také stačí chápat graf jako 
dvojici $G = (V,E)$, kde hrany jsou neprázdné maximálně dvouprvkové podmnožiny $V$.\label{prosty}

\bb{Smyčkou} nazveme takovou hranu, která je $e \in E$ a pro $\eps(e) = \bc{u,v}$ platí $u=v$. \label{smycka}

$\S \dots$ je množina všech neorientovaných prostých grafů bez smyček.

\subsubsection{Základní typy grafů}
Rozlišujeme 2 základní typy grafů, orientované a neorientované.
\begin{enumerate}[(a)]
    \item Orientovaný graf: $\eps: E \rightarrow \bc{(u,v) \mid u,v \in V}$; $u \in P_V(\eps), v \in K_V(\eps)$
    \item Neorientovaný graf: $\eps: E \rightarrow \bc{(u,v) \mid u,v \in V}$; $u,v$ jsou krajní vrcholy $\eps$
\end{enumerate}

\subsubsection{Sled, tah, cesta}\label{stc}
\begin{enumerate}[(a)]
    \item Sled je taková posloupnost, která začíná a končí vrcholem a kde po každém vrcholu následuje 
    hrana, tedy $v_1, e_1, v_2, e_2, \dots, v_k$.\\
    V orientovaném případě vždy platí $P_V(e_1) = v_i$, $K_V(e_i)=v_{i+1}$. Neorientovaný pouze říká, že $v_i$ a 
    $v_{i+1}$ jsou krajní vrcholy.
    \item Tah je sled, ve kterém se nesmí opakovat hrany.
    \item Cesta je sled, ve kterém se nesmí opakovat vrcholy, s výjimkou počátečeního, ve kterém cesta 
    může končit.
\end{enumerate}

\subsubsection{Kružnice a cyklus}
\bb{Kružnice} je uzavřená \underline{ne}orientovaná cesta v grafu, \bb{cyklus} uzavřená orientovaná cesta.

Příklad kružnice:
\begin{figure}[H]
    \begin{tikzpicture}[scale=1]
        \graph [layered layout, orient=north]{
            1 -> [orient=45] 2 -> 3;
            1 -> 3;
        };
    \end{tikzpicture}
\end{figure}

\newpage
\subsubsection{Stupně vrcholů}
Pokud $G = (V, E, \eps)$, pak
\begin{itemize}
    \item vstupní stupeň v $d^- (v) = \left\|\bc{e \mid K_V(e) = v}\right\|$
    \item výstupní stupeň v $d^+ (v) = \left\| \bc{e \mid P_V(e)=V}\right\|$
    \item stupeň v $d(v) = d^-(v) + d^+(v)$
\end{itemize}
\begin{figure}[H]
    \centering
    \begin{minipage}[c]{0.32\textwidth}
        Příklad
        \begin{figure}[H]
            \begin{tikzpicture}[scale=1]
                \graph [layered layout, orient=north]{
                    1 -> [orient=45] 2 -> 3;
                    2 -> [in=120, out=60, looseness=6] 2;
                    1 -> 3;
                };
            \end{tikzpicture}
        \end{figure}
        
    \end{minipage}%
    \hspace{-0.1\textwidth}
    \begin{minipage}[c]{0.68\textwidth}
        \begin{align*}
            d^-(2) &= 2 \\
            d^+(2) &= 3 \\
            d(2) &= 5
        \end{align*}
    \end{minipage}
\end{figure}
Pro $G=(V,E)$ je pouze $d(v) = \left\|\bc{e \mid v \text{ je krajní vrchol } e, \text{ smyčku počítáme } 
2\times}\right\|$.

Z toho máme důsledek
\begin{equation}
    \sum_{v \in V} d(v) = 2 \|E\|
\end{equation}
Tedy každý graf má sudý počet vrcholů lichého stupně.

\subsection{Skóre}
Skóre grafu $\underset{\underset{\|V\|=d}{G=(V,E)}}{(G \in \S)}$ je $D = (d_1, d_2, \dots, d_n)$, kde $d_i$ je stupeň 
vrcholu $v_i$.

Mějme příklad skóre (1,1,1,2,2,3). Jak by mohl vypadat graf s takovým skóre?
\begin{figure}[H]
    \centering
    \begin{minipage}[c]{0.25\textwidth}
        \begin{figure}[H]
            \begin{tikzpicture}[scale=1]
                \graph [layered layout, orient=east]{
                    "$x$" -> {
                        "$y$",
                        "$t$" -> "$u$" -> "$v$",
                        "$z$"
                    };
                };

            \end{tikzpicture}
        \end{figure}
    \end{minipage}%
    \hspace{0\textwidth}
    nebo
    \hspace{0.05\textwidth}
    \begin{minipage}[c]{0.25\textwidth}
        \begin{figure}[H]
            \begin{tikzpicture}
                    \graph[layered layout, orient=east]{
                    "$x$" -> {
                        "$y$",
                        "$t$" -> "$u$",
                        "$z$" -> "$v$",
                    };
                };
            \end{tikzpicture}
        \end{figure}
    \end{minipage}
\end{figure}
Jak vidíme, skóre jednoznačně neurčuje graf. Můžeme ze skóre ale říct, jestli je takové skóre validním skóre nějakého 
grafu?
\newpage
\subsection{Hledání grafu ke skóre}

\bb{Tvrzení.} Máme $D = (d_1, d_2, \dots, d_n)$, $d_1 \leq d_2 \leq \dots \leq d_n$. \\
Pak $D$ je skóre některého grafu $G = (V,E)$ právě tehdy, když $D^\prime = (d_1^\prime, \dots, d_{n-1}^\prime)$ 
definovaná tak, že
\[
    d_i = 
    \begin{cases}
        d_i &\text{ pokud } i < n-d_n \\
        d_i-1  &\text{ pokud } i \geq n-d_n
    \end{cases}
\]
je skóre nějakého $G^\prime \in \S$.

\bb{Důkaz.}\\
\enquote{$\Leftarrow$}: Existuje $G^\prime$ pro $D^\prime$. $G$ vytvoříme tak, že k $G^\prime$ přidáme vrchol $v_n$ a 
spojíme se všemi vrcholy $v_{n-d_n}, v_{n-d_1+1}, \dots, v_{n-1}$. Pak $G$ má skóre $D$. \qed
\vspace{1em}

\enquote{$\Rightarrow$}: Máme $G$ s $D = (d_1, d_2, \dots, d_n)$, kde $d_1$ je stupeň $v_1$, $d_2$ je stupeň $v_2$ a tak 
dále.

Mějme $\G = \bc{G \mid G \text{ má } D} \not= \emptyset$.

\bb{Cíl}: Chceme dokázat, že mezi všemi grafy $\G$ existuje jeden, který má vlastnost, že poslední vrchol je spojen 
hranami s $d_n$ předcházejícími vrcholy.

$\forall G \in \G$ mějme $j_G$, což bude největší index vrcholu, tak že $\bc{v_{j_G}, v_n} \not\in E$, tedy není mezi 
nimi hrana. To znamená, že pro ideální $G$ chceme docílit $j_G = n-d_n-1$.

Jako $G_1$ označíme ten $G_1 \in \G$, že $j_{G_1}$ je nejmenší. (Může být $j_{G_1}$ menší jak $n-d_n-1$? Ne. $v_n$ má 
stupeň $d_n$, a kdyby bylo $j_{G_1}$ menší, tak by bylo vrcholů více, tzn. ne všechny by měly hranu s $v_n$.)

Označme $j_1 = j_{G_1}$.

Víme $j_1 \geq n-d_n-1$. Teď nás ale zajímá, jestli $j_1 = n - d_n - 1$. Dokažme sporem.\\
Kdyby $j_1 > n-d_n-1$, tak
\begin{figure}[H]
    \centering
    \begin{tikzpicture}[scale=1, dot/.style={circle, fill=black, inner sep=1pt, minimum size=4pt}]
        \node[dot, label=below:$v_1$] (v1) at (0.5,0) {};
        \node[dot, label=below:$v_2$] (v2) at (2,0) {};
        \node (d1) at (3.5,0) {$\dots$};
        \node[dot, label=below:$v_k$] (vk) at (4.5,0) {};
        \node[dot, label=below:$v_{n-d_n-1}$] (vn1) at (6,0) {};
        \node[dot, label=below:$v_{n-d_n}$] (vn2) at (7.5,0) {};
        \node[dot, label=below:$v_\ell$] (vl) at (9,0) {};
        \node (d2) at (10.5,0) {$\dots$};
        \node[dot, label=below:$v_{j_1}$] (vj1) at (12,0) {};
        \node (d3) at (13.5,0) {$\dots$};
        \node[dot, label=below:$v_{n-1}$] (vn3) at (15,0) {};
        \node[dot, label=below:$v_n$] (vn)  at (16.5,0) {};

        \draw[thick, purple, bend right] (vk) to (vn);
        \draw[thick, blue, bend left] (vl) to (vj1);
        \draw[thick, blue, decorate, decoration={crosses, segment length=4pt}, bend left] (vk) to (vl);
        \draw[thick, green!60!black, decorate, decoration={crosses, segment length=4pt}, bend left] (vj1) to (vn);
    \end{tikzpicture}
\end{figure}
\vspace{-1.5cm}
\[
    d(v_n) = d_n
\]
Protože mezi $d_n$ předcházejícími vrcholy je nějaký, který není spojen hranou s $v_n$, v našem případě $v_{j_1}$, nutně 
to znamená, že $v_n$ musí mít hranu s nějakým vrcholem, řekněme $v_k$, který má ještě nižší index.
\[
    \textcolor{purple}{d(v_k)} \leq \textcolor{green!60!black}{d(v_{j_1})}
\]
$v_k$ je v pořadí dříve, než $v_{j_1}$, tudíž musí mít nutně menší roven stupeň. To ale nutně znamená, že $v_{j_1}$ musí
být spojen s alespoň jedním vrcholem, označme si ho $v_\ell$, se kterým není spojen $v_k$, protože $v_k$ je spojen s 
$v_n$, zatímco $v_{j_1}$ není.

Vytvořme
\vspace{-0.5cm}
\begin{align*}
    G_0 &= (V_0, E_0) \\
    V_0 &= V_1 = V \\
    E_0 &= (E_1 \setminus \bc{\bc{v_n, v_k}, \bc{v_\ell, v_{j_1}}}) \cup \bc{\bc{v_k, v_\ell}, \bc{v_n, v_{j_1}}}
\end{align*}
$G_0$ má skóre $D$ a zároveň $j_{G_0} < j_1$. To ale znamená, že $G_1$ nebyl graf s nejmenším $j_G$, což je spor. A 
proto nejmenší $j_G$ je $j_{G_0} = n-d_n-1$.

Ověřili jsme, že takový graf určitě existuje, takže $G^\prime$ dostaneme z $G_0$ odstraněním $v_n$. $G^\prime$ pak má 
skóre $D^\prime$. 
\hspace{\fill}\qed

\subsection{Příklad hledání grafu pro skóre}
Mějme $D = (1,1,2,3,3)$; $n=5, d_n=3$; $n-d_n=2$.

$D_1 = (1,0,1,2)$ $\overset{\text{uspo.}}{\rightarrow} (0,1,1,2)$; $n_1=4, d_{n_1} = 2$; $n_1 - d_{n_1}=2$.

$D_2 = (0, 0, 0)$ $\dots$ tento graf je určitě existuje, jedná se o diskrétní graf.

Kresleme postupně, začněme u $D_2$.
\begin{figure}[H]
    \begin{tikzpicture}[scale=1]
        \graph [layered layout, orient=east]{
            "$x$";
            "$y$";
            "$z$";
        };
    \end{tikzpicture}
\end{figure}
Pak přidejme vrchol a hrany tak, aby skóre odpovídalo $D_1$.
\begin{figure}[H]
    \begin{tikzpicture}[scale=1]
        \graph [layered layout, orient=north]{
            "$y$";
            "$t$" -- {
                "$x$",
                "$z$",
            };
        };
    \end{tikzpicture}
\end{figure}
A nakonec tak, aby odpovídalo $D$.
\begin{figure}[H]
    \begin{tikzpicture}[scale=1]
        \graph [layered layout, orient=north]{
            "$u$" -- {
                "$y$",
                "$t$" -- {
                    "$x$",
                    "$z$",
                },
                "$z$",
            };
        };
    \end{tikzpicture}
\end{figure}

\subsection{Další pojmy založené na stupních vrcholů}
\bb{Definice.} Je dán neorientovaný prostý graf bez smyček. Pak definujme
\begin{itemize}
    \item $\delta(G) = \min\bc{d(v) \mid v \in V}$ je minimální stupeň grafu G.
    \item $\Delta(G) = \max\bc{d(v) \mid v \in V}$ je maximální stupeň grafu G.
    \item $d(G) = \frac{2|E|}{|V|} = \frac{\sum_{v \in V} d(v)}{|V|}$ je průměrný stupeň grafu G.
    \item $\eps(G) = \frac{|E|}{|V|} = \frac{1}{2}d(G)$ je poměr počtu hran ku počtu vrcholů.
\end{itemize}
Označme $n = |V|$ a $m = |E|$. Pak $d(G) = \frac{2m}{n}$ a $\eps(G) = \frac{m}{n}$. 

Zřejme platí $\delta(G) \leq d(G) \leq \Delta(G)$.

\subsection{Tvrzení o podgrafech}
\bb{Tvrzení.} Pro každý $G \in \S$ s $|E| \geq 1$ existuje podgraf $H$ takový, že $\delta(H) > \eps(H) \geq \eps(G)$.

\bb{Důkaz.} Máme dvě situace
\begin{enumerate}
    \item Buď $\delta(G) > \eps(G)$, pak $H=G$.
    \item Nebo $\delta(G) \leq \eps(G)$, tj. $v_1 \in V$, $d(v_1) = \delta(G) \leq \frac{m}{n}$.
\end{enumerate}
Dokažme tedy ještě platnost pro 2.

Označme $G_1 \coloneq G \setminus v_1$. A tedy $m_1 = m - \delta(G)$ a $n_1 = n - 1$.

Chceme $\underbrace{\frac{m_1}{n_1}}_{\eps(G_1)} \geq \underbrace{\frac{m}{n}}_{\eps(G)}$.
\begin{equation}
    \frac{m_1}{n_1} - \frac{m}{n} = \frac{m-\delta(G)}{n-1} - \frac{m}{n} = \frac{nm - n\delta(G) - nm + m}{(n-1)n}
    = \frac{m-n\delta(G)}{(n-1)n}, \delta(G) \leq \frac{m}{n}, m \geq n \delta(G)
\end{equation}
A tedy
\begin{align*}
    m-n\delta(G) &\geq 0 \\
    n(n-1) &\geq 0
\end{align*}
Což dává
\begin{align*}
    m \geq n \delta(G), \text{ tj. } \eps(G_1) \geq \eps(G)
\end{align*}
Algoritmus dále pokračuje:
\[
\text{Pokud }
\begin{cases}
    \delta(G_1) > \eps(G_1), &\text{ tak } H \coloneq G_1, \\
    \delta(G_1) \leq \eps(G_1), &\text{ tak } v_2 \in V \setminus \bc{v_1}, d_{G_1}(v_2) = \delta(G_1).
\end{cases}
\]
A tedy $G_2 \coloneq G_1 \setminus v_2$, $\eps(G_2) \geq \eps(G_1)$. A takto postupně dále. Algoritmus končí a nikdy 
nedostaneme prázdný graf, díky předpokladu, že $G$ mělo alespoň jednu hranu, tedy $\eps(G)>0$.
\hspace{\fill}\qed

\subsection{Souislý graf}\label{souvisly}
Graf nazýváme souvislým, jestliže každé jeho dva vrcholy jsou spojeny neorientovanou cestou. 

\subsection{Pojmy založené na vzdálenosti}
\subsubsection{Vzdálenost}
Mějme $G \in \S$, $G = (V,E)$, $x, y \in V$.
Vzdálenost $x,y$ je $d_G(x,y)$, což značí počet hran v nejméně početné cestě z $x$ do $y$, když existuje cesta. Jinak
$d_G(x,y) = \infty$.

\subsubsection{Průměr}
Ať $G$ je \hyperref[souvisly]{souvislý}. Průměr $G$ je $\diam(G) = \max\bc{d_G(x,y) \mid x, y \in V}$.

\subsubsection{Excentricita}
Ať $G$ je \hyperref[souvisly]{souvislý}. Excentricita vrcholu $v \in V$ je $\ex(V) = \max\bc{d_G(v,x) \mid x \in V}$.

\subsubsection{Centrum}
Ať $v \in V$ je centrální $\rightarrow \ex(v)$ je nejmenší mezi $\ex(x), x\in V$. Centrum (staře \ii{střed}) grafu je
$C(G) = \bc{v \mid v \text{ je centrální}}$.
\begin{figure}[H]
    \centering
    \begin{minipage}[c]{0.3\textwidth}
        Uveďme si příklad. \\ Zde $C(G) = \bc{2,3,5}$.
    \end{minipage}%
    \hspace{0.05\textwidth}
    \begin{minipage}[c]{0.6\textwidth}
        \begin{figure}[H]
            \begin{tikzpicture}[scale=1]
                \graph [layered layout, orient=west]{
                    1 -- 2;
                    3 -- {
                        4,5
                    };
                    2 -- {
                        3,5
                    };
                };
            \end{tikzpicture}
        \end{figure}
    \end{minipage}
\end{figure}

\subsubsection{Poloměr}
Poloměr $G$ je $\rad(G) = \ex(v), v \in C(G)$.

Platí $\rad(G) \leq \underbrace{\diam(G) \leq 2 \rad(G)}_{\star}$.

\bb{Zdůvodnění} $\star$. Chceme $d_G(x,y) \leq 2 \rad(G) \forall x,y \in V$.
\begin{figure}[H]
    \begin{tikzpicture}[rounded corners, scale=1]
        \graph []{
            "$x$" --["$P_1$"] "$v$" --["$P_2$"] "$y$"
        };
    \end{tikzpicture}
\end{figure}
$P_1, P_2$ sled z $x$ do $y$ o $\leq 2 \rad(G)$.

$P_1, P_2$ obsahuje cestu $P$ z $x$ do $y$ o $\leq P_1, P_2 \leq 2\rad(G)$.

\subsection{\texorpdfstring{$k$}{k}-souvislost}
$G = (V,E) \in \S$. Řekněme, že $G$ je $k$-souvislý, pokud $|V| > k$ a pro každou $X \subseteq V$, $|X| = k-1$ je 
$G \setminus X$ souvislý.\\
Mějmě
\begin{figure}[H]
    \centering
    \begin{minipage}[c]{0.3\textwidth}
        \begin{figure}[H]
            \begin{tikzpicture}[scale=1]
                \graph [layered layout, orient=west]{
                    1 -- {2, 3};
                    2 -- 4;
                    3 -- 4;
                    4 -- 5;
                };
            \end{tikzpicture}
        \end{figure}
        Je souvislý, ale ne 2-souvislý.
    \end{minipage}%
    \hspace{0.1\textwidth}
    \begin{minipage}[c]{0.3\textwidth}
        \begin{figure}[H]
            \begin{tikzpicture}[scale=1]
                \graph [tree layout, orient=west]{
                    1 -- {2, 4, 5};
                    2 -- 3;
                    3 -- 4;
                    4 -- 5;
                };
            \end{tikzpicture}
        \end{figure}
        Je 2-souvislý.
    \end{minipage}
\end{figure}
Každý graf je $0$-souvislý, i nesouvislý graf je $0$-souvislý.\\
$1$-souvislý je každý souvislý graf.

\subsection{Souvislost v grafu}
Souvislost v grafu $G$ je největší $k$ takové, že $G$ je $k$-souvislý. Značíme $\kappa(G)$. \\ Úplný graf má 
$\kappa(G) = |V|-1$.

\subsection{Vrcholový řez}\label{vrchRez}
Vrcholový řez grafu $G \in \S$ je množina vrcholů $X \subsetneqq V$, že $G \setminus X$ je nesouvislý.

\subsection{Vztah neúplnosti a vrcholového řezu}
Je-li $G \in S$, $G$ není úplný, pak $\kappa(G)=k$ právě tehdy, když nemá vrcholový řez o $k-1$ vrcholech a má vrcholový 
řez o $k$ vrcholech.

\subsection{Věta o vztahu podgrafu a souvislosti}
Mějme $G \in \S$, $G = (V,E)$, splňující $d(G) \geq 4k$. Pak $G$ obsahuje podgraf, který je $k$-souvislý.

\bb{Důkaz}. 
\begin{itemize}
    \item Pro $k=0$ triviální. Všechny grafy jsou $0$-souvislé.
    \item Pro $k=1$: Pokud $\frac{2m}{n} \geq 4k$, tedy $m \geq 1$ (takže má hranu), tak sama hrana je $1$-souvislý 
    podgraf.
    \item Pro $k \geq 2$: tj. $\frac{2m}{n} \geq 4k$
    \begin{align*}
        2m &\geq 4kn \\
        m &\geq 2kn \\
        m &\geq 4n \, (\text{dosazeno } k \geq 2)
    \end{align*}
\end{itemize}
Průběh důkazu $d(G) \geq 4k, k\geq 2$ $\xRightarrow{\text{Lemma 1}} (\text{i}), (\text{ii}) \xRightarrow{\text{Lemma 2}} 
G \text{ má } k \text{-souislost.}$
\subsubsection{Pomocné lemma 1}
Pokud $k \geq 2$ a $d(G) \geq 4k$, pak
\begin{enumerate}[(i), noitemsep]
    \item $n \geq 2k-1$
    \item $m \geq (2k-3)(n-k+1)+1$
\end{enumerate}
\bb{Důkaz}. (i) Kdyby ne, tak $n<2k-1$.
\begin{align*}
    n+1 &< 2k\\
    \frac{n+1}{2} &<k
\end{align*}
Teď použijme předpoklad $m \geq 2kn > (n+1)n$ % TODO: dodělat argumentaci, jaký předpoklad? nahrávka

(ii) Mějme
\begin{align*}
    m \geq 2kn - ((2k-3)(n-k+1)+1) &= 2kn - (2kn - 2k^2 + 2k - 3n+ 3k-3+1)\\
    &= 2k^2 - 5k + 3n + 2
\end{align*}
Teď aplikujme již dokázané (i):
\begin{align*}
    2k^2 - 5k + 3n + 2 \geq 2k^2 - 5k + 6k - 3 + 2 = 2k^2 + k - 1 
\end{align*}
\begin{figure}[H]
    \begin{minipage}[c]{0.4\textwidth}
        Vyšetřeme průběh funkce
        \begin{figure}[H]
            \begin{tikzpicture}[>=latex, scale=1]
                % x axis
                \draw[->] (-2.5,0) -- (2.5,0) node[below] {$k$};
                % y axis
                \draw[->] (0,-1.5) -- (0,3.5) node[left] {$y$};

                \draw[thick, red, domain=-1.5:1.2, samples=200] 
                    plot (\x, {2*\x*\x + \x - 1});

                \foreach \x in {-2, -1, 0.5, 1, 2}
                    \draw[shift={(\x,0)}] (0pt,2pt) -- (0pt,-2pt) node[below] {\footnotesize $\x$};
                \foreach \y in {-1, 1, 2, 3}
                    \draw[shift={(0,\y)}] (2pt,0pt) -- (-2pt,0pt) node[left] {\footnotesize $\y$};
                \node[below left] at (0,0) {\footnotesize $0$};
            \end{tikzpicture}
        \end{figure}
    \end{minipage}%
    \hspace{0.05\textwidth}
    \begin{minipage}[c]{0.5\textwidth}
        Funkce je očividně konvexní, a protože nás zajímá průběh funkce na $k \geq 2$, můžeme prohlásit, že 
        $2k^2 + k - 1 > 0$. \hspace{\fill} \qed
    \end{minipage}
\end{figure}

\subsubsection{Pomocné lemma 2}
Pokud $G$ splňuje (i) a (ii), tak $G$ má $k$-souvislý podgraf.

\bb{Důkaz}. $G$ není $k$-souvislý.\\
Indukcí podle $|V| = n$.

Základní krok: n $\stackrel{\text{(i)}}{=}2k-1$, $m \geq (2k-3)(n-k+1)+1$.\\
Dosaďme $k = \frac{n+1}{2}$:
\begin{equation}
    m \geq (n+1-3)\left(n-\frac{n+1}{2}+1\right)+1 = \frac{(n-2)(n+1)}{2}+1 = \frac{n(n-1)}{2}
\end{equation}
A tedy graf je úplný na $n$ vrcholech. Teď potřebujeme $n>k$.
\begin{equation}
    n = 2k-1 = k + \underbrace{k-1}_{\geq 1} \geq k+1
\end{equation}
Indukční krok: Každý graf $G^\prime$ splňující (i) a (ii) s méně než $n$ vrcholy (s alespoň $2k-1$ vrcholy) má 
$k$-souvislý podgraf.

Vezmeme $G$ splňující (i) a (ii) s $n$ vrcholy.
\begin{enumerate}[(a)]
    \item Kdyby $\delta(G) \leq 2k-3$, tak $v \in V$ s $d_G(v) \leq 2k-3$.\\
    $G \setminus v = G_1$, $n_1=n-1$,
    \begin{equation*}
        m_1 \geq m-(2k-3) \geq (2k-3)(n-k+1)+1 - (2k-3) = (2k-3)(\underbrace{n-1}_{n_1}-k+1)+1
    \end{equation*}
    Tudíž $G_1$ má $k$-souvislý podgraf, tedy i ho má $G$.
    \item Ať $\delta(G) > 2k-3m$, $\delta(G) \geq 2k-2$; $\forall v \in G, d_G(v)\geq 2k-2$.\\
    $G$ není $k$-souvislý, tj. $X \subseteq V$, $|X|=k-1$ a $X$ je \hyperref[vrchRez]{vrcholový řez}.
\end{enumerate}

% TODO: dodělat diagram

$G_1$ graf indukovaný $C_1$ v $X$ má alespoň $2k-1$ vrcholů.

Kdyby $G_1$ i $G_2$ nesplňovaly (ii), $G_i$ má $n_i$ vrcholů a $m_i$ hran, $i=1,2$.
\begin{equation}
    m_i \not\geq (2k-3)(n_i-k+1)+1, \hspace{1.5em} \text{tj. } m_i \leq (2k-3)(n_i-k+1)
\end{equation}
$m_1+m_2 \geq m$ víme. $n_1+n_2 = n+(k-1)$, počítali jsme vrcholy v $X$ dvakrát.
\begin{align*}
    m \leq n_1 + n_2 \leq (2k-3)(n_1-k+1)+(2k-3)(n_2-k+1) &= (2k-3)(n_1+n_2-2k+2) \\
    &= (2k-3)(n+(k-1)-2k+2) \\
    &= (2k-3)(n-k+1)
\end{align*}
Tedy spor s (ii). 
\hspace{\fill} \qed

\newpage
\subsection{Artikulace}\label{artikulace}
Vrchol v grafu $G$ se nazývá artikulace, jestliže $G \setminus v$ má více komponent souvislosti, než $G$.\\
\bb{Platí.} $G \in \S$ s alespoň 3 vrcholy je 2-souvislý $\iff$ je 1-souvislý a nemá artikulaci.
\begin{figure}[H]
    \centering
    \begin{minipage}[c]{0.3\textwidth}
        \begin{figure}[H]
            \begin{tikzpicture}[scale=1]
                \graph [tree layout, orient=south]{ 
                    1 -- {2,3};
                    2 -- 3;
                    3 -- {4,5};
                    4 -- {5,6};
                    5 -- 6;
                };
            \end{tikzpicture}
        \end{figure}
        Není 2-souvislý.
    \end{minipage}%
    \hspace{0.1\textwidth}
    \begin{minipage}[c]{0.3\textwidth}
        \begin{figure}[H]
            \begin{tikzpicture}[scale=1]
                \graph [spring layout, node distance=1cm]{
                    1 -- {2, 5, 6};
                    2 -- {3, 6};
                    3 -- 4;
                    4 -- {5, 6};
                    5 -- 6;        
                };
            \end{tikzpicture}
        \end{figure}
        Je 2-souvislý.
    \end{minipage}
\end{figure}

\subsection{Operace nad 2-souvislými grafy}
Mějme operace
\begin{enumerate}[(a)]
    \item $G \in \S$ a $e \in \bc{u,v}$; $u, v \in V(G)$, $e \not\in E(G)$, pak
    $G + e$ je graf s $V(G)$ a $E(G) \cup \bc{e}$. \\ Je-li $G$ 2-souvislý, tak $G+e$ je 2-souvislý.
    \item $G \in \S$, $G = (V,E)$, $e \in E$, pak
    $G \% e = \left(V \cup \bc{w}, (E \setminus \bc{e}) \cup \bc{e_1, e_2}\right)$. \\
    \enquote{Do hrany $e$ vložíme vrchol se stupněm 2.}
\end{enumerate}

\subsection{Tvrzení o 2-souvislých grafech a kružnicích}
Každý 2-souvislý graf obsahuje kružnici. 

\bb{Důkaz.} Každý 2-souvislý graf je souvislý. Kdyby souvislý neobsahoval kružnici, jedná se o strom. A každý strom s 
alespoň 3 vrcholy má artikulaci. Protože stromy nemohou být 2-souvislé, a zároveň všechny ostatní souvislé grafy 
obsahují kružnici, i každý 2-souvislý graf obsahuje kružnici.
\hspace{\fill}\qed

\subsection{Věta o vrcholech na společné kružnici}
$G \in \S$, $G = (V,E)$ je 2-souvislý právě tehdy, když každé 2 vrcholy $u \not= v$ leží na společné kružnici.

\bb{Důkaz.}\\
\enquote{$\Leftarrow$}: Předpokládejme, že pro každé $u \not= v$ existuje kružnice $K$, která je obsahuje.

To znamená, že graf je souvislý. Musíme ještě dokázat, že v něm neexistuje \hyperref[artikulace]{artikulace}.
Kdyby graf měl artikulaci $v$: 
\begin{figure}[H]
    \centering
    \begin{tikzpicture}[scale=1]
        \node[ellipse, draw, thick, minimum width=2.5cm, minimum height=1.5cm, 
            label={[xshift=-0.5cm, yshift=-0.8cm]$C_1$}, label={[xshift=0cm, yshift=-1.5cm]$x$}] (C1) at (-3,0) {};
        \node[ellipse, draw, thick, minimum width=2.5cm, minimum height=1.5cm, 
            label={[xshift=0.5cm, yshift=-0.8cm]$C_2$}, label={[xshift=0cm, yshift=-1.5cm]$y$}] (C2) at (3,0) {};
        \node[circle, fill, inner sep=1.5pt, label=below:$v$] (v) at (0,0) {};

        \draw (C1.east) -- (v);
        \draw (C2.west) -- (v);

        \node[anchor=west] at (1, 2.5) {komponenty souvislosti};
        \node[anchor=west] at (2.2, 2) {$G \setminus v$};

        \draw[-Stealth, thick] (1.5, 2.2) -- (-2, 0.8);

        \node[anchor=west, text width=5cm] at (5, -0.5) {alespoň jeden vrchol};
        \node[anchor=west, text width=5cm] at (5.2, -1.0) {$u \in C_1$ a $C_2$.};
    \end{tikzpicture}
\end{figure}
Znamenalo by to, že v jedné komponentě souvislosti by ležely alespoň 2 vrcholy (protože máme minimálně 3 vrcholy). 
Zároveň ale vrchol $x \in C_1$ a $y \in C_2$ rozhodně neleží na společné kružnici, tudíž graf nemůže mít artikulaci, 
takže $G$ je 2-souvislý. 
\hspace{\fill}\qed

\enquote{$\Rightarrow$}: Předpokládejme, že $G$ je 2-souvislý. Dokažme indukcí podle vzdálenosti $d(u,v)$.
\begin{enumerate} [(a)]
    \item Základní krok: $u, v$ s $d(u,v)=1$. \\
    Budeme se snažit ukázat, že když zrušíme hranu, souvislost zůstane. 
    \begin{enumerate}[(1)]
        \item $G \setminus e$ je souvislý. Kdyby ne, tak
        \begin{figure}[H]
            \centering
            \begin{tikzpicture}[scale=1]
                \node[ellipse, draw, thick, minimum width=2.5cm, minimum height=1.5cm, 
                    label={[xshift=-0.5cm, yshift=-0.8cm]$C_1$}, label={[xshift=0cm, yshift=-1.5cm]$x$}] (C1) at (-3,0) {};
                \node[ellipse, draw, thick, minimum width=2.5cm, minimum height=1.5cm, 
                    label={[xshift=0.5cm, yshift=-0.8cm]$C_2$}, label={[xshift=0cm, yshift=-1.5cm]$y$}] (C2) at (3,0) {};

                \coordinate (StartEdge) at (-1.5, 0);
                \coordinate (EndEdge) at (1.5, 0);

                \draw[thick, dashed, gray] (C1.east) -- (C2.west);

                \coordinate (CrossCenter) at (0, 0);
                
                \draw[very thick, red] (CrossCenter) ++ (-0.5, 0.5) -- ++ (1, -1);
                \draw[very thick, red] (CrossCenter) ++ (-0.5, -0.5) -- ++ (1, 1);
            \end{tikzpicture}
        \end{figure}
        Přitom $G$ má alespoň 3 vrcholy, tedy v jedné komponentě leží alespoň 2 vrcholy. \ii{BÚNO} exsistuje 
        $x \in C_1$, $x \not= u$, tj. $u$ je \hyperref[artikulace]{artikulace}. Což je spor % TODO: s čím? doplnit
        Takže $G \setminus e$ je souvislý. Tedy existuje cesta $P$ z $u$ do $v$. Pak $P$ je kružnice obsahující $u,v$. 
    \end{enumerate}
    \item Indukční předpoklad: Pro každé $x,y$ s $d(x, y) = n \geq 1$ existuje kružnice obsahující $x,y$.
    \item Indukční krok: Vezměme libovolné $u,v$ s $d(u,v) = n+1$. \\
    Vyberme nejkratší cestu:
    \begin{figure}[H]
        \centering
        \begin{tikzpicture}[scale=1]
            \node[circle, draw, fill=white, inner sep=1.5pt, label=left:$u$] (u_node) at (-4, 1) {};
            \node[circle, draw, fill=white, inner sep=1.5pt, label=right:$v$] (v_node) at (4, 1) {};
            \node[circle, fill, inner sep=1.5pt, label=below:$x_1$] (x1_node) at (-1.5, 1) {};
            \node[circle, fill, inner sep=1.5pt, label=below:$x_n$] (xn_node) at (1.5, 1) {};
            \node[circle, draw, fill=green!50!black, inner sep=1.5pt, label=below:$w$] (w_node) at (0, -2) {};
            \draw[thick] (u_node) -- (x1_node);
            \node at (0, 1) {$\dots$};
            \draw[thick] (xn_node) -- (v_node);
            % \draw[orange, thick] (u_node.north west) .. controls (-2.5, 2.5) and (2.5, 2.5) .. (v_node.north east);
            \draw[red, thick] (u_node.north) .. controls (-2.5, 2.0) and (2.5, 2.0) .. (v_node.north);
            \draw[green!70!black, thick] (u_node.north east) .. controls (-2.5, 1.5) and (2.5, 1.5) .. (v_node.north west);
            \draw[green!70!black, very thick, -] (u_node.south west) .. controls (-4.5, -0.5) and (-2.5, -2) .. (w_node.west);
            \node[below left=0.1cm of u_node, color=red] {$P$};
            \draw[red, very thick, -] (u_node.south) .. controls (-3, -1) and (-1.5, -2) .. (w_node.north west);
            \draw[orange, very thick] (v_node.south) .. controls (2, 3) and (0, 3) .. (x1_node.south);
            \draw[orange, very thick] (x1_node.south west) .. controls (-1.5, -2) and (-0.5, -1.5) .. (w_node.north);
            \draw[red, very thick, -] (w_node.north east) .. controls (1.5, -2) and (3, -1) .. (v_node.south);
            \node[below right=0.1cm of v_node, color=green!70!black] {$K_1$};
            \draw[green!70!black, very thick, -] (w_node.east) .. controls (1.5, -2.5) and (3.5, -2) .. (v_node.south east);
        \end{tikzpicture}
    \end{figure}
    Použijme I.P.: tj. existuje kružnice $K_1$ obsahující $x_1, v$. $x_1$ není \hyperref[artikulace]{artikulace}, tj. 
    existuje cesta $P$ z $u$ do $v$ neobsahující $x_1$. $w$ je prvním vrcholem cesty $P$, který leží v $K_1$ 
    % TODO: dodělat
\end{enumerate}
\hspace{\fill}\qed

\subsection{Tvrzení o 2-souvislých grafech a \% operaci}
$G \in \S$ je 2-souvislý právě tehdy, když $G \% e$, $e \in E(G)$ je 2-souvislý.

\bb{Důkaz.}\\
\enquote{$\Rightarrow$}: Předpokládejme, že $G$ je 2-souvislý, tj. souvislý a nemá \hyperref[artikulace]{artikulaci}.

Vrchol $w$, který vložíme do hrany $e$, není artikulace. A žádný jiný se nemohl stát artikulací, to by už museli být 
artikulací předtím, a tedy by se v prvé řadě nejednalo o 2-souvislý.
\hspace{\fill}\qed

\enquote{$\Leftarrow$}: Předpokládejme, že $G \% e$ je 2-souvislý, tj. každé 2 vrcholy leží na společné kružnici.
\[
    x, y \in V(G) \dots \text{existuje } K \text{ v } G \% e \text{ obsahující } x, y
    \begin{cases}
        K \text{ neobsahuje } e_1, e_2 & K \text{ je kružnice } G. \\
        \\
        K \text{ obsahuje } e_1, e_2 & \text{ z } K \text{ odstraníme } e_1, e_2, \\ 
        & \text{ nahradíme } e \text{ a máme } K^\prime.
    \end{cases}
\]
$K^\prime$ je kružnice v $G$. 
\hspace{\fill}\qed

\subsection{Algoritmus sestrojení 2-souvislého grafu}
Každý 2-souvislý graf $G \in \S$, $G = (V,E)$ je možné sestrojit postupem:
\[ G_0 \coloneq K \text{ je nějaká kružnice}\]
Máme-li $G_i$, že $G_i \not= G$, tak $G_{i+1}$ je $G_i$, ke kterému přidáme cestu $P$ (v $G$), která vede mezi 2 vrcholy
z $G_i$ a zároveň všechny vcholy této cesty nejsou v $G_i$.

\newpage
\subsection{Příklad sestrojení 2-souvislého grafu}
Mějme 2-souvislý graf, tj. bez artikulace: 
\begin{figure}[H]
    \centering
    \begin{tikzpicture}[scale=1]
        \graph [spring layout, node distance=1cm]{ 
            1 -- {2, 3, 4};
            2 -- {3, 5};
            3 -- {4, 6};
            4 -- 5;
            5 -- 6;
        };
    \end{tikzpicture}
\end{figure}
Začněme $G_0$:
\begin{figure}[H]
    \centering
    \begin{tikzpicture}[scale=1]
        \graph [spring layout, node distance=1cm, orient=north]{ 
            1 -- {2, 3};
            2 -- 3;
        };
    \end{tikzpicture}
\end{figure}
Přidáme cestu z 1 do 2, tedy $G_1$:
\begin{figure}[H]
    \centering
    \begin{tikzpicture}[scale=1]
        \graph [spring layout, node distance=1cm, orient=north]{ 
            1 -- {2, 3, 4};
            2 -- {3, 5};
            4 -- 5;
        };
    \end{tikzpicture}
\end{figure}
Teď přidáme cestu z 3 do 4, $G_2$:
\begin{figure}[H]
    \centering
    \begin{tikzpicture}[scale=1]
        \graph [spring layout, node distance=1cm, orient=north]{ 
            1 -- {2, 3, 4};
            2 -- {3, 4, 5};
            4 -- 5;
        };
    \end{tikzpicture}
\end{figure}
A posledně z 3 do 5, $G_3 = G$:
\begin{figure}[H]
    \centering
    \begin{tikzpicture}[scale=1]
        \graph [spring layout, node distance=1cm]{ 
            1 -- {2, 3, 4};
            2 -- {3, 5};
            3 -- {4, 6};
            4 -- 5;
            5 -- 6;
        };
    \end{tikzpicture}
\end{figure}
% TODO: přidat i druhý postup z fotky
\newpage
\subsection{Komponenty 2-souvislosti - blok}
Mějme $G \in \S$, $G = (V,E)$, pak $A \subseteq V(G)$ se nazývá \bb{blok}, jestliže je maximální podmnožina taková, že 
jí indukovaný podgraf je 2-souvislý. \\
\ii{Pozn. maximální v tomto kontextu neznamená nejpočetnější, nýbrž, že do takové podmnožiny již nelze přidat další 
vrchol.}
\begin{figure}[H]
    \centering
    \begin{tikzpicture}[scale=1] % TODO: dodělat kroužky ke komponentám souvislosti
        \graph [spring layout, node distance=1cm]{ 
            1 -- {2, 3};
            2 -- 3;
            3 -- {4, 6};
            4 -- {5,6};
            5 -- 6;
        };
    \end{tikzpicture}
\end{figure}
Když nejsou jednotlivé bloky vzájemně disjunktní, tak jejich průnik je \hyperref[artikulace]{artikulace}.

\subsection{Hranový řez}
Máme $G \in \S$, $G = (V,E)$, pak $G$ je $k$-hranově souvislý, jestliže neexistuje $F \subseteq E$, $|F| = k-1$, taková,
že $G \setminus F$ je nesouvislý.

Množině $F \subseteq E$, že $G \setminus F$ je nesouvislá, se říká hranový řez.

\subsection{Hranová souvislost}
Hranová souvislost grafu $G$, značíme $\lambda(G)$, je největší $k$, že $G$ je $k$-souvislý.\\
\ii{Pozn. největší znamená, že nemá hranový řez s $\lambda(G)-1$ hranami, ale má s $\lambda(G)$ hranami.}

% TODO: příklad z nahrávky

\subsection{Most}
Nazvěme most hranu $e\in E(G)$, že $\bc{e}$ je hranový řez.

\subsection{Souvislost krajních vrcholů a mostů} % TODO: dodělat důkaz
Každý most má alespoň jeden krajní vrchol, který je \hyperref[artikulace]{artikulace}.

\subsection{Základní vlastnosti hranově souvislých grafů}
\vspace{-1em}
\begin{flalign*}
    &G \text{ je } 0\text{-hranově souvislý pro každé } G. && \\
    &G \text{ je } 1\text{-hranově souvislý} \iff G \text{ je souvislý.} && \\
    &G \text{ je } 2\text{-hranově souvislý} \iff G \text{ je souvislý a nemá most.}&& \\
\end{flalign*}

\subsection{Tvrzení o hranové a vrcholové souvislosti}
Platí, že $\lambda(G) \leq \kappa(G)$.
% TODO: dodělat z nahrávky