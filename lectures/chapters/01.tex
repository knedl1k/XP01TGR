\section{Neorientované grafy}

\subsection{Základní pojmy a definice}
Graf je soubor vrcholů, hran a vztahů incidence. Zapíšeme jako $G = (V, E, \eps)$, kde $V$ je neprázdná množina vrcholů, 
$E$ množina hran a $\eps$ říká \enquote{co hrany představují}, respektive
\begin{equation}
    \eps: E \rightarrow \bc{\bc{u, v} \mid u, v \in V}.
\end{equation}

Jestliže pro dvě hrany $e_1, e_2 \in E$ platí, že $\eps(e_1) = \eps(e_2)$, pak se hrany $e_1, e_2$ nazývají 
\bb{paralelní}. Pokud graf nemá paralelní hrany, nazýváme jej \bb{prostý}. V takovém případě také stačí chápat graf jako 
dvojici $G = (V,E)$, kde hrany jsou neprázdné maximálně dvouprvkové podmnožiny $V$.\label{prosty}

\bb{Smyčkou} nazveme takovou hranu, která je $e \in E$ a pro $\eps(e) = \bc{u,v}$ platí $u=v$. \label{smycka}

$\S \dots$ je množina všech neorientovaných prostých grafů bez smyček.

\subsubsection{Základní typy grafů}
Rozlišujeme 2 základní typy grafů, orientované a neorientované.
\begin{enumerate}[(a)]
    \item Orientovaný graf: $\eps: E \rightarrow \bc{(u,v) \mid u,v \in V}$; $u \in P_V(\eps), v \in K_V(\eps)$
    \item Neorientovaný graf: $\eps: E \rightarrow \bc{(u,v) \mid u,v \in V}$; $u,v$ jsou krajní vrcholy $\eps$
\end{enumerate}

\subsubsection{Sled, tah, cesta}\label{stc}
\begin{enumerate}[(a)]
    \item Sled je taková posloupnost, která začíná a končí vrcholem a kde po každém vrcholu následuje 
    hrana, tedy $v_1, e_1, v_2, e_2, \dots, v_k$.\\
    V orientovaném případě vždy platí $P_V(e_1) = v_i$, $K_V(e_i)=v_{i+1}$. Neorientovaný pouze říká, že $v_i$ a 
    $v_{i+1}$ jsou krajní vrcholy.
    \item Tah je sled, ve kterém se nesmí opakovat hrany.
    \item Cesta je sled, ve kterém se nesmí opakovat vrcholy, s výjimkou počátečeního, ve kterém cesta 
    může končit.
\end{enumerate}

\subsubsection{Kružnice a cyklus}
\bb{Kružnice} je uzavřená \underline{ne}orientovaná cesta v grafu, \bb{cyklus} uzavřená orientovaná cesta.

Příklad kružnice:
\begin{figure}[H]
    \begin{tikzpicture}[scale=1]
        \graph [layered layout, orient=north]{
            1 -> [orient=45] 2 -> 3;
            1 -> 3;
        };
    \end{tikzpicture}
\end{figure}

\newpage
\subsubsection{Stupně vrcholů}
Pokud $G = (V, E, \eps)$, pak
\begin{itemize}
    \item vstupní stupeň v $d^- (v) = \left\|\bc{e \mid K_V(e) = v}\right\|$
    \item výstupní stupeň v $d^+ (v) = \left\| \bc{e \mid P_V(e)=V}\right\|$
    \item stupeň v $d(v) = d^-(v) + d^+(v)$
\end{itemize}
\begin{figure}[H]
    \centering
    \begin{minipage}[c]{0.32\textwidth}
        Příklad
        \begin{figure}[H]
            \begin{tikzpicture}[scale=1]
                \graph [layered layout, orient=north]{
                    1 -> [orient=45] 2 -> 3;
                    2 -> [in=120, out=60, looseness=6] 2;
                    1 -> 3;
                };
            \end{tikzpicture}
        \end{figure}
        
    \end{minipage}%
    \hspace{-0.1\textwidth}
    \begin{minipage}[c]{0.68\textwidth}
        \begin{align*}
            d^-(2) &= 2 \\
            d^+(2) &= 3 \\
            d(2) &= 5
        \end{align*}
    \end{minipage}
\end{figure}
Pro $G=(V,E)$ je pouze $d(v) = \left\|\bc{e \mid v \text{ je krajní vrchol } e, \text{ smyčku počítáme } 
2\times}\right\|$.

Z toho máme důsledek
\begin{equation}
    \sum_{v \in V} d(v) = 2 \|E\|
\end{equation}
Tedy každý graf má sudý počet vrcholů lichého stupně.

\subsection{Skóre}
Skóre grafu $\underset{\underset{\|V\|=d}{G=(V,E)}}{(G \in \S)}$ je $D = (d_1, d_2, \dots, d_n)$, kde $d_i$ je stupeň 
vrcholu $v_i$.

Mějme příklad skóre (1,1,1,2,2,3). Jak by mohl vypadat graf s takovým skóre?
\begin{figure}[H]
    \centering
    \begin{minipage}[c]{0.25\textwidth}
        \begin{figure}[H]
            \begin{tikzpicture}[scale=1]
                \graph [layered layout, orient=east]{
                    "$x$" -> {
                        "$y$",
                        "$t$" -> "$u$" -> "$v$",
                        "$z$"
                    };
                };

            \end{tikzpicture}
        \end{figure}
    \end{minipage}%
    \hspace{0\textwidth}
    nebo
    \hspace{0.05\textwidth}
    \begin{minipage}[c]{0.25\textwidth}
        \begin{figure}[H]
            \begin{tikzpicture}
                    \graph[layered layout, orient=east]{
                    "$x$" -> {
                        "$y$",
                        "$t$" -> "$u$",
                        "$z$" -> "$v$",
                    };
                };
            \end{tikzpicture}
        \end{figure}
    \end{minipage}
\end{figure}
Jak vidíme, skóre jednoznačně neurčuje graf. Můžeme ze skóre ale říct, jestli je takové skóre validním skóre nějakého 
grafu?
\newpage
\subsection{Hledání grafu ke skóre}

\bb{Tvrzení.} Máme $D = (d_1, d_2, \dots, d_n)$, $d_1 \leq d_2 \leq \dots \leq d_n$. \\
Pak $D$ je skóre některého grafu $G = (V,E)$ právě tehdy, když $D^\prime = (d_1^\prime, \dots, d_{n-1}^\prime)$ 
definovaná tak, že
\[
    d_i = 
    \begin{cases}
        d_i &\text{ pokud } i < n-d_n \\
        d_i-1  &\text{ pokud } i \geq n-d_n
    \end{cases}
\]
je skóre nějakého $G^\prime \in \S$.

\bb{Důkaz.}\\
\enquote{$\Leftarrow$}: Existuje $G^\prime$ pro $D^\prime$. $G$ vytvoříme tak, že k $G^\prime$ přidáme vrchol $v_n$ a 
spojíme se všemi vrcholy $v_{n-d_n}, v_{n-d_1+1}, \dots, v_{n-1}$. Pak $G$ má skóre $D$. \qed
\vspace{1em}

\enquote{$\Rightarrow$}: Máme $G$ s $D = (d_1, d_2, \dots, d_n)$, kde $d_1$ je stupeň $v_1$, $d_2$ je stupeň $v_2$ a tak 
dále.

Mějme $\G = \bc{G \mid G \text{ má } D} \not= \emptyset$.

\bb{Cíl}: Chceme dokázat, že mezi všemi grafy $\G$ existuje jeden, který má vlastnost, že poslední vrchol je spojen hranami s 
$d_n$ předcházejícími vrcholy.

$\forall G \in \G$ mějme $j_G$, což bude největší index vrcholu, tak že $\bc{v_{j_G}, v_n} \not\in E$, tedy není mezi 
nimi hrana. To znamená, že pro ideální $G$ chceme docílit $j_G = n-d_n-1$.

Jako $G_1$ označíme ten $G_1 \in \G$, že $j_{G_1}$ je nejmenší. (Může být $j_{G_1}$ menší jak $n-d_n-1$? Ne. $v_n$ má 
stupeň $d_n$, a kdyby bylo $j_{G_1}$ menší, tak by bylo vrcholů více, tzn. ne všechny by měly hranu s $v_n$.)

Označme $j_1 = j_{G_1}$.

Víme $j_1 \geq n-d_n-1$. Teď nás ale zajímá, jestli $j_1 = n - d_n - 1$. Dokažme sporem.\\
Kdyby $j_1 > n-d_n-1$, tak
\begin{figure}[H]
    \centering
    \begin{tikzpicture}[scale=1, dot/.style={circle, fill=black, inner sep=1pt, minimum size=4pt}]
        \node[dot, label=below:$v_1$] (v1) at (0.5,0) {};
        \node[dot, label=below:$v_2$] (v2) at (2,0) {};
        \node (d1) at (3.5,0) {$\dots$};
        \node[dot, label=below:$v_k$] (vk) at (4.5,0) {};
        \node[dot, label=below:$v_{n-d_n-1}$] (vn1) at (6,0) {};
        \node[dot, label=below:$v_{n-d_n}$] (vn2) at (7.5,0) {};
        \node[dot, label=below:$v_\ell$] (vl) at (9,0) {};
        \node (d2) at (10.5,0) {$\dots$};
        \node[dot, label=below:$v_{j_1}$] (vj1) at (12,0) {};
        \node (d3) at (13.5,0) {$\dots$};
        \node[dot, label=below:$v_{n-1}$] (vn3) at (15,0) {};
        \node[dot, label=below:$v_n$] (vn)  at (16.5,0) {};

        \draw[thick, purple, bend right] (vk) to (vn);
        \draw[thick, blue, bend left] (vl) to (vj1);
        \draw[thick, blue, decorate, decoration={crosses, segment length=4pt}, bend left] (vk) to (vl);
        \draw[thick, green!60!black, decorate, decoration={crosses, segment length=4pt}, bend left] (vj1) to (vn);
    \end{tikzpicture}
\end{figure}
\vspace{-1.5cm}
\[
    d(v_n) = d_n
\]
Protože mezi $d_n$ předcházejícími vrcholy je nějaký, který není spojen hranou s $v_n$, v našem případě $v_{j_1}$, nutně 
to znamená, že $v_n$ musí mít hranu s nějakým vrcholem, řekněme $v_k$, který má ještě nižší index.
\[
    \textcolor{purple}{d(v_k)} \leq \textcolor{green!60!black}{d(v_{j_1})}
\]
$v_k$ je v pořadí dříve, než $v_{j_1}$, tudíž musí mít nutně menší roven stupeň. To ale nutně znamená, že $v_{j_1}$ musí
být spojen s alespoň jedním vrcholem, označme si ho $v_\ell$, se kterým není spojen $v_k$, protože $v_k$ je spojen s 
$v_n$, zatímco $v_{j_1}$ není.

Vytvořme
\vspace{-0.5cm}
\begin{align*}
    G_0 &= (V_0, E_0) \\
    V_0 &= V_1 = V \\
    E_0 &= (E_1 \setminus \bc{\bc{v_n, v_k}, \bc{v_\ell, v_{j_1}}}) \cup \bc{\bc{v_k, v_\ell}, \bc{v_n, v_{j_1}}}
\end{align*}
$G_0$ má skóre $D$ a zároveň $j_{G_0} < j_1$. To ale znamená, že $G_1$ nebyl graf s nejmenším $j_G$, což je spor. A 
proto nejmenší $j_G$ je $j_{G_0} = n-d_n-1$.

Ověřili jsme, že takový graf určitě existuje, takže $G^\prime$ dostaneme z $G_0$ odstraněním $v_n$. $G^\prime$ pak má 
skóre $D^\prime$. \qed

\subsection{Příklad hledání grafu pro skóre}
Mějme $D = (1,1,2,3,3)$; $n=5, d_n=3$; $n-d_n=2$.

$D_1 = (1,0,1,2)$ $\overset{\text{uspo.}}{\rightarrow} (0,1,1,2)$; $n_1=4, d_{n_1} = 2$; $n_1 - d_{n_1}=2$.

$D_2 = (0, 0, 0)$ $\dots$ tento graf je určitě existuje, jedná se o diskrétní graf.

Kresleme postupně, začněme u $D_2$.
\begin{figure}[H]
    \begin{tikzpicture}[scale=1]
        \graph [layered layout, orient=east]{
            "$x$";
            "$y$";
            "$z$";
        };
    \end{tikzpicture}
\end{figure}
Pak přidejme vrchol a hrany tak, aby skóre odpovídalo $D_1$.
\begin{figure}[H]
    \begin{tikzpicture}[scale=1]
        \graph [layered layout, orient=north]{
            "$y$";
            "$t$" -- {
                "$x$",
                "$z$",
            };
        };
    \end{tikzpicture}
\end{figure}
A nakonec tak, aby odpovídalo $D$.
\begin{figure}[H]
    \begin{tikzpicture}[scale=1]
        \graph [layered layout, orient=north]{
            "$u$" -- {
                "$y$",
                "$t$" -- {
                    "$x$",
                    "$z$",
                },
                "$z$",
            };
        };
    \end{tikzpicture}
\end{figure}

\subsection{Další pojmy založené na stupních vrcholů}
\bb{Definice.} Je dán neorientovaný prostý graf bez smyček. Pak definujme
\begin{itemize}
    \item $\delta(G) = \min\bc{d(v) \mid v \in V}$ je minimální stupeň grafu G.
    \item $\Delta(G) = \max\bc{d(v) \mid v \in V}$ je maximální stupeň grafu G.
    \item $d(G) = \frac{\sum_{v \in V} d(v)}{|V|}$ je průměrný stupeň grafu G.
    \item $\eps(G) = \frac{|E|}{|V|} = \frac{1}{2}d(G)$ je poměr počtu hran ku počtu vrcholů.
\end{itemize}
Označme $n = |V|$ a $m = |E|$. Pak $d(G) = \frac{2m}{n}$ a $\eps(G) = \frac{m}{n}$. 

Zřejme platí $\delta(G) \leq d(G) \leq \Delta(G)$.