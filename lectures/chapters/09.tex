\section{Pokrývání}
\subsection{Nezávislé množiny, nezávislost}\label{nezavislost}
Mějme graf $G = (V,E) \in \S$. Množinu vrcholů $A \subseteq V$ nazveme \ii{nezávislou množinou}, jestliže v grafu
neexistuje hrana s oběma krajními vrcholy v $A$. 

\ii{Nezávislost grafu}, značíme ji $\alpha_0(G)$, je rovna počtu vrcholů v nejpočetnější nezávislé množině v $G$.

% TODO: nákres

\subsection{Vrcholové pokrytí}\label{vrchPok}
Je dán prostý neorientovaný $G = (V,E) \in \S$. Množina vrcholů $K \subseteq V$ se nazývá \ii{vrcholové pokrytí}, 
jestliže každá hrana $e \in E$ má alespoň jeden krajní vrchol v $K$.

Počet vrcholů v nejméně početném vrcholovém pokrytí v grafu $G$ značíme $\beta_0(G)$.

Jestliže $K$ je vrcholové pokrytí a $K^\prime \subseteq K$, tak $K^\prime$ je také vrcholové pokrytí.

% TODO: nákres

\subsection{Věta o vztahu vrcholového pokrytí a nezávislosti}
Je dán prostý neorientovaný graf $G = (V,E)$ bez smyček s $n$ vrcholy. Platí
\begin{equation}
    \alpha_0(G) + \beta_0(G) = n.
\end{equation}
\dukaz Máme $G = (V,E) \in \S$. $|V| = n$.\\
\enquote{$\geq$}: Uvažujme $N$ nezávislou, $|N| = \alpha_0(G)$. Pak $V \setminus N$ je \hyperref[vrchPok]{vrcholové 
pokrytí}. Což znamená
\begin{equation}
    |V \setminus N| = n - \alpha_0(G) \geq \beta_0(G).
\end{equation}
Takže
\begin{equation}
    n \geq \alpha_0(G) + \beta_0(G).
\end{equation}
\enquote{$\leq$}: Uvažujme vrcholové pokrytí $K$ s $|K| = \beta_0(K)$. Pak $V \setminus K$ je 
\hyperref[nezavislost]{nezávislá} množina. A
\begin{equation}
    |V \setminus K| = n - \beta_0(G) \leq \alpha_0(G).
\end{equation}
Tudíž
\begin{equation}
    n \leq \alpha_0(G) + \beta_0(G).
\end{equation}

Dostáváme tedy 
\begin{equation}
    n = \alpha_0(G) + \beta_0(G).
\end{equation}
\hspace{\fill}\qed

\subsection{Hranové pokrytí}\label{hranPok}
Je dán prostý neorientovaný graf $G = (V,E)$ bez smyček a izolovaných vrcholů. Množina hran $B \subseteq E$ se nazývá 
\ii{hranové pokrytí}, jestliže každý vrchol je krajním vrcholem alespoň jedné hrany z $B$.

Definujme $\beta_1(G)$ jako počet hran v nejméně početném hranovém pokrytí.

% TODO: nákres

\subsection{Věta o vztahu hranového pokrytí a maximálního párování}
Je dán prostý neorientovaný graf $G = (V,E) \in \S$ bez izolovaných vrcholů s $n$ vrcholy. Pak platí
\begin{equation}
    \alpha_1(G) + \beta_1(G) = n,
\end{equation}
kde $\alpha_1$ je počet hran v maximálním párování grafu $G$ (tj. $\alpha_1(G) = P_{\max}$).
\dukaz\\
\enquote{$\leq$}: Vezmeme nejméně početné \hyperref[hranPok]{hranové pokrytí} $B \subseteq E$.
Jak vypadají komponenty souvislosti $(U,B)$?
% TODO: nákres

Takže 
\begin{equation}
    \beta_1 = |B| = \underbrace{n_1 + n_2 + \dots + n_k}_{n} - k = n-k.    
\end{equation}
Mějme párování $P$, kde z každé komponenty souvislosti vezmeme jednu libovolnou hranu, takže $|P| = k$. Tedy máme 
$b_1(G) = n-k$ a $|P| = k \leq \alpha_1(G)$.
\begin{align}
    n = \beta_1(G) + k &\leq \beta_1(G) + \alpha_1(G) \\
    n &\leq \alpha_1(G) + \beta_1(G)
\end{align}

\enquote{$\geq$}: Mějme $P_{\max}$, párování s $|P_{\max} = \alpha_1(G)$.
% TODO: nákres

$2\alpha_1(G)$ vrcholů je pokryto $P_{\max}$. Nepokryto je $n - 2\alpha_1(G)$ vrcholů. $B = P_{\max}$ v $\bc{e \mid e 
\text{ je přidaná}}$.
\begin{align}
    |B| = \alpha_1(G) + n - 2\alpha_1(G) = n-\alpha_1(G) &\geq \beta_1(G) \\
    n &\geq \alpha_1(G) + \beta_1(G)
\end{align}

Dostáváme tedy
\begin{equation}
    n = \alpha_1(G) + \beta_1(G).
\end{equation}
\hspace{\fill}\qed