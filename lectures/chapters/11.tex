\section{Grafy a vektorové prostory}

\subsection{Těleso \texorpdfstring{$\Z_2$}{Z2}}\label{teleso}
$(\Z_2, +, \cdot)$ se skládá ze dvou prvků, tj. $\Z_2 = \bc{0, 1}$, a na této množině jsou definovány dvě operace:
\begin{figure}[H]
    \centering
    \begin{minipage}[c]{0.4\textwidth}
		\centering
		sčítání + \hspace{3mm} 
		\begin{tabular}{c|c c|}
		+ & 0 & 1 \\
		\hline
		0 & 0 & 1 \\
		1 & 1 & 0
		\end{tabular}
    \end{minipage}
	\hspace{-0.05\textwidth}
    \begin{minipage}[c]{0.4\textwidth}
		\centering
		násobení $\cdot$ \hspace{3mm}
		\begin{tabular}{c|c c|}
		$\cdot$ & 0 & 1 \\
		\hline
		0 & 0 & 0 \\
		1 & 1 & 1
		\end{tabular}
    \end{minipage}
\end{figure}

\subsection{Symetrická diference}\label{symDif}
Mějme neorientovaný graf $G = (V, E, \eps)$, $W_G = P(E) = \bc{x \mid x \subseteq E}$ je množina všech podmnožin množin hran $E$ ($P$ je potenční množina). Na $W_G$ definujme \ii{symetrickou diferenci}, značíme $\oplus$, tak, že
\begin{equation}
	X \oplus Y = (X \setminus Y) \cup (Y \setminus X) = (X \cup Y) \setminus (X \cap Y).
\end{equation}
Platí
\begin{enumerate}[(a)]
	\item \ii{Komutativita}. $X \oplus Y = Y \oplus X$.
	\item \ii{Inverz}. $X \oplus X = \emptyset$. Každý prvek je inverz sám k sobě.
	\item \ii{Neutrálnost}. $X \oplus \emptyset = X$.
	\item \ii{Asociativita}. $X \oplus (Y \oplus Z) = (X \oplus Y) \oplus Z$
\end{enumerate}
Když $(W_G, \oplus, \emptyset)$, pak se jedná o \ii{komutativní grupu} s neutrálním prvkem $\emptyset$, kde $X \oplus X = \emptyset$, tj. prvek $X$ je opačný sám k sobě.

Definujme násobení $\star$ prvků z $W_G$ skaláry 0 a 1 takto
\begin{equation}
1 \star X = X, \hspace{1em} 0 \star X = \emptyset.
\end{equation}
$(W_G, \emptyset, \star)$ tvoří vektorový prostor nad \hyperref[teleso]{tělesem} $\Z_2$.

\subsection{Charakteristická funkce očíslování}\label{charFunOcislovani}
Zvolme libovolné (pevné) očíslování hran $E$ grafu $G$, že $E = \bc{e_1, e_2, \dots, e_m}$. Pak každá podmnožina $X \subseteq E$ odpovídá své charakteristické funkci, tj. funkci $\chi_X: \bc{1, 2, \dots, m} \rightarrow \bc{0,1}$ definované
\begin{align}
	\chi_X(i) =
	\begin{cases}
		0, & e_i \not\in X \\
		1, & e_i \in X
	\end{cases}
\end{align}
Funkci $f_X$ si můžeme představit jako $m$-členný vektor
\begin{equation*}
	(a_1, a_2, \dots, a_m).
\end{equation*}
Přiřazení, které každé podmnožině $X \subseteq E$ přiřazuje charakteristickou funkci $\chi_X$ je vzájemně jednoznačné.
Navíc platí
\begin{align}
\emptyset &\leadsto (0, 0, \dots, 0) \\
E &\leadsto (1, 1, \dots, 1) \\
Y &\leadsto (b_1, b_2, \dots, b_m) \\
X \oplus Y &\leadsto (a_1 + b_1, a_2 + b_2, \dots, a_m + b_m), \, \text{kde } + \text{ je z } \Z_2
\end{align}
\subsection{Tvrzení o isomorfnosti \texorpdfstring{$W_G$}{WG}}
$(W_G, \oplus, \cdot)$ je isomorfní s $\Z_2^m$, kde $m$ je počet hran grafu $G$.

\dukaz
Tvrzení nám říká
\begin{equation}
	(W_G, \oplus, \cdot) \leadsto (Z_2^m, +, \cdot).
\end{equation}
A z \hyperref[charFunOcislovani]{charakteristické funkce očíslování} víme, že to znamená
\begin{equation}
	(a_1, \dots, a_m) + (b_1, \dots, b_m) = (a_1 + b_1, \dots, a_m + b_m).
\end{equation}
\hspace{\fill}\qed
