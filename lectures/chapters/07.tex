\section{Toky v sítích}
\subsection{Síť}
Síť je prostý orientovaný graf bez smyček $G = (V,E)$. Máme zdroj $s$ (source) a spotřebič $t$ (target).

Zapisujeme síť $(G, l, c, s, t)$.

\subsection{Omezení toku}
Mějme $l, c: E \rightarrow \Z$, $l(e) \leq c(e)$; kde $l$ je dolní omezení toku a $c$ hhorní omezení (kapacita).

\subsection{Tok v síti}
Tok v síti je $f : E \rightarrow \Z$, že pro každý v $\not= s,t$ platí
\begin{equation}
    \sum_{e \in E^-(v)} f(e) = \sum_{e \in E^+(v)} f(e) \quad \text{(Kirchhoffův zákon)}
\end{equation}

\subsection{Různé vlastnosti sítí a toků}
Tok $f$ je přípustný pokud pro každou hranu $e \in E$ je $l(e) \leq f(e) \leq c(e)$. 
Přípustný tok obecně nemusí existovat; stačí například aby do některého vrcholu mohlo celkově přitéci méně, než z něho 
musí odtéci. Jestliže ale je dolní omezení v každé hraně nulové, tak přípustný tok vždy existuje.

Síť je transparentní pokud $l(e)=0 \forall e \in E$.

Velikost přípustného toku od $s$ do $t$ je
\begin{equation}
    \vel(f) = \sum_{e \in E^+(s)}f(e) - \sum_{e \in E^-(s)}f(e).
\end{equation}
Přípustný tok $f$ se nazývá \ii{maximální tok}, jestliže má největší velikost mezi všemi přípustnými toky.

\subsection{Řez oddělující zdroj od spotřebiče}
Je dána síť $G = (V,E)$ se zdrojem $z$, spotřebičem $s$, a omezeními $l, c$. Množinu vrcholů $A \subseteq V$ takovou, že 
$z \in A$, $s \not\in A$ nazýváme \ii{množina oddělující zdroj od spotřebiče}. Dále definujme $W^+(A)$ jako množinu hran 
vycházejících z množiny $A$, a $W^-(A)$ jako množinu hran vcházejících do množiny $A$. Přesněji
\begin{align}
    W^+(A) &= \bc{e \mid P_V(e)\in A, K_V(e) \not\in A},\\
    W^-(A) &= \bc{e \mid P_V(e)\not\in A, K_V(e) \in A}.
\end{align}
Množina 
\begin{equation}
    W(A) = W^+(A) \cup W^-(A)
\end{equation}
se nazývá \ii{řez určený množinou $A$}.

\ii{Kapacita řezu $W(A)$} je rovna
\begin{equation}
    \fcap(W(A)) = \sum_{e \in W^+(A)}c(e) - \sum_{e \in W^-(A)} l(e).
\end{equation}
\ii{Maximální řez} je řez $W(A)$ s nejmenší možnou hodnotou $\fcap(W(A))$.

\subsection{Tvrzení o tocích a řezech}
Pro každý řez $W(A)$ a každý tok $f$ platí
\begin{equation}
    \vel(f) = \sum_{e \in W^+(A)}f(e) - \sum_{e \in W^-(A)}f(e).
\end{equation}
Důkaz. % TODO: dodělat Důkaz

\subsection{Tvrzení o přípustných tocích a řezech}
Pro každý přípustný tok $f$ a řez $W(A)$ platí $\vel(f) \leq \fcap(W(A))$.

Důkaz.
\begin{equation}
    \vel(f) = \sum_{e \in W^+(A)}f(e) - \sum_{e \in W^- (A)}f(e) \leq \sum_{e \in W^+(A)}c(e) - 
    \sum_{c \in W^-(A)}l(e) = \fcap(W(A))
\end{equation}
Navíc každý přípustný tok splňuje $l(e) \leq f(e) \leq c(e)$.
\hspace{\fill}\qed

\subsection{Zlepšující cesta vůči toku \texorpdfstring{$f$}{f}}
Je dán přípustný tok $f$. Neorientovaná cesta v grafu $G$ od zdroje $z$ ke spotřebiči $s$ se nazývá \ii{zlepšující cesta 
vůči $f$}, jestliže
\begin{align}
    f(e) < c(e) &\quad \text{pro každou hranu cesty vpřed,}\\
    l(e) < f(e) &\quad \text{pro každou hranu cesty vzad.}
\end{align}
\ii{Kapacita zlepšující cesty} je 
\begin{equation}
    \fcap(C) = \min \left(\bc{c(e) - f(e) \mid e \in C \text{ vpřed}} \cup 
    \bc{f(e) - c(e) \mid e \in C \text{ vzad}}\right)
\end{equation}

\subsection{Změna toku \texorpdfstring{$f$}{f}}
Změna toku $f$ podél zlepšující cesty $C$ s kapacitou $d$ je tok $f^\prime$ definovaný
\begin{equation}
    f^\prime(e) = 
    \begin{cases}
        f(e) + d & \text{pro hrany $e$ cesty vpřed}\\
        f(e) - d & \text{pro hrany $e$ cesty vzad}\\
        f(e) & \text{pro hrany $e$ neležící na cestě}
    \end{cases}
\end{equation}

\subsection{Změna toku podle zlepšující cesty je přípustný tok}
Je-li $C$ zlepšující cesta kapacity $d$ vzhledem k přípustnému toku $f$, pak zlepšující tok $f^\prime$ je také 
přípustným tokem a $\vel(f^\prime) = \vel(f) + d$.

Důkaz. $f^\prime$ je tok, $v \in C$, $v \not= s,t$. % TODO: dodělat důkaz podle nahrávky.

\subsection{Značkovací procedura}
\bb{Vstup}: přípustný tok $f$.\\
\bb{Výstup}: zlepšující cesta $C$ vůči $f$, nebo odpověď \enquote{ne} a množina označkovaných vrcholů $A$.

\begin{enumerate}[1)]
    \item \ii{Inicializace}. Označkujeme zdroj $s$, ostatní vrcholy jsou bez značky.
    \item \ii{Test nalezení zlepšující cesty}. Jestliže byl označkován $t$, zpětným postupem zkonstruujeme zlepšující 
    cestu, kterou vrátíme.
    \item \ii{Značkování dopředu}. Jestliže existuje hrana $e$ taková, že $P_V(e)$ má značku, $K_V(e)$ nemá značku a 
    $f(e) < c(e)$, označkujeme $K_V(e)$; pro $K_V(e)$ si zapatujeme $e$. \verb+goto 2)+
    \item \ii{Značkování dozadu}. Jestliže existuje hrana $e$ taokvá, že $K_V(e)$ má značku, $P_V(e)$ nemá značku a 
    $l(e) < f(e)$, označkujeme $P_V(e)$; pro $P_V(e)$ so zapatujeme $e$. \verb+goto 2)+
    \item \ii{Neexistuje zlepšující cesta}. Nemůžeme-li již značkovat a nebyl označkován $t$, vrátíme odpověď 
    \enquote{ne} a množinu označkovaných vrcholů $A$.
\end{enumerate}

\subsection{Tvrzení o výsledku značkovací procedury}
Jestliže značkovací procedura skončila odpovědí \enquote{ne} a vrátila množinu označkovaných vrcholů $A$, pak
\begin{equation}
    \vel(f) = \fcap(W(A)).
\end{equation}
To znamená, že tok $f$ má maximální velikost a řez určený množinou $A$ má nejmenší kapacitu.

Důkaz. Mějme množinu označkovaných vrcholů $A$. $s \in A$ a $t \not\in A$. % TODO: dodělat důkaz; nahrávka + nákres
\begin{align}
    \vel(f) = \sum_{e \in W^+(A)}f(e) - \sum_{e \in W^-(A)}f(e) \\
    \vdots \\
    \sum_{e \in W^+(A)}c(e) - \sum_{e \in W^-(A)}l(e) = \fcap(W(A))
\end{align}

\subsection{Věta o přípustném toku a maximálním přípustném toku}
\ii{Ford-Fulkersonova věta.} Jestliže v síti $G= (V,E)$, s omezeními $l, c$, zdrojem $s$ a spotřebičem $t$ existuje 
přípustný tok, pak existuje maximální přípustný tok $f_{\max}$ a jeho vleikost je rovna kapacitě minimálního řezu.

\subsection{Přírůstková síť vzhledem k toku}
Mějme přípustný tok $f$ v $G = (V,E)$, s omezeními $l, c$, zdrojem $s$ a spotřebičem $t$. Pak síť $G = (V, E_f)$ je 
přírůstková síť toku $f$, se zdrojem $s$, spotřebičem $t$, nulovým dolním omezením a kapacitou $c_f$, kde
\begin{align}
    (u,v) \in E_f \text{ a } c_f(u,v) = c(e)-f(e) \quad \text{jestliže pro } e \in E, e=(u,v) \text{ a platí } 
    f(e) < c(e).\\ 
    (u,v) \in E_f \text{ a } c_f(u,v) = f(e)-l(e) \quad \text{jestliže pro } e \in E, e=(u,v) \text{ a platí } 
    l(e) < f(e). 
\end{align}
Každá zlepšující cesta od zdroje $s$ ke spotřebiči $t$ je zlepšující cestou v $G_f$ s tím, že všechny hrany cesty jsou 
hranami vpřed; tj. jedná se o orientovanou cestu v $G_f$.

\subsection{Vrstvená síť}
Mějme přírustkovou síť $G_f$. Pak její podgraf, který obsahuje všechny hrany některé nejkratší (na počet hran v cestě) 
zlepšující cesty od $s$ k $t$ se nazývá \ii{vrstvená síť}.

Sestrojit k dané přírůstkové síti vrstvenou síť je možné v čase úměrném počtu hran přírůstkového grafu $G_f$, a tudíž i 
původního grafu $G$. \\
\ii{Uvědomme si, že $G_f$ má nejvýše dvakrát tolik hran jako $G$.}

\subsection{Cirkulace}
Je dán orientovaný graf $G=(V,E)$ a omezení $l, c$. Zobrazení $f: E \rightarrow \Z$ se nazývá \ii{cirkulace}, jestliže 
Kirchhoffův zákon platí pro všechny vrcholy $v \in V$.

Cirkulace se nazývá \ii{přípustná}, jestliže navíc pro každou hranu $e \in E$ platí $l(e) \leq f(e) \leq c(e)$.
% TODO: nákres

\subsection{Řez, kapacita řezu}
Mějme síť $G = (V,E)$ s ohodnoceními $l,c$. Uvažujme neprázdnou množinu vrcholů $A \subset V$ takovou, že $A \not= V$. 
Pak kapacita řezu $W(A)$ je definovaná jako
\begin{align}
    \fcap(W(A)) = \sum_{e \in W^+(A)}c(e) - \sum_{e \in W^-(A)}l(e). 
\end{align}

\subsection{Souvislost přípustné cirkulace a řezu}
V síti $G = (V,E)$ s ohodnoceními $l,c$ existuje přípustná cirkulace právě tehdy, když neexistuje řez záporné kapacity.
\\ \ii{Přípustná cirkulace tedy existuje právě tehdy, když každá množina vrcholů $A$ má tu vlastnost, že tok, který do 
ní povinně musí vtéci kvůli dolnímu omezení $l$ na hranách z $W^-(A)$, může z této množiny také odtéci díky hornímu 
omezení $c$ na hranách z $W^+(A)$.}

Důkaz. Existuje-li přípustná cirkulace $f$, pak pro každý řez platí
\begin{equation}
    C(A) = \sum_{e \in W^+(A)}c(e) - \sum_{e \in W^-(A)}l(e) \geq \sum_{e \in W^+(A)}f(e) - \sum_{e \in W^-(A)}f(e) = 0.
\end{equation}
Jestliže naopak přípustná cirkulace neexistuje, pak existence řezu se zápornou kapacitou vyplne z algoritmu pro hledání
přípustné cirkulace.
